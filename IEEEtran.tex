%%
%% IEEEtran.cls 2005/09/13 version V1.6c
%%
%% NOTE: This text file uses MS Windows line feed conventions. When (human)
%% reading this file on other platforms, you may have to use a text
%% editor that can handle lines terminated by the MS Windows line feed
%% characters (0x0D 0x0A).
%% 
%% 
%% This is the official IEEE LaTeX class for authors of the Institute of 
%% Electrical and Electronics Engineers (IEEE) Transactions journals and
%% conferences.
%% 
%% The latest version and documentation of IEEEtran can be obtained at:
%% http://www.ieee.org
%% and/or
%% http://www.ctan.org/tex-archive/macros/latex/contrib/IEEEtran/
%% 
%% The CTAN page may have additional files related to obscure, 
%% non-IEEE standard and/or platform dependent use of this tool.
%%
%% Based on the original 1993 IEEEtran.cls, but with many bug fixes
%% and enhancements (from both JVH and MDS) over the 1996/7 version.
%%
%%
%% Contributors:
%% Gerry Murray (1993), Silvano Balemi (1993),
%% Jon Dixon (1996), Peter N"uchter (1996),
%% Juergen von Hagen (2000), and Michael Shell (2001-2005)
%% 
%% 
%% Copyright (c) 1993-2005 by Gerry Murray, Silvano Balemi, 
%%                         Jon Dixion, Peter N"uchter,
%%                         Juergen von Hagen and Michael Shell
%%
%% Current maintainer (V1.3 to V1.6): Michael Shell
%%                                    mshell@ece.gatech.edu
%%                                    See the CTAN website above
%%                                    for current contact information.
%%
%% Special thanks to Peter Wilson (CUA) and Donald Arseneau
%% for allowing the inclusion of the \@ifmtarg command 
%% from their ifmtarg LaTeX package. 
%% 
%%**********************************************************************
%% Legal Notice:
%% This code is offered as-is without any warranty either
%% expressed or implied; without even the implied warranty of
%% MERCHANTABILITY or FITNESS FOR A PARTICULAR PURPOSE!
%% User assumes all risk.
%% In no event shall IEEE or any contributor to this code
%% be liable for any damages or losses, including, but not limited to,
%% incidental, consequential, or any other damages, resulting from the
%% use or misuse of any information contained here.
%% 
%% All comments are the opinions of their respective authors and are not
%% necessarily endorsed by the IEEE.
%%
%% This code is distributed under the Perl Artistic License 
%% ( http://language.perl.com/misc/Artistic.html ) 
%% and may be freely used, distributed and modified.
%% Retain the contribution notices and credits.
%% 
%% Major changes to the user interface should be indicated by an 
%% increase in the version numbers. If a version is a beta, it will 
%% be indicated with a BETA suffix, i.e., 1.4 BETA.
%% Small changes can be indicated by appending letters to the version
%% such as "IEEEtran_v14a.cls".
%% In all cases, \Providesclass, any \typeout messages to the user,
%% \IEEEtransversionmajor and \IEEEtransversionminor must reflect the
%% correct version information.
%% The changes should also be documented via source comments.
%%**********************************************************************
%%
%
% Available class options 
% (e.g., \documentclass[10pt,conference]{IEEEtran} 
% 
%             *** choose only one from each category ***
%
% 9pt, 10pt, 11pt, 12pt
%    Sets normal font size. The default is 10pt.
% 
% conference, journal, technote, peerreview, peerreviewca
%    determines format mode - conference papers, journal papers,
%    correspondence papers (technotes), or peer review papers. The user
%    should also select 9pt when using technote. peerreview is like
%    journal mode, but provides for a single-column "cover" title page for
%    anonymous peer review. The paper title (without the author names) is
%    repeated at the top of the page after the cover page. For peer review
%    papers, the \IEEEpeerreviewmaketitle command must be executed (will
%    automatically be ignored for non-peerreview modes) at the place the
%    cover page is to end, usually just after the abstract (keywords are
%    not normally used with peer review papers). peerreviewca is like
%    peerreview, but allows the author names to be entered and formatted
%    as with conference mode so that author affiliation and contact
%    information can be easily seen on the cover page.
%    The default is journal.
%
% draft, draftcls, draftclsnofoot, final
%    determines if paper is formatted as a widely spaced draft (for
%    handwritten editor comments) or as a properly typeset final version.
%    draftcls restricts draft mode to the class file while all other LaTeX
%    packages (i.e., \usepackage{graphicx}) will behave as final - allows
%    for a draft paper with visible figures, etc. draftclsnofoot is like
%    draftcls, but does not display the date and the word "DRAFT" at the foot
%    of the pages. If using one of the draft modes, the user will probably
%    also want to select onecolumn.
%    The default is final.
%
% letterpaper, a4paper
%    determines paper size: 8.5in X 11in or 210mm X 297mm. CHANGING THE PAPER
%    SIZE WILL NOT ALTER THE TYPESETTING OF THE DOCUMENT - ONLY THE MARGINS
%    WILL BE AFFECTED. In particular, documents using the a4paper option will
%    have reduced side margins (A4 is narrower than US letter) and a longer
%    bottom margin (A4 is longer than US letter). For both cases, the top
%    margins will be the same and the text will be horizontally centered. 
%    For final submission to IEEE, authors should use US letter (8.5 X 11in)
%    paper. Note that authors should ensure that all post-processing 
%    (ps, pdf, etc.) uses the same paper specificiation as the .tex document.
%    Problems here are by far the number one reason for incorrect margins.
%    IEEEtran will automatically set the default paper size under pdflatex 
%    (without requiring a change to pdftex.cfg), so this issue is more
%    important to dvips users. Fix config.ps, config.pdf, or ~/.dvipsrc for
%    dvips, or use the dvips -t papersize option instead as needed. See the
%    testflow documentation
%    http://www.ctan.org/tex-archive/macros/latex/contrib/supported/IEEEtran/testflow
%    for more details on dvips paper size configuration.
%    The default is letterpaper.
%
% oneside, twoside
%    determines if layout follows single sided or two sided (duplex)
%    printing. The only notable change is with the headings at the top of
%    the pages.
%    The default is oneside.
%
% onecolumn, twocolumn
%    determines if text is organized into one or two columns per page. One
%    column mode is usually used only with draft papers.
%    The default is twocolumn.
%
% nofonttune
%    turns off tuning of the font interword spacing. Maybe useful to those
%    not using the standard Times fonts or for those who have already "tuned"
%    their fonts.
%    The default is to enable IEEEtran to tune font parameters.
%
%
%*******
% 09/2005 V1.6c (MDS) changes:
% 
% 1) Changed endfigure/endfloat definitions so as not to cause problems with
%    preview-LaTeX and other packages. Thanks to Stephan Heuel and David
%    Kastrup for reporting this problem.
%
%
%*******
% 11/2002 V1.6b (MDS) changes:
%
% 1) Fixed problem with figure captions when using hyperref. Thanks to 
%    Leandro Barajas and Michael Bassetti for reporting this bug.
%
% 2) Provide a fake nabib command \NAT@parse so that hyperref will not
%    interfere with the operation of cite.sty. However, as a result citation
%    numbers will not be hyperlinked. Also, natbib will not be able to work
%    with IEEEtran. However, this is perhaps the best solution until cite.sty
%    and hyperref.sty are able to co-exist with each other.
%    It easy enough to override the fake command via:
%    \makeatletter
%    \let\NAT@parse\undefined
%    \makeatother
%
% 3) Revised font selection method so as not to have problems when used
%    with setspace.sty. Thanks to Zhang Yan for reporting this bug.
%
% 4) Added \special to feed papersize to dvips. Thanks to Moritz Borgmann 
%    for suggesting this feature.
%
% 5) In addition to the IEEE IED lists, the original IED style list
%    environments (as is done in article.cls) are now provided as
%    LaTeXitemize, LaTeXenumerate, and LaTeXdescription. Also, users can
%    now redefine \makelabel within IEEE IED list controls. There may be
%    some use for this in specialized applications. Thanks to Eli Barzilay
%    for suggesting this feature.
%
% 6) \table* now defaults to \footnotesize text like \table.
%
% 7) The draft modes now no longer force a pagebreak after the title.
%    Thanks to Christian Peel for suggesting this change.
%
% 8) New draftclsnofoot mode is like draftcls, but does not display the
%    date and the word "DRAFT" at the foot of the page. Thanks to
%    Christian Peel for suggesting this feature.
%
% 9) New peerreview and peerreviewca modes with \IEEEpeerreviewmaketitle
%    command allows for a "cover" titlepage for anonymous peer review.
%    Except for the cover titlepage, peerreview is much like journal mode.
%    peerreviewca is like peerreview, but allows the author names to be
%    entered and formatted as under conference mode so that author
%    affiliations and contact information can be easily seen on the cover
%    page. Thanks to Eric Benedict for suggesting this feature.
%
%
%*******
% 7/2002 V1.6 (MDS) changes:
% 
% 1) Added conference mode via conference option. Defaults to the traditional
%    journal mode. i.e., \documentclass[conference]{IEEEtran}
% 
% 2) Added support for A4 paper via new a4paper option. Pdflatex's paper size
%    lengths are now automatically set to the proper paper size being used.
% 
% 3) Revised margins again. Page text is now horizontally centered.
%    Conference mode increases the top and bottom margins with the bottom
%    margin being slightly larger. For A4 paper, the top margin and text
%    typesetting will not change from those of US letter paper, but the side
%    margins will be smaller and the bottom margin will be larger than that of
%    US letter. All per IEEE specs.
%
% 4) Fixed footnote line spacing anomaly in draft mode. 
%    Thanks to Alberto Rodriguez for reporting this bug.
%    
%    Also, slightly revised footnote and \thanks note spacing.
%    Set \interfootnotelinepenalty=10000 to prevent LaTeX
%    from breaking footnotes across multiple pages or columns.
%
% 5) Fixed bug that caused overwritten photo areas and sometimes anomalous
%    spacing when a new paragraph was started within a biography. Also,
%    the presence of \par's, new lines or spaces at the beginning of
%    abstract, keywords, biography, or biographynophoto will no longer
%    affect the first word spacing.
%    Thanks to Eric Durant for reporting this bug.
%    
%    The biography environment now does a better job in preventing
%    a biography photo area from being broken across pages or columns.
%
% 6) Fixed whitespace between \cite entries bug. i.e.,
%    both \cite{einstein24, knuth84} and \cite{einstein24,knuth84}
%    are now valid. \cite is now a robust command as it should be.
%    IEEEtran now no longer defines the old non-standard \shortcite or 
%    \citename.
%    
%    The base IEEEtran.cls does not sort citation numbers or produce ranges
%    for three or more consecutive numbers. However, V1.6 of IEEEtran.cls 
%    now pre-defines the following format control macros to facilitate easy 
%    use with Donald Arseneau's cite.sty package (tested with cite.sty V3.9):  
%
%    \def\citepunct{], [}
%    \def\citedash{]--[}
%    
%    cite.sty is standard on most LaTeX sytems and can be obtained from
%    www.ctan.org. Thanks to Donald Arseneau for creating cite.sty, providing
%    the required format arguments to produce the IEEE style. and
%    designing a cite interface capable of handling the IEEE citation
%    style.
%    
%    Note: Historically, IEEE has wanted authors to "hardcode" symbolics.
%    (i.e., replace all \cite{} with fixed [x]). However, it now seems that
%    most electronic manuscript submissions to IEEE are in .pdf format, and 
%    as such, do not require the LaTeX document reference numbers to be hard
%    coded. If an author is required to submit actual LaTeX files, I do 
%    recommend that the bibliography file (.bbl) be copied into the .tex 
%    document and the \bibliographystyle{} and \bibliography{} commands be 
%    commented out so that the .tex file does not depend on (potentially 
%    lengthy and/or confidential) external bibliography database files.
%
% 7) Adjusted some spacing parameters. The spacing above and below equations
%    has been revised (to a typical IEEE value). \jot now has a decent value.
%    The title text is now exactly 24pt. (On a related note, \fontsubfuzz has
%    been increased to 0.9pt to prevent annoying font substitution warnings
%    when using the Computer Modern fonts that use the 24.88pt size.)
%    In V1.6, \small is now 8.5pt in 9pt docs because \footnotesize is 8pt. 
%    For 9pt docs, you should probably go ahead and use \footnotesize when you
%    need text a little smaller than \normalsize.
%    The interword spacing has been adjusted to be extremely close to that
%    which IEEE uses. You can use a new class option, nofonttune, if you need
%    to disable the adjusting of the interword spacing. This adjustment and
%    an increase to \hyphenpenalty have greatly reduced the amount of
%    hyphenation in a typical paper.
%       The baselineskip for the normalsize fonts has been tweaked to reduce
%    underfull vboxes on journal paper columns with only paragraphs. 
%    Conference mode does the same thing but by also tweaking the \textheight
%    slightly off 9.25in (IEEE spec) to ensure an integer number of lines per
%    page. Draft (also draftcls) mode has also been revised to reduce 
%    underfull vbox warnings. However, draft mode can still produce underfull
%    vboxes (a direct result of the increase in line spacing and margins) if:
%    A non-normalsize font occupies an entire column (abstract and index terms
%    take up a whole column by themselves); or the beginning of a section
%    occurs near the end of a column and cannot be squeezed into the bottom,
%    etc. This is normal as draft mode's liberal spacings cannot guarantee
%    perfect formatting.
%
% 8) New biographynophoto environment for biographies without photos.
%    Usage: 
%    
%    \begin{biographynophoto}{author name}
%    biography text here 
%    \end{biographynophoto}
%    
% 9) Fixed bug that produced multiple table of contents entries for papers
%    with more than one biography. Also, biography now works better with
%    hyperref.
% 
%10) New \sublargesize font size command provides for 11pt text in a 10pt
%    document. (Needed for things like author names.) For documents not
%    using 10pt normal size text, \sublargesize is currently identical 
%    to \large.
%
%11) New \IEEEmembership command to provide correct font to indicate IEEE
%    membership for journal papers.
%
%12) Fixed author name line overflow problem when in journal mode. This
%    problem had been introduced in V1.5 in my rush to get \and to work for
%    conferences. \and is unneeded (and invalid) in journal mode. For 
%    conference mode, \and will work as expected and features an optional
%    spacing argument. i.e., \and[\hspace{5ex}] 
%    \and will default (recommended) to using \hfill which will result in
%    equal spacing between author blocks.
% 
%13) New \authorblockN, \authorblockA and \authorrefmark commands to 
%    facilitate easy formatting of author names, affiliations and cross
%    reference symbols, respectively, when in conference mode. These 
%    three commands are to be used only for conference papers.
%    In conference mode, \author text is placed within a modified tabular
%    environment (somewhat like article.cls). So, within \author in conference
%    mode, you should not try to enclose multiple \\ within an environment or
%    command (other than the argument braces of \authorblockX{}). For example:
%    
%    \author{\authorblockN{{John Doe \\ Jane Doe}}} % WRONG!
%    
%    will generate an error. 
%  
%    Note that font size/attribute changes will now persists across \\ within
%    \author. (But, not across author blocks nor across \and.)
%    However, with the new commands, there should be no need to alter any
%    font attributes within \author. All text sizing and spacing within 
%    \author{} and the author block commands is per IEEE specs for both 
%    conference and journal modes. (In conference mode, the author names
%    are only very slightly larger than the affiliations which are in normal
%    size.) For specialized applications you can alter the justification of
%    author lines by placing \hfill at the beginning or at the end of a line.
%    The interline spacing within \author is determined by the font attributes
%    that are in effect at the end of each line within author.
% 
%14) Because the titles and author name blocks use different font sizes/styles
%    from the main text, it was possible that two column papers with titles that
%    span both columns (standard journal and conference papers, but not technotes)
%    with certain numbers of lines for the title and authors' name/affiliations
%    can cause underfull vbox problems (paragraphs with large spacings between
%    them) in the second column of the main text on the title page - if there were
%    no new sections, equations or figures in this column (they would provide some
%    needed rubber spacing). The use of things like special paper notices and
%    publisher ID marks also affected this issue. The problem could not happen
%    in the first column because the first column has a rubber length around the
%    heading of the first section. Furthermore, problems seldom occurred on pages
%    after the first as the margins had been chosen not to cause it with the popular
%    font sizes. Rubber lengths after the author names would not fix this problem.
%       Auto-calculating a "good" spacing after the title is a tad difficult
%    to do in LaTeX. However, I am pleased to report that V1.6 has this new
%    capability - "dynamically determined title spacing". IEEEtran will now
%    measure the height of all the title and author text in \maketitle
%    and then calculate a rigid (non-rubber) spacer to follow that meets
%    IEEE specs and also produces a \textheight on the title page that ensures
%    an integer number of normalsized lines on the rest of the page. Single
%    column  papers, and two column papers with the title entirely in column
%    one (technotes) do not need dynamic rigid spacing and therefore use
%    standard rubber spacers.
%    
%    Note: This problem can still crop up if you use floats that span both
%    columns (i.e., figure*). It has been a decade+ long limitation with LaTeX
%    that the stretchable portion of \dbltextfloatsep is ignored. 
%    If you get a problem with underful vbox warnings and paragraphs that "are
%    pulled apart" on page with a float that spans both columns, tweak the 
%    space between the figure and the main text a little:
%    
%    \vskip 5pt
%    \end{figure*}
%    
%    If you can't find a value that fixes both columns, you are going to
%    have to put a rubber spacer somewhere in one or both of the columns.
% 
%15) Because of change #14 above, those of you using \pubid will, as of V1.6,
%    have to place it *before* \maketitle in order for it have the intended 
%    affect. The dynamic spacer algorithm must see if you are using \pubid when
%    \maketitle is called. \pubidadjcol works as before except that it now 
%    has additional logic to prevent it from doing anything if \pubid was never
%    called.
%    
%16) In some unusual, non-standard circumstances, an author may desire to
%    alter the spacing after the title area or put some unusual text above
%    the main text. For instance, to stop a bad break when a new section
%    occurs right at the start of the second page. This is difficult to do
%    when the title spans both columns of two column text since LaTeX treats
%    such title text as a type of float. A new command, \IEEEaftertitletext{}, 
%    gives access to the end of that produced by \maketitle. The types of
%    things that can go into \IEEEaftertitletext are the same as those into 
%    \twocolumn[] - no \par, but \\ are OK. There is no restriction on the
%    range of spacings that can be used. 
%    i.e., \IEEEaftertitletext{\vspace{-100pt}} will push the main text well
%    into the title and \IEEEaftertitletext{\vspace{100pt}} will push it far down
%    the page. You will have complete control. If used, place
%    \IEEEaftertitletext{} before \maketitle like \title and \author. IEEEtran's
%    dynamic title spacing intentionally does not take into consideration the 
%    contents of \IEEEaftertitletext{} when determining the spacer after the title
%    area (otherwise it would try to second guess you), so the user will have
%    manually adjust the height of the contents in \IEEEaftertitletext{} if the
%    problem discussed in #14 above should develop. A safe bet is to keep
%    the height of contents of \IEEEaftertitletext{} to integer multiples of 
%    \baselineskip, i.e., \IEEEaftertitletext{\vspace{-1\baselineskip}} 
%    
%    Because it can result in an IEEE nonstandard format, the use of
%    \IEEEaftertitletext{} is discouraged. Possible uses include (1) the use of
%    IEEEtran for non-IEEE work with different title spacing requirements,
%    or (2) as an emergency manual override if a problem should develop in 
%    IEEEtran's automatic spacing algorithm.
% 
%17) completely rewritten \PARstart to:
%    a. no longer have problems when the user begins an environment
%       within the paragraph that uses \PARstart.
%    b. auto-detect and use the current font family
%    c. revise handling of the space at the end of the first word so that
%       interword glue will now work as normal.
%    d. produce correctly aligned edges for the (two) indented lines.
%
%    Because the current font family is now auto-detected, there is no
%    longer any need for \CMPARstart - it is now the same as \PARstart.
%   
%18) There is now a new "open box" Q.E.D. symbol (\QEDopen) as well as the
%    original default (\QED) closed one (\QEDclosed). Some journals use
%    the open form. To make \proof use the open form, just do:
%    \renewcommand{\QED}{\QEDopen}
%
%19) Additional \typeout{} notices added to warn the user when unusual 
%    settings/commands are detected or as reminders to avoid common errors
%    when in conference mode.
%
%20) IEEEtran now provides \abovecaptionskip and \belowcaptionskip skip
%    registers because article class provides them and some packages
%    may error if they are missing. However, IEEEtran only uses 
%    \abovecaptionskip for actual caption spacing.
%
%21) Fixed bug that prevented users from redefining the section headings
%    to use arabic digits. Thanks to Richardt H. Wilkinson for reporting
%    this bug.
%
%22) Code cleaned up to be more efficient with the use of TeX registers;
%    removed some old LaTeX 2.09 code; revised option processing to 
%    LaTeX2e standard; eliminated unwanted "phantom" spaces in some
%    environments.
%
%23) Added new \IEEEeqnarray, \IEEEeqnarraybox, \IEEEeqnarrayboxm and 
%    \IEEEeqnarrayboxt environments to provide superior alternatives to the
%    standard LaTeX \eqnarray, \array and \tabular. Additional new support
%    commands include \IEEEeqnarraydecl, \IEEEeqnarrayboxdecl,
%    \IEEEeqnarraymathstyle, \IEEEeqnarraytextstyle, \yesnumber. \IEEEnonumber,
%    \IEEEyesnumber, \IEEEyessubnumber, \IEEEeqnarraynumspace, \IEEEeqnarraymulticol,
%    \IEEEeqnarrayomit, \IEEEeqnarraydefcol, \IEEEeqnarraydefcolsep, \IEEEeqnarrayseprow,
%    \IEEEeqnarrayseprowcut, \IEEEeqnarrayrulerow, \IEEEeqnarraydblrulerowcut,
%    \IEEEeqnarraystrutmode, \IEEEeqnarraystrutsize, \IEEEeqnarraystrutsizeadd, 
%    \IEEEvisiblestrutstrue, \IEEEvisiblestrutsfalse and \IEEEstrut.
%    These are documented in the user's guide.
%    
%24) V1.6 changed back to using () around theorem names (which are also now in italics)
%    as this is what IEEE is using now. Thanks to Christian Peel for reporting this.
%    Also, when section numbers are used as the first part of theorem numbers, display
%    them in arabic, not Roman.
%    
%25) New \IEEEtriggeratref{X} command allows a page break to be triggered just
%    before the given reference number "X". This is most useful when balancing
%    the columns on the last page and a \newpage between references is desired.
%    \IEEEtriggercmd{X} allows a different command to be executed at trigger.
%
%
%*******
% 7/2001 V1.5 (MDS) changes:
%
%
% 1) Fixed \and within \author bug: (! Misplaced \crcr. \endtabular ->\crcr)
%    Thanks to Rainer Dorsch for discovering and reporting that \and 
%    did not work.
%    
% 2) Fixed the biography environment so that if a biography's text is shorter
%    than the area allocated for the photo, a collision with the next
%    biography does not occur. You can now put real graphics (using the
%    graphicx package) into the biography photo box with a new optional 
%    argument of the biography command! For example:
%    
%    \begin{biography}[{\includegraphics[width=1in,height=1.25in,clip,keepaspectratio]{./tux.eps}}]{Linux Penguin}
%    
%    will use the specified graphic as the author's photo. The photo area is
%    exactly 1in wide by 1.25in high - as is done in IEEE Transactions. Try
%    to keep the same 4:5 aspect ratio if scanning/cropping your photos. 
%    Note the need for the extra set of enclosing braces around the
%    \includegraphics. Without it, The LaTeX parser may get confused when it
%    sees the \includegraphics's brackets within the biography's optional
%    argument. Due to the length of the \includegraphics command, you may wish
%    to define your own shorthand form of it. I have not done so with IEEEtran 
%    to prevent dependence on the graphicx package. If you do not use the 
%    optional argument, or leave it empty, a standard frame box with the 
%    words "Place Photo Here" will be used. If you want the space to remain
%    completely empty, you can do:
%    
%    \begin{biography}[\mbox{}]{The Invisible Man} 
%    
%    The interface to biography's optional argument is into a
%    1in X 1.25in minipage in which the argument text is centered both 
%    horizontally and vertically:
%    
%    \begin{minipage}[b][1.25in][c]{1in}%
%    \centering
%    #1%
%    \end{minipage}
%    
%    Within the biography environment, \unitlength is set to 1in.
%    With this in mind, you can even design your own custom frameboxes.
%    For instance:
%    
%    \begin{biography}[\framebox(1,1.25){\parbox[][\height][c]{0.9in}{\centering PLACE\\ PHOTO\\ HERE}}]{Author Name}
%    
%    will yield the same type of result as the default photo box.
%    
%    Thanks to Herbert Voss for discovering the collision bug, suggesting the ability
%    to handle graphics and providing some prototype code.
%
%
%
%*******
% 3/2001 V1.4 (MDS) changes:
%
%
% 1) New "draftcls" and "final" options have been added.
%    Thanks to Dragan Cvetkovic for suggesting an option like draftcls.
%    
% 2) Documentation changes to reflect the fact that this IEEEtran.cls 
%    is no longer beta test.
%
% 3) Slightly revised caption sizes. Figure and table captions are now 
%    in \footnotesize, not \small as before.
%
% 4) Allow user to control figure caption justification. IEEEtran.cls 
%    normally defaults to left justified as is done in Transactions.
%    However, for conferences, you may wish to issue the command:
%    \centerfigcaptionstrue
%    in the preamble. Short (less than one line long) figure captions
%    will then be centered. Multi-line figure captions will always be 
%    properly left justified. V1.6: This is already done for you when
%    using the conference mode.
%
%
%
%*******
% 1/2001 V1.3 
% Michael Shell (MDS) made extensive changes and additions:
%
%
% BUGS FIXED (and many others too numerous to mention!):
% 1) Fixed improper alignment with itemized, enumerated and
%    description lists. Added new controls to these three
%    environments so that it is easy to get the alignment IEEE
%    uses. Furthermore, the itemize, enumerate and description lists
%    no longer force a new paragraph to begin at the end the list
%    (\par). (Sometimes lists are used within paragraphs.) 
% 
% 2) JVH's fixes now allow things like $\mathbf{N}(0,P(0))$
%    to work properly without needing the extra braces:
%    ${\mathbf{N}}(0,P(0))$. There is no longer any dependence
%    on the "rawfonts" and "oldlfont" packages. Thanks Juergen! 
% 
% 3) Fixed underfull hbox errors and incorrect reference number
%    alignment when the number of references in the bibliography
%    exceeded 9 entries (which is almost every paper!).
%  
% 4) Removed dependence on the LaTeX sizexx.clo files.
%    Now, 9pt documents should work correctly even on systems that
%    lack a size9.clo file. This is most often used in conjunction
%    with the option "technote" for "correspondence" papers like those
%    in IEEE Transactions on Information Theory. For virtually all
%    other papers, 10pt is used and so it is the default.
%    Some improper font sizes have been corrected. \footnotesize is
%    now 8pt in 9pt docs, so footnotes in technotes should be the
%    correct size now. 
% 
% 5) Added \interlinepenalty within the bibliography section to discourage
%    LaTeX from breaking within a reference. IEEE almost never breaks within
%    a reference and when they do it is usually in technotes
%    (correspondence papers). You may get an underfull vbox warning in the
%    bibliography indicating that the spacing just before the "REFERENCES" 
%    section is larger than normal, but the final result will be more like 
%    what IEEE will publish. See the comments in the BIBLIOGRAPHY section
%    around line 2034 below if you want to change this behavior.
%
% 6) No longer "blows up" when you use \paragraph and have a table
%    of contents.
% 
% 7) Theorem environment changed, (but for V1.6, back to the old way, sigh).
% 
% 8) Figure captions adjusted: IEEE left (not center) justifies
%    figure captions (for journals) and does not indent figure caption text.
% 
% 9) Adjusted some spacings in the table of contents(TOC))/list-of-figures/
%    list-of-tables so that section/table numbers will not so easily 
%    collide with the titles. Section VIII was usually the worst offender.
%    Still doesn't right justify the section numbers, but neither does 
%    article.cls (This must be why LaTeX likes the x.y.z section numbering
%    scheme unlike I, II, III, etc. of IEEE. )
%    It may be "normal" as it is (left justified). sigh.
% 
%10) Now uses "index terms" now as a heading instead of "keywords".
%    Furthermore, the "index terms" and "abstract" headings are in bold
%    italic. This is how IEEE does things.
%
%11) \thebibliography and \biography now put entries into
%    the table of contents for you.
%
% *******
%
%
%
%
%
% *******
% 9/2000 (JVH) changes: (now designated as V1.2)
% 
% made some corrections to get closer to LaTeX2e
% 20000906 Juergen v.Hagen
% vonhagen@ihefiji.etec.uni-karlsruhe.de
% 
% Permission to redistribute granted as of December 2000.
% *******
%
%
%
%
%
% *******
% 
% 1996 (JWD) LaTeX2e version: (now designated as V1.1)
%  
% In the most recent TeXhax digest, there was a request for a copy of
% IEEEtrans.sty modified to work with LaTeX2e.  I have a version I
% modified to make it IEEEtrans.cls, which I have sent to the person
% making the request and am now sending to you to consider posting to
% the archives.
% --
% Jon Dixon
% dixonj@colorado.edu
% http://spot.colorado.edu/~dixonj/
%
%*******
%
%
%
%
%
%*******
%
% 30-August-1993 original LaTeX 2.09 version (IEEEtran.sty),
% (now designated as V1.0):
%
% by Gerry Murray and Silvano Balemi
% Automatic Control Lab, ETH Zurich, Switzerland
% balemi@aut.ee.ethz.ch
%
%*******
%
%%%%%%%%%%%%%%%%%%%%%%%%%%%%%%%%%%%%%%%%%%%%%%%%%%%%%%%%%%%%%%%%%%%%%%%%%%%%%%%
%
%
%
%
%
\ProvidesClass{IEEEtran}[2005/09/13 revision V1.6c by Michael Shell]
\typeout{-- See the "IEEEtran_HOWTO" manual for usage information.}
\typeout{-- The source comments contain changelog notes.}
\NeedsTeXFormat{LaTeX2e}

% define new needed flags to indicate document options
% and set a few "failsafe" defaults
\newif\if@twocolumnmode      \global\@twocolumnmodetrue
\newif\if@draftversion       \global\@draftversionfalse
\newif\if@draftclsmode       \global\@draftclsmodefalse
\newif\if@draftclsmodefoot   \global\@draftclsmodefootfalse
\newif\if@confmode           \global\@confmodefalse
\newif\if@peerreviewoption   \global\@peerreviewoptionfalse
\newif\if@peerreviewcaoption \global\@peerreviewcaoptionfalse

% we HAVE to turn off technote as there is no
% "not a tech note" option
\newif\if@technote      \global\@technotefalse

% V1.6 we allow the user to control whether or not the
% font interword spacings are tuned to be more like
% that of IEEE. The default is to tune things.
\newif\if@fonttunesettings   \global\@fonttunesettingstrue

% V1.6b flag to show if using a4paper
\newif\if@IEEEusingAfourpaper      \global\@IEEEusingAfourpaperfalse

% IEEEtran class scratch pad registers
% dimen
\newdimen\@IEEEtrantmpdimenA
\newdimen\@IEEEtrantmpdimenB
% count
\newcount\@IEEEtrantmpcountA
\newcount\@IEEEtrantmpcountB
% token list
\newtoks\@IEEEtrantmptoksA

% we use \@IEEEptsize so that we can ID the point size (even for 9pt docs)
% as well as LaTeX's \@ptsize to retain some compatability with some
% external packages
\def\@IEEEptsize{10}
\def\@ptsize{0}
% LaTeX does not support 9pt, so we set \@ptsize to 0 - same as that of 10pt
\DeclareOption{9pt}{\def\@IEEEptsize{9}\def\@ptsize{0}}
\DeclareOption{10pt}{\def\@IEEEptsize{10}\def\@ptsize{0}}
\DeclareOption{11pt}{\def\@IEEEptsize{11}\def\@ptsize{1}}
\DeclareOption{12pt}{\def\@IEEEptsize{12}\def\@ptsize{2}}


% \@IEEEmarginE is the side margin for equal margins
% \@IEEEmarginW is the wider side margin when the margins are not equal
% NOTE: BOTH of the above margins are as they appear
% on the paper - they are NOT offset by 1 inch
\DeclareOption{letterpaper}{\setlength{\paperheight}{11in}%
                            \setlength{\paperwidth}{8.5in}%
                            \def\@IEEEmarginE{0.680in}%
                            \def\@IEEEmarginW{0.775in}%
                            \@IEEEusingAfourpaperfalse}


\DeclareOption{a4paper}{\setlength{\paperheight}{297mm}%
                        \setlength{\paperwidth}{210mm}%
                        \def\@IEEEmarginE{14.32mm}%
                        \def\@IEEEmarginW{17mm}
                        \@IEEEusingAfourpapertrue}

\DeclareOption{oneside}{\@twosidefalse \@mparswitchfalse}
\DeclareOption{twoside}{\@twosidetrue \@mparswitchtrue}

\DeclareOption{onecolumn}{\global\@twocolumnmodefalse}
% the file twocolumn.sty is not read as it changes \textwidth.
\DeclareOption{twocolumn}{\global\@twocolumnmodetrue}

% If the user selects draft, then this class AND any packages
% will go into draft mode.
\DeclareOption{draft}{\global\@draftversiontrue \global\@draftclsmodetrue
\global\@draftclsmodefoottrue} 
% draftcls is for a draft mode which will not affect any packages
% used by the document.
\DeclareOption{draftcls}{\global\@draftversionfalse \global\@draftclsmodetrue
\global\@draftclsmodefoottrue} 
% draftclsnofoot is like draftcls, but without the footer.
\DeclareOption{draftclsnofoot}{\global\@draftversionfalse \global\@draftclsmodetrue
\global\@draftclsmodefootfalse} 
% we provide a final option just for completeness (article.cls has one)
\DeclareOption{final}{\global\@draftversionfalse \global\@draftclsmodefalse
\global\@draftclsmodefootfalse}

\DeclareOption{journal}{\global\@peerreviewoptionfalse \global\@peerreviewcaoptionfalse
\global\@confmodefalse \global\@technotefalse}

\DeclareOption{conference}{\global\@peerreviewoptionfalse \global\@peerreviewcaoptionfalse
\global\@confmodetrue \global\@technotefalse}

\DeclareOption{technote}{\global\@peerreviewoptionfalse \global\@peerreviewcaoptionfalse
\global\@confmodefalse \global\@technotetrue}

\DeclareOption{peerreview}{\global\@peerreviewoptiontrue \global\@peerreviewcaoptionfalse
\global\@confmodefalse \global\@technotefalse}

\DeclareOption{peerreviewca}{\global\@peerreviewoptiontrue \global\@peerreviewcaoptiontrue
\global\@confmodefalse \global\@technotefalse}

\DeclareOption{nofonttune}{\global\@fonttunesettingsfalse}


% IEEE uses Times font, so we'll default to times.
% These three commands make up the entire times.sty package.
\renewcommand{\sfdefault}{phv}
\renewcommand{\rmdefault}{ptm}
\renewcommand{\ttdefault}{pcr}
% enable Times now - so that all class options can see the correct font families
\normalfont\selectfont


% default to US letter paper, 10pt, twocolumn, one sided, final, journal
\ExecuteOptions{letterpaper,10pt,twocolumn,oneside,final,journal}
% overrride these defaults per user requests
\ProcessOptions

% we can send console reminder messages to the user here
\AtEndDocument{\if@confmode%
\typeout{}%
\typeout{** Conference Paper **}%
\typeout{Before submitting the final camera ready copy, remember to:}%
\typeout{}%
\typeout{ 1. Manually equalize the lengths of two columns on the last page}%
\typeout{ of your paper;}%
\typeout{}%
\typeout{ 2. Ensure that any PostScript and/or PDF output post-processing}%
\typeout{ uses only Type 1 fonts and that every step in the generation}%
\typeout{ process uses the US letter (8.5in X 11in) paper size.}%
\typeout{}%
\fi}


% warn about the use of single column other than for draft mode
\if@twocolumnmode\else%
  \if@draftclsmode\else%
   \typeout{** ATTENTION: Single column mode is not normally used with IEEE publications.}%
  \fi%
\fi


% V1.6, if the user is using pdflatex, go ahead and set the output paper size.
% Otherwise, we declare the papersize via a \special for dvips.
% We keep the tests within braces because otherwise, if not using pdflatex,
% \pdfpageheight and \pdfpagewidth will be set to \relax - possibly affecting
% similar tests of other packages.
{\@ifundefined{pdfpageheight}{% not using pdflatex, setup paper size for dvips
\if@IEEEusingAfourpaper
\special{papersize=210mm,297mm}%
\else
\special{papersize=8.5in,11in}%
\fi}%
{% using pdftex, set paper size for pdftex
\global\pdfpageheight\paperheight\global\pdfpagewidth\paperwidth}}



% The idea hinted here is for LaTeX to generate markleft{} and markright{}
% automatically for you after you enter \author{}, \journal{},
% \journaldate{}, journalvol{}, \journalnum{}, etc.
% However, there may be some backward compatibility issues here as
% well as some special applications for IEEEtran.cls and special issues
% that may require the flexible \markleft{}, \markright{} and/or \markboth{}.
% We'll leave this as an open future suggestion.
%\newcommand{\journal}[1]{\def\@journal{#1}}
%\def\@journal{}



% pointsize values
% used with ifx to determine the document's normal size
\def\@IEEEptsizenine{9}
\def\@IEEEptsizeten{10}
\def\@IEEEptsizeeleven{11}
\def\@IEEEptsizetwelve{12}



% FONT DEFINITIONS (No sizexx.clo file needed) 
% V1.6 revised font sizes, displayskip values and
%      revised normalsize baselineskip to reduce underfull vbox problems
%      on the 58pc = 696pt = 9.5in text height we want
%      normalsize     #lines/column  baselineskip (aka leading)
%             9pt     63             11.0476pt (truncated down)
%            10pt     58             12pt      (exact)
%            11pt     52             13.3846pt (truncated down)
%            12pt     50             13.92pt   (exact)
%

% we need to store the nominal baselineskip for the given font size
% in case baselinestretch ever changes.
\newskip\@IEEEnormalsizefontbaselineskip
\@IEEEnormalsizefontbaselineskip\baselineskip

\ifx\@IEEEptsize\@IEEEptsizenine
\typeout{-- This is a 9 point document.}
\def\normalsize{\@setfontsize{\normalsize}{9}{11.0476pt}}%
\setlength{\@IEEEnormalsizefontbaselineskip}{11.0476pt}%
\normalsize
\abovedisplayskip 1.5ex plus3pt minus1pt%
\belowdisplayskip \abovedisplayskip%
\abovedisplayshortskip 0pt plus3pt%
\belowdisplayshortskip 1.5ex plus3pt minus1pt
\def\small{\@setfontsize{\small}{8.5}{10pt}}
\def\footnotesize{\@setfontsize{\footnotesize}{8}{9pt}}
\def\scriptsize{\@setfontsize{\scriptsize}{7}{8pt}}
\def\tiny{\@setfontsize{\tiny}{5}{6pt}}
% sublargesize is the same as large - 10pt
\def\sublargesize{\@setfontsize{\sublargesize}{10}{12pt}}
\def\large{\@setfontsize{\large}{10}{12pt}}
\def\Large{\@setfontsize{\Large}{12}{14pt}}
\def\LARGE{\@setfontsize{\LARGE}{14}{17pt}}
\def\huge{\@setfontsize{\huge}{17}{20pt}}
\def\Huge{\@setfontsize{\Huge}{20}{24pt}}
\fi


% Check if we have selected 10 points
\ifx\@IEEEptsize\@IEEEptsizeten
\typeout{-- This is a 10 point document.}
\def\normalsize{\@setfontsize{\normalsize}{10}{12.00pt}}%
\setlength{\@IEEEnormalsizefontbaselineskip}{12pt}%
\normalsize
\abovedisplayskip 1.5ex plus4pt minus2pt%
\belowdisplayskip \abovedisplayskip%
\abovedisplayshortskip 0pt plus4pt%
\belowdisplayshortskip 1.5ex plus4pt minus2pt
\def\small{\@setfontsize{\small}{9}{10pt}}
\def\footnotesize{\@setfontsize{\footnotesize}{8}{9pt}}
\def\scriptsize{\@setfontsize{\scriptsize}{7}{8pt}}
\def\tiny{\@setfontsize{\tiny}{5}{6pt}}
% sublargesize is a tad smaller than large - 11pt
\def\sublargesize{\@setfontsize{\sublargesize}{11}{13.4pt}}
\def\large{\@setfontsize{\large}{12}{14pt}}
\def\Large{\@setfontsize{\Large}{14}{17pt}}
\def\LARGE{\@setfontsize{\LARGE}{17}{20pt}}
\def\huge{\@setfontsize{\huge}{20}{24pt}}
\def\Huge{\@setfontsize{\Huge}{24}{28pt}}
\fi


% Check if we have selected 11 points
\ifx\@IEEEptsize\@IEEEptsizeeleven
\typeout{-- This is an 11 point document.}
\def\normalsize{\@setfontsize{\normalsize}{11}{13.3846pt}}%
\setlength{\@IEEEnormalsizefontbaselineskip}{13.3846pt}%
\normalsize
\abovedisplayskip 1.5ex plus5pt minus3pt%
\belowdisplayskip \abovedisplayskip%
\abovedisplayshortskip 0pt plus5pt%
\belowdisplayshortskip 1.5ex plus5pt minus3pt
\def\small{\@setfontsize{\small}{10}{12pt}}
\def\footnotesize{\@setfontsize{\footnotesize}{9}{10.5pt}}
\def\scriptsize{\@setfontsize{\scriptsize}{8}{9pt}}
\def\tiny{\@setfontsize{\tiny}{6}{7pt}}
% sublargesize is the same as large - 12pt
\def\sublargesize{\@setfontsize{\sublargesize}{12}{14pt}}
\def\large{\@setfontsize{\large}{12}{14pt}}
\def\Large{\@setfontsize{\Large}{14}{17pt}}
\def\LARGE{\@setfontsize{\LARGE}{17}{20pt}}
\def\huge{\@setfontsize{\huge}{20}{24pt}}
\def\Huge{\@setfontsize{\Huge}{24}{28pt}}
\fi


% Check if we have selected 12 points
\ifx\@IEEEptsize\@IEEEptsizetwelve
\typeout{-- This is a 12 point document.}
\def\normalsize{\@setfontsize{\normalsize}{12}{13.92pt}}%
\setlength{\@IEEEnormalsizefontbaselineskip}{13.92pt}%
\normalsize
\abovedisplayskip 1.5ex plus6pt minus4pt%
\belowdisplayskip \abovedisplayskip%
\abovedisplayshortskip 0pt plus6pt%
\belowdisplayshortskip 1.5ex plus6pt minus4pt
\def\small{\@setfontsize{\small}{10}{12pt}}
\def\footnotesize{\@setfontsize{\footnotesize}{9}{10.5pt}}
\def\scriptsize{\@setfontsize{\scriptsize}{8}{9pt}}
\def\tiny{\@setfontsize{\tiny}{6}{7pt}}
% sublargesize is the same as large - 14pt
\def\sublargesize{\@setfontsize{\sublargesize}{14}{17pt}}
\def\large{\@setfontsize{\large}{14}{17pt}}
\def\Large{\@setfontsize{\Large}{17}{20pt}}
\def\LARGE{\@setfontsize{\LARGE}{20}{24pt}}
\def\huge{\@setfontsize{\huge}{22}{26pt}}
\def\Huge{\@setfontsize{\Huge}{24}{28pt}}
\fi


% V1.6 The Computer Modern Fonts will issue a substitution warning for
% 24pt titles (24.88pt is used instead) increase the substitution
% tolerance to turn off this warning
\def\fontsubfuzz{.9pt}
% However, the default (and correct) Times font will scale exactly as needed.


% warn the user in case they forget to use the 9pt option with
% technote
\if@technote%
 \ifx\@IEEEptsize\@IEEEptsizenine\else%
  \typeout{** ATTENTION: Technotes are normally 9pt documents.}%
 \fi%
\fi


% set \baselinestretch
\def\baselinestretch{1}
\if@draftclsmode% draft mode uses larger than normal spacing
\def\baselinestretch{1.5} % controls line spacing for draft version
\fi                       % some people may like 1.7 or greater
                          % so that there will be even more space
                          % for hand written comments 

\normalsize % make \baselinestretch take affect


% V1.6
% store the normalsize baselineskip
\newskip\normalsizebaselineskip
\normalsizebaselineskip=\baselineskip\relax
% store the nominal value of jot
\newskip\normaljot
\normaljot=0.25\normalsizebaselineskip\relax

% set \jot
\jot=\normaljot\relax


% abstract and keywords are in \small, except 
% for 9pt docs in which they are in \footnotesize
% Since 9pt docs use an 8pt footnotesize, \small
% becomes a rather awkward 8.5pt
\let\@IEEEabskeysecsize=\small
\ifx\@IEEEptsize\@IEEEptsizenine
 \let\@IEEEabskeysecsize=\footnotesize
\fi



% V1.6, we are now going to fine tune the interword spacing
% The default interword glue for Times under TeX appears to use a
% nominal interword spacing of 25% (relative to the font size, i.e., 1em)
% a maximum of 40% and a minimum of 19%.
% For example, 10pt text uses an interword glue of:
% 
% 2.5pt plus 1.49998pt minus 0.59998pt
% 
% However, IEEE allows for a more generous range which reduces the need
% for hyphenation, especially for two column text. Furthermore, IEEE
% tends to use a little bit more nominal space between the words.
% IEEE's interword spacing percentages appear to be:
% 35% nominal
% 23% minimum
% 50% maximum
% (They may even be using a tad more for the largest fonts such as 24pt.)
% 
% for bold text, IEEE increases the spacing a little more:
% 37.5% nominal
% 23% minimum
% 55% maximum

% here are the interword spacing ratios we'll use
% for medium (normal weight)
\def\@IEEEinterspaceratioM{0.35}
\def\@IEEEinterspaceMINratioM{0.23}
\def\@IEEEinterspaceMAXratioM{0.50}

% for bold
\def\@IEEEinterspaceratioB{0.375}
\def\@IEEEinterspaceMINratioB{0.23}
\def\@IEEEinterspaceMAXratioB{0.55}


% command to revise the interword spacing for the current font under TeX:
% \fontdimen2 = nominal interword space
% \fontdimen3 = interword stretch
% \fontdimen4 = interword shrink
% since all changes to the \fontdimen are global, we can enclose these commands
% in braces to confine any font attribute or length changes
\def\@@@IEEEsetfontdimens#1#2#3{{%
\setlength{\@IEEEtrantmpdimenB}{\f@size pt}% grab the font size in pt, could use 1em instead.
\setlength{\@IEEEtrantmpdimenA}{#1\@IEEEtrantmpdimenB}%
\fontdimen2\font=\@IEEEtrantmpdimenA\relax
\addtolength{\@IEEEtrantmpdimenA}{-#2\@IEEEtrantmpdimenB}%
\fontdimen3\font=-\@IEEEtrantmpdimenA\relax
\setlength{\@IEEEtrantmpdimenA}{#1\@IEEEtrantmpdimenB}%
\addtolength{\@IEEEtrantmpdimenA}{-#3\@IEEEtrantmpdimenB}%
\fontdimen4\font=\@IEEEtrantmpdimenA\relax}}

% revise the interword spacing for each font weight
\def\@@IEEEsetfontdimens{{%
\mdseries
\@@@IEEEsetfontdimens{\@IEEEinterspaceratioM}{\@IEEEinterspaceMAXratioM}{\@IEEEinterspaceMINratioM}%
\bfseries
\@@@IEEEsetfontdimens{\@IEEEinterspaceratioB}{\@IEEEinterspaceMAXratioB}{\@IEEEinterspaceMINratioB}%
}}

% revise the interword spacing for each font shape
% \slshape is not often used for IEEE work and is not altered here. The \scshape caps are
% already a tad too large in the free LaTeX fonts (as compared to what IEEE uses) so we
% won't alter these either.
\def\@IEEEsetfontdimens{{%
\normalfont
\@@IEEEsetfontdimens
\normalfont\itshape
\@@IEEEsetfontdimens
}}

% if the nofonttune class option is not given, revise the interword spacing
% for each font size (and shape and weight). Only the \rmfamily is done here
% as \ttfamily uses a fixed spacing and \sffamily is not used as the main
% text of IEEE papers.
\def\@IEEEtunefonts{%
\if@fonttunesettings
{\selectfont\rmfamily
\tiny\@IEEEsetfontdimens
\scriptsize\@IEEEsetfontdimens
\footnotesize\@IEEEsetfontdimens
\small\@IEEEsetfontdimens
\normalsize\@IEEEsetfontdimens
\sublargesize\@IEEEsetfontdimens
\large\@IEEEsetfontdimens
\LARGE\@IEEEsetfontdimens
\huge\@IEEEsetfontdimens
\Huge\@IEEEsetfontdimens}\fi}

% if needed, revise the interword spacing now - in case IEEEtran makes any default
% length measurements, and make sure all the default fonts are loaded
\@IEEEtunefonts

% and again at the start of the document in case the user loaded different fonts
\AtBeginDocument{\@IEEEtunefonts}



% V1.6 
% LaTeX is a little to quick to use hyphenations
% So, we increase the penalty for their use and raise
% the badness level that triggers an underfull hbox
% warning. The author may still have to tweak things,
% but the appearance will be much better "right out
% of the box" than that under V1.5 and prior.
% TeX default is 50
\hyphenpenalty=750
% If we didn't adjust the interword spacing, 2200 might be better.
% The TeX default is 1000
\hbadness=1350
% IEEE does not use extra spacing after punctuation
\frenchspacing



% we want to maintain textheight as an integer multiple of
% \baselineskip.  Keep \topsep in with this game plan too.
\topskip=\baselineskip
% set sizes and margins
% Book typesetting is a world where point and pica (12pt) reign supreme.
% IEEE textwidth is 21 pica. They have a colsep of 1 pica.
% V1.6 conference mode margins
\if@confmode
 \topmargin        -0.25in
 \textheight        9.25in % The standard for conferences
  % However, we will adjust this a  tad so that an integer number
  % of lines will always fit on each page
  % The baselineskip (leading) for each document point size is used 
  % to determine these values
  % rounded up an extra 0.1pt or so to prevent trouble with any rounding errors
 \ifx\@IEEEptsize\@IEEEptsizenine\textheight=674.0pt\fi      %9.3261in  61 lines/page
 \ifx\@IEEEptsize\@IEEEptsizeten\textheight=672.1pt\fi       %9.2998in  56 lines/page
 \ifx\@IEEEptsize\@IEEEptsizeeleven\textheight=669.3pt\fi   %9.2611in  50 lines/page
 \ifx\@IEEEptsize\@IEEEptsizetwelve\textheight=668.3pt\fi   %9.2473in  48 lines/page
\else
 \topmargin        -49.0pt
 \textheight        58pc % = 9.63in or 696pt
\fi


\textwidth         43pc   % 2 x 21pc + 1pc = 43pc
\columnsep          1pc


% IEEE MARGIN INFO and new \overrideIEEEmargins command
% V1.6 revised margins again
% IEEE wants the side margins to be equal under both US letter
% and A4 paper
% 
% However, for those of you who need to bind copies of your work
% (for review distribution, etc.) the \overrideIEEEmargins
% command will shift the text a tad away from the binding
% edge.
% 
%


% the default side margins are equal
\oddsidemargin    \@IEEEmarginE
\addtolength{\oddsidemargin}{-1in}% compensate for LaTeX's 1in offset
\evensidemargin   \@IEEEmarginE
\addtolength{\evensidemargin}{-1in}% compensate for LaTeX's 1in offset

% execute \overrideIEEEmargins in the preamble to make the side margin
% near the spine slightly wider so that the paper will be much more 
% agreeable to being bound.  \overrideIEEEmargins will have no effect
% when in draft or draftcls mode.
\def\overrideIEEEmargins{\if@draftclsmode\relax\else%
 \typeout{** ATTENTION: Overriding IEEE standard margins (line \the\inputlineno).}%
 \if@twoside
  % for double sided, odd pages have the bound side on the left
  % make this the wide margin
  \oddsidemargin\@IEEEmarginW
  % and even pages have the narrow margin on the left
  % as they are bound on the right
  % evensidemargin is to be the narrow margin
  % calculate the narrow margin and set evensidemargin
  \setlength{\evensidemargin}{\paperwidth}%
  \addtolength{\evensidemargin}{-\@IEEEmarginW}%
  \addtolength{\evensidemargin}{-\textwidth}%
 \else
  % for single sided the bound side is always on the left
  % make this the wide margin
  \oddsidemargin\@IEEEmarginW
  \evensidemargin\@IEEEmarginW
 \fi
 \addtolength{\oddsidemargin}{-1.0in}%  compensate for LaTeX's 1in offset
 \addtolength{\evensidemargin}{-1.0in}%
\fi}


\parindent        1.0em

% conference papers do not have headers, other papers need
% to reserve space for them
\if@confmode
\headsep          0in
\headheight       0in
\else
\headsep          0.25in
\headheight       12pt
\fi

% V1.6, if things get too close, go ahead and let them touch
\lineskip            0pt
\normallineskip      0pt
\lineskiplimit       0pt
\normallineskiplimit 0pt


% The distance from the lower edge of the text body to the
% footline
\footskip 0.4in

% normally zero, should be relative to font height.
% put in a little rubber to help stop some bad breaks (underfull vboxes)
\parskip 0ex plus 0.2ex minus 0.1ex



% draft mode settings override that of all other modes
% provides a nice 1" margin all around the paper and extra
% space between the lines for editor's comments
\if@draftclsmode 
\headsep          0.25in
\headheight       12pt
% want 1" from top of paper to text
\setlength{\topmargin}{-\headsep}%
\addtolength{\topmargin}{-\headheight}%

% we want 1in side margins regardless of paper type
\oddsidemargin      0in
\evensidemargin     0in

% set the text width - start with the entire page
\setlength{\textwidth}{\paperwidth}%
% subtract for the 1" top/bottom margins
\addtolength{\textwidth}{-2.0in}%
% give them a textheight that won't have underfull
% vbox problems, but can't help them if they later change 
% baselinestretch from its default
\setlength{\textheight}{\paperheight}%
\addtolength{\textheight}{-2.0in}%
% subtract of first line taken by \topskip
\addtolength{\textheight}{-1\topskip}%
% now digitize \textheight so that the length after
% the first line is an integer multiple of  \baselineskip
% to cut down on underfull vbox errors in draft mode
\divide\textheight  by \baselineskip%
\multiply\textheight  by \baselineskip%
% add back the first line
\addtolength{\textheight}{\topskip}%
\fi



% margin note stuff
\marginparsep      10pt
\marginparwidth    20pt
\marginparpush     25pt


% LIST SPACING CONTROLS

% Controls the amount of EXTRA spacing
% above and below \trivlist 
% Both \list and IED lists override this.
% However, \trivlist will use this as will most
% things built from \trivlist like the \center
% environment.
\topsep           0.5\baselineskip

% Controls the additional spacing around lists preceded
% or followed by blank lines. IEEE does not increase
% spacing before or after paragraphs so it is set to zero.
% \z@ is the same as zero, but faster.
\partopsep          \z@

% Controls the spacing between paragraphs in lists. 
% IEEE does not increase spacing before or after paragraphs
% so this is also zero. 
% With IEEEtran.cls, global changes to
% this value DO affect lists (but not IED lists).
\parsep             \z@

% Controls the extra spacing between list items. 
% IEEE does not put extra spacing between items.
% With IEEEtran.cls, global changes to this value DO affect
% lists (but not IED lists).
\itemsep            \z@

% \itemindent is the amount to indent the FIRST line of a list
% item. It is auto set to zero within the \list environment. To alter
% it, you have to do so when you call the \list.
% However, IEEE uses this for the theorem environment
% There is an alternative value for this near \leftmargini below
\itemindent         -1em

% \leftmargin, the spacing from the left margin of the main text to
% the left of the main body of a list item is set by \list.
% Hence this statement does nothing for lists.
% But, quote and verse do use it for indention.
\leftmargin         2em

% we retain this stuff from the older IEEEtran.cls so that \list
% will work the same way as before. However, itemize, enumerate and
% description (IED) could care less about what these are as they
% all are overridden.
\leftmargini        2em
%\itemindent         2em  % Alternative values: sometimes used.
%\leftmargini        0em
\leftmarginii       1em
\leftmarginiii    1.5em
\leftmarginiv     1.5em
\leftmarginv      1.0em
\leftmarginvi     1.0em
\labelsep         0.5em 
\labelwidth         \z@


% The old IEEEtran.cls behavior of \list is retained.
% However, the new V1.3 IED list environments override all the
% @list stuff (\@listX is called within \list for the
% appropriate level just before the user's list_decl is called). 
% \topsep is now 2pt as IEEE puts a little extra space around
% lists - used by those non-IED macros that depend on \list.
% Note that \parsep and \itemsep are not redefined as in 
% the sizexx.clo \@listX (which article.cls uses) so global changes
% of these values DO affect \list
% 
\def\@listi{\leftmargin\leftmargini \topsep 2pt plus 1pt minus 1pt}
\let\@listI\@listi
\def\@listii{\leftmargin\leftmarginii\labelwidth\leftmarginii%
    \advance\labelwidth-\labelsep \topsep 2pt}
\def\@listiii{\leftmargin\leftmarginiii\labelwidth\leftmarginiii%
    \advance\labelwidth-\labelsep \topsep 2pt}
\def\@listiv{\leftmargin\leftmarginiv\labelwidth\leftmarginiv%
    \advance\labelwidth-\labelsep \topsep 2pt}
\def\@listv{\leftmargin\leftmarginv\labelwidth\leftmarginv%
    \advance\labelwidth-\labelsep \topsep 2pt}
\def\@listvi{\leftmargin\leftmarginvi\labelwidth\leftmarginvi%
    \advance\labelwidth-\labelsep \topsep 2pt}


% IEEE uses 5) not 5.
\def\labelenumi{\theenumi)}     \def\theenumi{\arabic{enumi}}

% IEEE uses a) not (a)
\def\labelenumii{\theenumii)}  \def\theenumii{\alph{enumii}}

% IEEE uses iii) not iii.
\def\labelenumiii{\theenumiii)} \def\theenumiii{\roman{enumiii}}

% IEEE uses A) not A.
\def\labelenumiv{\theenumiv)}   \def\theenumiv{\Alph{enumiv}}

% exactly the same as in article.cls
\def\p@enumii{\theenumi}
\def\p@enumiii{\theenumi(\theenumii)}
\def\p@enumiv{\p@enumiii\theenumiii}

% itemized list label styles
\def\labelitemi{$\scriptstyle\bullet$}
\def\labelitemii{\textbf{--}}
\def\labelitemiii{$\ast$}
\def\labelitemiv{$\cdot$}


% IEEEtran.cls version numbers, provided as of V1.3
% These values serve as a way a .tex file can
% determine if the new features are provided.
% The version number of this IEEEtrans.cls can be obtained from 
% these values. i.e., V1.4
% KEEP THESE AS INTEGERS! i.e., NO {4a} or anything like that-
% (no need to enumerate "a" minor changes here)
\def\IEEEtransversionmajor{1}
\def\IEEEtransversionminor{6}


% **** V1.3 ENHANCEMENTS ****
% Itemize, Enumerate and Description (IED) List Controls
% ***************************
% 
% 
% IEEE seems to use at least two different values by
% which ITEMIZED list labels are indented to the right
% For The Journal of Lightwave Technology (JLT) and The Journal
% on Selected Areas in Communications (JSAC), they tend to use
% an indention equal to \parindent. For Transactions on Communications
% they tend to indent ITEMIZED lists a little more--- 1.3\parindent.
% We'll provide both values here for you so that you can choose 
% which one you like in your document using a command such as:
% setlength{\IEEEilabelindent}{\IEEEilabelindentB}
\newdimen\IEEEilabelindentA
\IEEEilabelindentA \parindent

\newdimen\IEEEilabelindentB
\IEEEilabelindentB 1.3\parindent
% However, we'll default to using \parindent
% which makes more sense to me
\newdimen\IEEEilabelindent
\IEEEilabelindent \IEEEilabelindentA


% This controls the default amount the enumerated list labels
% are indented to the right.
% Normally, this is the same as the paragraph indention
\newdimen\IEEEelabelindent
\IEEEelabelindent \parindent

% This controls the default amount the description list labels
% are indented to the right.
% Normally, this is the same as the paragraph indention
\newdimen\IEEEdlabelindent
\IEEEdlabelindent \parindent

% This is the value actually used within the IED lists.
% The IED environments automatically set its value to
% one of the three values above, so global changes do 
% not have any effect
\newdimen\labelindent
\labelindent \parindent

% The actual amount labels will be indented is
% \labelindent multiplied by the factor below
% corresponding to the level of nesting depth
% This provides a means by which the user can
% alter the effective \labelindent for deeper
% levels
% There may not be such a thing as correct "standard IEEE"
% values. What IEEE actually does may depend on the specific
% circumstances.
% The first list level almost always has full indention.
% The second levels I've seen have only 75% of the normal indentation
% Three level or greater nestings are very rare. I am guessing
% that they don't use any indentation.
\def\IEEElabelindentfactori{1.0}   % almost always one
\def\IEEElabelindentfactorii{0.75} % 0.0 or 1.0 may be used in some cases
\def\IEEElabelindentfactoriii{0.0} % 0.75? 0.5? 0.0?
\def\IEEElabelindentfactoriv{0.0}
\def\IEEElabelindentfactorv{0.0}
\def\IEEElabelindentfactorvi{0.0}

% value actually used within IED lists, it is auto
% set to one of the 6 values above
% global changes here have no effect
\def\labelindentfactor{1.0}

% This controls the default spacing between the end of the IED
% list labels and the list text, when normal text is used for
% the labels.
\newdimen\IEEEiednormlabelsep
\IEEEiednormlabelsep 0.6em

% This controls the default spacing between the end of the IED
% list labels and the list text, when math symbols are used for
% the labels (nomenclature lists). IEEE usually increases the 
% spacing in these cases
\newdimen\IEEEiedmathlabelsep
\IEEEiedmathlabelsep 1.2em

% This controls the extra vertical separation put above and
% below each IED list. IEEE usually puts a little extra spacing
% around each list. However, this spacing is barely noticeable.
\newskip\IEEEiedtopsep
\IEEEiedtopsep 2pt plus 1pt minus 1pt


% This command is executed within each IED list environment
% at the beginning of the list. You can use this to set the 
% parameters for some/all your IED list(s) without disturbing 
% global parameters that affect things other than lists.
% i.e., renewcommand{\iedlistdecl}{\setlength{\labelsep}{5em}}
% will alter the \labelsep for the next list(s) until 
% \iedlistdecl is redefined. 
\def\iedlistdecl{\relax}

% This command provides an easy way to set \leftmargin based
% on the \labelwidth, \labelsep and the argument \labelindent
% Usage: \calcleftmargin{width-to-indent-the-label}
% output is in the \leftmargin variable, i.e., effectively:
% \leftmargin = argument + \labelwidth + \labelsep
% Note controlled spacing here, shield end of lines with %
\def\calcleftmargin#1{\setlength{\leftmargin}{#1}%
\addtolength{\leftmargin}{\labelwidth}%
\addtolength{\leftmargin}{\labelsep}}

% This command provides an easy way to set \labelwidth to the
% width of the given text. It is the same as
% \settowidth{\labelwidth}{label-text}
% and useful as a shorter alternative.
% Typically used to set \labelwidth to be the width
% of the longest label in the list
\def\setlabelwidth#1{\settowidth{\labelwidth}{#1}}

% When this command is executed, IED lists will use the 
% IEEEiedmathlabelsep label separation rather than the normal
% spacing. To have an effect, this command must be executed via
% the \iedlistdecl or within the option of the IED list
% environments.
\def\usemathlabelsep{\setlength{\labelsep}{\IEEEiedmathlabelsep}}

% A flag which controls whether the IED lists automatically
% calculate \leftmargin from \labelindent, \labelwidth and \labelsep
% Useful if you want to specify your own \leftmargin
% This flag must be set (\nocalcleftmargintrue or \nocalcleftmarginfalse) 
% via the \iedlistdecl or within the option of the IED list
% environments to have an effect.
\newif\ifnocalcleftmargin
\nocalcleftmarginfalse

% A flag which controls whether \labelindent is multiplied by
% the \labelindentfactor for each list level.
% This flag must be set via the \iedlistdecl or within the option 
% of the IED list environments to have an effect.
\newif\ifnolabelindentfactor
\nolabelindentfactorfalse


% internal variable to indicate type of IED label
% justification
% 0 - left; 1 - center; 2 - right
\def\@iedjustify{0}


% commands to allow the user to control IED
% label justifications. Use these commands within
% the IED environment option or in the \iedlistdecl
% Note that changing the normal list justifications
% is nonstandard and IEEE may not like it if you do so!
% I include these commands as they may be helpful to
% those who are using these enhanced list controls for
% other non-IEEE related LaTeX work.
% itemize and enumerate automatically default to right
% justification, description defaults to left.
\def\iedlabeljustifyl{\def\@iedjustify{0}}%left
\def\iedlabeljustifyc{\def\@iedjustify{1}}%center
\def\iedlabeljustifyr{\def\@iedjustify{2}}%right




% commands to save to and restore from the list parameter copies
% this allows us to set all the list parameters within
% the list_decl and prevent \list (and its \@list) 
% from overriding any of our parameters
% V1.6 use \edefs instead of dimen's to conserve dimen registers
% Note controlled spacing here, shield end of lines with %
\def\@IEEEsavelistparams{\edef\@IEEEiedtopsep{\the\topsep}%
\edef\@IEEEiedlabelwidth{\the\labelwidth}%
\edef\@IEEEiedlabelsep{\the\labelsep}%
\edef\@IEEEiedleftmargin{\the\leftmargin}%
\edef\@IEEEiedpartopsep{\the\partopsep}%
\edef\@IEEEiedparsep{\the\parsep}%
\edef\@IEEEieditemsep{\the\itemsep}%
\edef\@IEEEiedrightmargin{\the\rightmargin}%
\edef\@IEEEiedlistparindent{\the\listparindent}%
\edef\@IEEEieditemindent{\the\itemindent}}

% Note controlled spacing here
\def\@IEEErestorelistparams{\topsep\@IEEEiedtopsep\relax%
\labelwidth\@IEEEiedlabelwidth\relax%
\labelsep\@IEEEiedlabelsep\relax%
\leftmargin\@IEEEiedleftmargin\relax%
\partopsep\@IEEEiedpartopsep\relax%
\parsep\@IEEEiedparsep\relax%
\itemsep\@IEEEieditemsep\relax%
\rightmargin\@IEEEiedrightmargin\relax%
\listparindent\@IEEEiedlistparindent\relax%
\itemindent\@IEEEieditemindent\relax}


% v1.6b provide original LaTeX IED list environments
% note that latex.ltx defines \itemize and \enumerate, but not \description
% which must be created by the base classes
% save original LaTeX itemize and enumerate
\let\LaTeXitemize\itemize
\let\endLaTeXitemize\enditemize
\let\LaTeXenumerate\enumerate
\let\endLaTeXenumerate\endenumerate

% provide original LaTeX description environment from article.cls
\newenvironment{LaTeXdescription}
               {\list{}{\labelwidth\z@ \itemindent-\leftmargin
                        \let\makelabel\descriptionlabel}}
               {\endlist}
\newcommand*\descriptionlabel[1]{\hspace\labelsep
                                 \normalfont\bfseries #1}


% override LaTeX's default IED lists
\def\itemize{\@IEEEitemize}
\def\enditemize{\@endIEEEitemize}
\def\enumerate{\@IEEEenumerate}
\def\endenumerate{\@endIEEEenumerate}
\def\description{\@IEEEdescription}
\def\enddescription{\@endIEEEdescription}

% provide the user with aliases - may help those using packages that
% override itemize, enumerate, or description
\def\IEEEitemize{\@IEEEitemize}
\def\endIEEEitemize{\@endIEEEitemize}
\def\IEEEenumerate{\@IEEEenumerate}
\def\endIEEEenumerate{\@endIEEEenumerate}
\def\IEEEdescription{\@IEEEdescription}
\def\endIEEEdescription{\@endIEEEdescription}


% V1.6 we want to keep the IEEEtran IED list definitions as our own internal
% commands so they are protected against redefinition
\def\@IEEEitemize{\@ifnextchar[{\@@IEEEitemize}{\@@IEEEitemize[\relax]}}
\def\@IEEEenumerate{\@ifnextchar[{\@@IEEEenumerate}{\@@IEEEenumerate[\relax]}}
\def\@IEEEdescription{\@ifnextchar[{\@@IEEEdescription}{\@@IEEEdescription[\relax]}}
\def\@endIEEEitemize{\endlist}
\def\@endIEEEenumerate{\endlist}
\def\@endIEEEdescription{\endlist}


% DO NOT ALLOW BLANK LINES TO BE IN THESE IED ENVIRONMENTS
% AS THIS WILL FORCE NEW PARAGRAPHS AFTER THE IED LISTS
% IEEEtran itemized list MDS 1/2001
% Note controlled spacing here, shield end of lines with %
\def\@@IEEEitemize[#1]{%
                \ifnum\@itemdepth>3\relax\@toodeep\else%
                \ifnum\@listdepth>5\relax\@toodeep\else%
                \advance\@itemdepth\@ne%
                \edef\@itemitem{labelitem\romannumeral\the\@itemdepth}%
                % get the labelindentfactor for this level
                \advance\@listdepth\@ne% we need to know what the level WILL be
                \edef\labelindentfactor{\csname IEEElabelindentfactor\romannumeral\the\@listdepth\endcsname}%
                \advance\@listdepth-\@ne% undo our increment
                \def\@iedjustify{2}% right justified labels are default
                % set other defaults
                \nocalcleftmarginfalse%
                \nolabelindentfactorfalse%
                \topsep\IEEEiedtopsep%
                \labelindent\IEEEilabelindent%
                \labelsep\IEEEiednormlabelsep%
                \partopsep 0ex%
                \parsep 0ex%
                \itemsep 0ex%
                \rightmargin 0em%
                \listparindent 0em%
                \itemindent 0em%
                % calculate the label width
                % the user can override this later if
                % they specified a \labelwidth
                \settowidth{\labelwidth}{\csname labelitem\romannumeral\the\@itemdepth\endcsname}%
                \@IEEEsavelistparams% save our list parameters
                \list{\csname\@itemitem\endcsname}{%
                \@IEEErestorelistparams% override any list{} changes
                                       % to our globals
                \let\makelabel\@IEEEiedmakelabel% v1.6b setup \makelabel
                \iedlistdecl% let user alter parameters
                #1\relax%
                % If the user has requested not to use the
                % labelindent factor, don't revise \labelindent
                \ifnolabelindentfactor\relax%
                \else\labelindent=\labelindentfactor\labelindent%
                \fi%
                % Unless the user has requested otherwise,
                % calculate our left margin based
                % on \labelindent, \labelwidth and
                % \labelsep
                \ifnocalcleftmargin\relax%
                \else\calcleftmargin{\labelindent}%
                \fi}\fi\fi}%


% DO NOT ALLOW BLANK LINES TO BE IN THESE IED ENVIRONMENTS
% AS THIS WILL FORCE NEW PARAGRAPHS AFTER THE IED LISTS
% IEEEtran enumerate list MDS 1/2001
% Note controlled spacing here, shield end of lines with %
\def\@@IEEEenumerate[#1]{%
                \ifnum\@enumdepth>3\relax\@toodeep\else%
                \ifnum\@listdepth>5\relax\@toodeep\else%
                \advance\@enumdepth\@ne%
                \edef\@enumctr{enum\romannumeral\the\@enumdepth}%
                % get the labelindentfactor for this level
                \advance\@listdepth\@ne% we need to know what the level WILL be
                \edef\labelindentfactor{\csname IEEElabelindentfactor\romannumeral\the\@listdepth\endcsname}%
                \advance\@listdepth-\@ne% undo our increment
                \def\@iedjustify{2}% right justified labels are default
                % set other defaults
                \nocalcleftmarginfalse%
                \nolabelindentfactorfalse%
                \topsep\IEEEiedtopsep%
                \labelindent\IEEEelabelindent%
                \labelsep\IEEEiednormlabelsep%
                \partopsep 0ex%
                \parsep 0ex%
                \itemsep 0ex%
                \rightmargin 0em%
                \listparindent 0em%
                \itemindent 0em%
                % calculate the label width
                % We'll set it to the width suitable for all labels using
                % normalfont 1) to 9)
                % The user can override this later
                \settowidth{\labelwidth}{9)}%
                \@IEEEsavelistparams% save our list parameters
                \list{\csname label\@enumctr\endcsname}{\usecounter{\@enumctr}%
                \@IEEErestorelistparams% override any list{} changes
                                       % to our globals
                \let\makelabel\@IEEEiedmakelabel% v1.6b setup \makelabel
                \iedlistdecl% let user alter parameters 
                #1\relax%
                % If the user has requested not to use the
                % labelindent factor, don't revise \labelindent
                \ifnolabelindentfactor\relax%
                \else\labelindent=\labelindentfactor\labelindent%
                \fi%
                % Unless the user has requested otherwise,
                % calculate our left margin based
                % on \labelindent, \labelwidth and
                % \labelsep
                \ifnocalcleftmargin\relax%
                \else\calcleftmargin{\labelindent}%
                \fi}\fi\fi}%


% DO NOT ALLOW BLANK LINES TO BE IN THESE IED ENVIRONMENTS
% AS THIS WILL FORCE NEW PARAGRAPHS AFTER THE IED LISTS
% IEEEtran description list MDS 1/2001
% Note controlled spacing here, shield end of lines with %
\def\@@IEEEdescription[#1]{%
                \ifnum\@listdepth>5\relax\@toodeep\else%
                % get the labelindentfactor for this level
                \advance\@listdepth\@ne% we need to know what the level WILL be
                \edef\labelindentfactor{\csname IEEElabelindentfactor\romannumeral\the\@listdepth\endcsname}%
                \advance\@listdepth-\@ne% undo our increment
                \def\@iedjustify{0}% left justified labels are default
                % set other defaults
                \nocalcleftmarginfalse%
                \nolabelindentfactorfalse%
                \topsep\IEEEiedtopsep% 
                \labelindent\IEEEdlabelindent%
                % assume normal labelsep
                \labelsep\IEEEiednormlabelsep%
                \partopsep 0ex%
                \parsep 0ex%
                \itemsep 0ex%
                \rightmargin 0em%
                \listparindent 0em%
                \itemindent 0em%
                % Bogus label width in case the user forgets
                % to set it.
                % TIP: If you want to see what a variable's width is you
                % can use the TeX command \showthe\width-variable to 
                % display it on the screen during compilation 
                % (This might be helpful to know when you need to find out
                % which label is the widest)
                \settowidth{\labelwidth}{Hello}%
                \@IEEEsavelistparams% save our list parameters
                \list{}{\@IEEErestorelistparams% override any list{} changes
                                               % to our globals
                \let\makelabel\@IEEEiedmakelabel% v1.6b setup \makelabel
                \iedlistdecl% let user alter parameters 
                #1\relax%
                % If the user has requested not to use the
                % labelindent factor, don't revise \labelindent
                \ifnolabelindentfactor\relax%
                \else\labelindent=\labelindentfactor\labelindent%
                \fi%
                % Unless the user has requested otherwise,
                % calculate our left margin based
                % on \labelindent, \labelwidth and
                % \labelsep
                \ifnocalcleftmargin\relax%
                \else\calcleftmargin{\labelindent}\relax%
                \fi}\fi}

% v1.6b we use one makelabel that does justification as needed.
\def\@IEEEiedmakelabel#1{\relax\if\@iedjustify 0\relax
\makebox[\labelwidth][l]{\normalfont #1}\else
\if\@iedjustify 1\relax
\makebox[\labelwidth][c]{\normalfont #1}\else
\makebox[\labelwidth][r]{\normalfont #1}\fi\fi}


% VERSE and QUOTE
\def\verse{\let\\=\@centercr
    \list{}{\itemsep\z@ \itemindent -1.5em \listparindent \itemindent
    \rightmargin\leftmargin\advance\leftmargin 1.5em}\item[]}
\let\endverse\endlist
\def\quotation{\list{}{\listparindent 1.5em \itemindent\listparindent
    \rightmargin\leftmargin \parsep 0pt plus 1pt}\item[]}
\let\endquotation=\endlist
\def\quote{\list{}{\rightmargin\leftmargin}\item[]}
\let\endquote=\endlist


% \titlepage
% provided only for backward compatibility. \maketitle is the correct
% way to create the title page. 
\newif\if@restonecol
\def\titlepage{\@restonecolfalse\if@twocolumn\@restonecoltrue\onecolumn
    \else \newpage \fi \thispagestyle{empty}\c@page\z@}
\def\endtitlepage{\if@restonecol\twocolumn \else \newpage \fi}

% standard values from article.cls
\arraycolsep     5pt
\arrayrulewidth .4pt
\doublerulesep   2pt

\tabcolsep       6pt
\tabbingsep      0.5em


%% FOOTNOTES
%
%\skip\footins 10pt plus 4pt minus 2pt
% V1.6 respond to changes in font size
% space added above the footnotes (if present)
\skip\footins 0.9\baselineskip  plus 0.4\baselineskip  minus 0.2\baselineskip

% V1.6, we need to make \footnotesep responsive to changes
% in \baselineskip or strange spacings will result when in
% draft mode. Here is a little LaTeX secret - \footnotesep
% determines the height of an invisible strut that is placed
% *above* the baseline of footnotes after the first. Since
% LaTeX considers the space for characters to be 0.7/baselineskip
% above the baseline and 0.3/baselineskip below it, we need to
% use 0.7/baselineskip as a \footnotesep to maintain equal spacing
% between all the lines of the footnotes. IEEE often uses a tad
% more, so use 0.8\baselineskip. This slightly larger value also helps
% the text to clear the footnote marks. Note that \thanks now uses
% its own value of \footnotesep.
{\footnotesize
\global\footnotesep 0.8\baselineskip}


\skip\@mpfootins = \skip\footins
\fboxsep = 3pt
\fboxrule = .4pt
% V1.6 use 1em, the use LaTeX2e's \@makefnmark
% Note that IEEE normally *left* aligns the footnote marks, so we don't need
% box resizing tricks here.
\long\def\@makefntext#1{\parindent 1em\indent\hbox{\@makefnmark}#1}% V1.6 use 1em

\def\footnoterule{}

% V1.6 do not allow LaTeX to break a footnote across multiple pages
\interfootnotelinepenalty=10000

% V1.6 discourage breaks within equations
% Note that amsmath normally sets this to 10000,
% but LaTeX2e normally uses 100.
\interdisplaylinepenalty=2500


\if@technote
   \setcounter{secnumdepth}{3}
\else
   \setcounter{secnumdepth}{4}
\fi


\newcounter{section}
\newcounter{subsection}[section]
\newcounter{subsubsection}[subsection]
\newcounter{paragraph}[subsubsection]

% used only by IEEEtran's IEEEeqnarray as other packages may
% have their own, different, implementations
\newcounter{IEEEsubequation}[equation]

% as shown when called by user from \ref, \label and in table of contents
\def\thesection{\Roman{section}}                             % I
\def\thesubsection{\thesection-\Alph{subsection}}            % I-A
\def\thesubsubsection{\thesubsection.\arabic{subsubsection}} % I-A.1
\def\theparagraph{\thesubsubsection.\alph{paragraph}}        % I-A.1.a
\def\theequation{\arabic{equation}}                          % 1
\def\theIEEEsubequation{\theequation\alph{IEEEsubequation}}  % 1a (used only by IEEEtran's IEEEeqnarray)

% Main text forms (how shown in main text headings)
% V1.6, using \thesection in \thesectiondis allows changes
% in the former to automatically appear in the latter
\def\thesectiondis{\thesection.}                   % I.
\def\thesubsectiondis{\Alph{subsection}.}          % B.
\def\thesubsubsectiondis{\arabic{subsubsection})}  % 3)
\def\theparagraphdis{\alph{paragraph})}            % d)
% just like LaTeX2e's \@eqnnum
\def\theequationdis{{\normalfont \normalcolor (\theequation)}}% (1)
% IEEEsubequation used only by IEEEtran's IEEEeqnarray
\def\theIEEEsubequationdis{{\normalfont \normalcolor (\theIEEEsubequation)}}% (1a)
% redirect LaTeX2e's equation number display and all that depend on
% it, through IEEEtran's \theequationdis
\def\@eqnnum{\theequationdis}

% LIST OF FIGURES AND TABLES AND TABLE OF CONTENTS
%
\def\@pnumwidth{1.55em}
\def\@tocrmarg{2.55em}
\def\@dotsep{4.5}
\setcounter{tocdepth}{3}

% adjusted some spacings here so that section numbers will not easily 
% collide with the section titles. 
% VIII; VIII-A; and VIII-A.1 are usually the worst offenders.
% MDS 1/2001
\def\tableofcontents{\section*{Contents}\@starttoc{toc}}
\def\l@section#1#2{\addpenalty{\@secpenalty}\addvspace{1.0em plus 1pt}%
    \@tempdima 2.75em \begingroup \parindent \z@ \rightskip \@pnumwidth%
    \parfillskip-\@pnumwidth {\bfseries\leavevmode #1}\hfil\hbox to\@pnumwidth{\hss #2}\par%
    \endgroup}
% argument format #1:level, #2:labelindent,#3:labelsep
\def\l@subsection{\@dottedtocline{2}{2.75em}{3.75em}}
\def\l@subsubsection{\@dottedtocline{3}{6.5em}{4.5em}}
% must provide \l@ defs for ALL sublevels EVEN if tocdepth
% is such as they will not appear in the table of contents
% these defs are how TOC knows what level these things are!
\def\l@paragraph{\@dottedtocline{4}{6.5em}{5.5em}}
\def\l@subparagraph{\@dottedtocline{5}{6.5em}{6.5em}}
\def\listoffigures{\section*{List of Figures}\@starttoc{lof}}
\def\l@figure{\@dottedtocline{1}{0em}{2.75em}}
\def\listoftables{\section*{List of Tables}\@starttoc{lot}}
\let\l@table\l@figure


%% Definitions for floats
%%
%% Normal Floats
\floatsep 1\baselineskip plus  0.2\baselineskip minus  0.2\baselineskip
\textfloatsep 1.7\baselineskip plus  0.2\baselineskip minus  0.4\baselineskip
\@fptop 0pt plus 1fil
\@fpsep 0.75\baselineskip plus 2fil 
\@fpbot 0pt plus 1fil
\def\topfraction{1.0}
\def\bottomfraction{.4}
\def\floatpagefraction{0.8}
\def\textfraction{.2}

%% Double Column Floats
\dblfloatsep 1\baselineskip plus  0.2\baselineskip minus  0.2\baselineskip

\dbltextfloatsep 1.7\baselineskip plus  0.2\baselineskip minus  0.4\baselineskip
% Note that it would be nice if the rubber here actually worked in LaTeX2e.
% There is a long standing limitation in LaTeX, first discovered (to the best
% of my knowledge) by Alan Jeffrey in 1992. LaTeX ignores the stretchable
% portion of \dbltextfloatsep, and as a result, double column figures can and
% do result in an non-integer number of lines in the main text columns with
% underfull vbox errors as a consequence. A post to comp.text.tex
% by Donald Arseneau confirms that this had not yet been fixed in 1998.
% IEEEtran V1.6 will fix this problem for you in the titles, but it doesn't
% protect you from other double floats. Happy vspace'ing.

\@dblfptop 0pt plus 1fil
\@dblfpsep 0.75\baselineskip plus 2fil
\@dblfpbot 0pt plus 1fil
\def\dbltopfraction{1.0}
\def\dblfloatpagefraction{0.8}
\setcounter{dbltopnumber}{4}

\intextsep 1\baselineskip plus 0.2\baselineskip minus  0.2\baselineskip
\setcounter{topnumber}{2}
\setcounter{bottomnumber}{2}
\setcounter{totalnumber}{4}


%% redefine CAPTION
% V1.4 add user control for short figure caption justification
\newif\ifcenterfigcaptions

% V1.6 set the default according to conference mode
\if@confmode
\centerfigcaptionstrue
\else
\centerfigcaptionsfalse
\fi

% article class provides these, we should too.
\newlength\abovecaptionskip
\newlength\belowcaptionskip
% but only \abovecaptionskip is used above figure captions and *below* table
% captions
\setlength\abovecaptionskip{0.5\baselineskip}
\setlength\belowcaptionskip{0pt}
% V1.6 create hooks in case the caption spacing ever needs to be
% overridden by a user
\def\@IEEEfigurecaptionsepspace{\vskip\abovecaptionskip\relax}%
\def\@IEEEtablecaptionsepspace{\vskip\abovecaptionskip\relax}%


% 1.6b revise caption system so that \@makecaption uses two arguments
% as with LaTeX2e. Otherwise, there will be problems when using hyperref.
\def\@IEEEtablestring{table}

\long\def\@makecaption#1#2{%
% test if is a for a figure or table
\ifx\@captype\@IEEEtablestring%
% if a table, do table caption
\begin{center}{\footnotesize #1}\\{\footnotesize\scshape #2}\end{center}%
\@IEEEtablecaptionsepspace% V1.6 was a hard coded 8pt
% if not a table, format it as a figure
\else
\@IEEEfigurecaptionsepspace% V1.6 was a hard coded 5pt
% 3/2001 use footnotesize, not small; use two nonbreaking spaces, not one
\setbox\@tempboxa\hbox{\footnotesize #1.~~ #2}%
\ifdim \wd\@tempboxa >\hsize%
% if caption is longer than a line, let it wrap around
\setbox\@tempboxa\hbox{\footnotesize #1.~~ }%
\parbox[t]{\hsize}{\footnotesize \noindent\unhbox\@tempboxa#2}%
% if caption is shorter than a line,
% allow user to control short figure caption justification (left or center)
\else%
\ifcenterfigcaptions \hbox to\hsize{\footnotesize\hfil\box\@tempboxa\hfil}%
\else \hbox to\hsize{\footnotesize\box\@tempboxa\hfil}%
\fi\fi\fi}


\newcounter{figure}
\def\thefigure{\@arabic\c@figure}
\def\fps@figure{tbp}
\def\ftype@figure{1}
\def\ext@figure{lof}
\def\fnum@figure{Fig.~\thefigure}
\def\figure{\@float{figure}}
\def\endfigure{\end@float}
\@namedef{figure*}{\@dblfloat{figure}}
\@namedef{endfigure*}{\end@dblfloat}
\newcounter{table}
\def\thetable{\@Roman\c@table}
\def\fps@table{tbp}
\def\ftype@table{2}
\def\ext@table{lot}
\def\fnum@table{TABLE~\thetable}
% V1.6 IEEE uses 8pt text for tables
% to default to footnotesize, we hack into LaTeX2e's \@floatboxreset and pray
\def\table{\def\@floatboxreset{\reset@font\footnotesize\@setminipage}\@float{table}}
\def\endtable{\end@float}
% v1.6b double column tables need to default to footnotesize as well.
\@namedef{table*}{\def\@floatboxreset{\reset@font\footnotesize\@setminipage}\@dblfloat{table}}
\@namedef{endtable*}{\end@dblfloat}




%%
%% START OF IEEEeqnarry DEFINITIONS
%%
%% Inspired by the concepts, examples, and previous works of LaTeX 
%% coders and developers such as Donald Arseneau, Fred Bartlett, 
%% David Carlisle, Tony Liu, Frank Mittelbach, Piet van Oostrum, 
%% Roland Winkler and Mark Wooding.
%% I don't make the claim that my work here is even near their calibre. ;)


% hook to allow easy changeover to IEEEtran.cls/tools.sty error reporting
\def\@IEEEclspkgerror{\ClassError{IEEEtran}}

\newif\if@IEEEeqnarraystarform% flag to indicate if the environment was called as the star form
\@IEEEeqnarraystarformfalse

\newif\if@advanceIEEEeqncolcnt% tracks if the environment should advance the col counter
% allows a way to make an \IEEEeqnarraybox that can be used within an \IEEEeqnarray
% used by IEEEeqnarraymulticol so that it can work properly in both
\@advanceIEEEeqncolcnttrue

\newcount\@IEEEeqnnumcols % tracks how many IEEEeqnarray cols are defined
\newcount\@IEEEeqncolcnt  % tracks how many IEEEeqnarray cols the user actually used


% The default math style used by the columns
\def\IEEEeqnarraymathstyle{\displaystyle}
% The default text style used by the columns
% default to using the current font
\def\IEEEeqnarraytextstyle{\relax}

% like the iedlistdecl but for \IEEEeqnarray
\def\IEEEeqnarraydecl{\relax}
\def\IEEEeqnarrayboxdecl{\relax}

% \yesnumber is the opposite of \nonumber
% a novel concept with the same def as the equationarray package
% However, we give IEEE versions too since some LaTeX packages such as 
% the MDWtools mathenv.sty redefine \nonumber to something else.
\providecommand{\yesnumber}{\global\@eqnswtrue}
\def\IEEEyesnumber{\global\@eqnswtrue}
\def\IEEEnonumber{\global\@eqnswfalse}


\def\IEEEyessubnumber{\global\@IEEEissubequationtrue\global\@eqnswtrue%
\if@IEEEeqnarrayISinner% only do something inside an IEEEeqnarray
\if@IEEElastlinewassubequation\addtocounter{equation}{-1}\else\setcounter{IEEEsubequation}{1}\fi%
\def\@currentlabel{\p@IEEEsubequation\theIEEEsubequation}\fi}

% flag to indicate that an equation is a sub equation
\newif\if@IEEEissubequation%
\@IEEEissubequationfalse

% allows users to "push away" equations that get too close to the equation numbers
\def\IEEEeqnarraynumspace{\hphantom{\if@IEEEissubequation\theIEEEsubequationdis\else\theequationdis\fi}}

% provides a way to span multiple columns within IEEEeqnarray environments
% will consider \if@advanceIEEEeqncolcnt before globally advancing the
% column counter - so as to work within \IEEEeqnarraybox
% usage: \IEEEeqnarraymulticol{number cols. to span}{col type}{cell text}
\long\def\IEEEeqnarraymulticol#1#2#3{\multispan{#1}%
% check if column is defined
\relax\expandafter\ifx\csname @IEEEeqnarraycolDEF#2\endcsname\@IEEEeqnarraycolisdefined%
\csname @IEEEeqnarraycolPRE#2\endcsname#3\relax\relax\relax\relax\relax%
\relax\relax\relax\relax\relax\csname @IEEEeqnarraycolPOST#2\endcsname%
\else% if not, error and use default type
\@IEEEclspkgerror{Invalid column type "#2" in \string\IEEEeqnarraymulticol.\MessageBreak
Using a default centering column instead}%
{You must define IEEEeqnarray column types before use.}%
\csname @IEEEeqnarraycolPRE@IEEEdefault\endcsname#3\relax\relax\relax\relax\relax%
\relax\relax\relax\relax\relax\csname @IEEEeqnarraycolPOST@IEEEdefault\endcsname%
\fi%
% advance column counter only if the IEEEeqnarray environment wants it
\if@advanceIEEEeqncolcnt\global\advance\@IEEEeqncolcnt by #1\relax\fi}

% like \omit, but maintains track of the column counter for \IEEEeqnarray
\def\IEEEeqnarrayomit{\omit\if@advanceIEEEeqncolcnt\global\advance\@IEEEeqncolcnt by 1\relax\fi}


% provides a way to define a letter referenced column type
% usage: \IEEEeqnarraydefcol{col. type letter/name}{pre insertion text}{post insertion text}
\def\IEEEeqnarraydefcol#1#2#3{\expandafter\def\csname @IEEEeqnarraycolPRE#1\endcsname{#2}%
\expandafter\def\csname @IEEEeqnarraycolPOST#1\endcsname{#3}%
\expandafter\def\csname @IEEEeqnarraycolDEF#1\endcsname{1}}


% provides a way to define a numerically referenced inter-column glue types
% usage: \IEEEeqnarraydefcolsep{col. glue number}{glue definition}
\def\IEEEeqnarraydefcolsep#1#2{\expandafter\def\csname @IEEEeqnarraycolSEP\romannumeral #1\endcsname{#2}%
\expandafter\def\csname @IEEEeqnarraycolSEPDEF\romannumeral #1\endcsname{1}}


\def\@IEEEeqnarraycolisdefined{1}% just a macro for 1, used for checking undefined column types


% expands and appends the given argument to the \@IEEEtrantmptoksA token list
% used to build up the \halign preamble
\def\@IEEEappendtoksA#1{\edef\@@IEEEappendtoksA{\@IEEEtrantmptoksA={\the\@IEEEtrantmptoksA #1}}%
\@@IEEEappendtoksA}

% also appends to \@IEEEtrantmptoksA, but does not expand the argument
% uses \toks8 as a scratchpad register
\def\@IEEEappendNOEXPANDtoksA#1{\toks8={#1}%
\edef\@@IEEEappendNOEXPANDtoksA{\@IEEEtrantmptoksA={\the\@IEEEtrantmptoksA\the\toks8}}%
\@@IEEEappendNOEXPANDtoksA}

% define some common column types for the user
% math
\IEEEeqnarraydefcol{l}{$\IEEEeqnarraymathstyle}{$\hfil}
\IEEEeqnarraydefcol{c}{\hfil$\IEEEeqnarraymathstyle}{$\hfil}
\IEEEeqnarraydefcol{r}{\hfil$\IEEEeqnarraymathstyle}{$}
\IEEEeqnarraydefcol{L}{$\IEEEeqnarraymathstyle{}}{{}$\hfil}
\IEEEeqnarraydefcol{C}{\hfil$\IEEEeqnarraymathstyle{}}{{}$\hfil}
\IEEEeqnarraydefcol{R}{\hfil$\IEEEeqnarraymathstyle{}}{{}$}
% text
\IEEEeqnarraydefcol{s}{\IEEEeqnarraytextstyle}{\hfil}
\IEEEeqnarraydefcol{t}{\hfil\IEEEeqnarraytextstyle}{\hfil}
\IEEEeqnarraydefcol{u}{\hfil\IEEEeqnarraytextstyle}{}

% vertical rules
\IEEEeqnarraydefcol{v}{}{\vrule width\arrayrulewidth}
\IEEEeqnarraydefcol{vv}{\vrule width\arrayrulewidth\hfil}{\hfil\vrule width\arrayrulewidth}
\IEEEeqnarraydefcol{V}{}{\vrule width\arrayrulewidth\hskip\doublerulesep\vrule width\arrayrulewidth}
\IEEEeqnarraydefcol{VV}{\vrule width\arrayrulewidth\hskip\doublerulesep\vrule width\arrayrulewidth\hfil}%
{\hfil\vrule width\arrayrulewidth\hskip\doublerulesep\vrule width\arrayrulewidth}

% horizontal rules
\IEEEeqnarraydefcol{h}{}{\leaders\hrule height\arrayrulewidth\hfil}
\IEEEeqnarraydefcol{H}{}{\leaders\vbox{\hrule width\arrayrulewidth\vskip\doublerulesep\hrule width\arrayrulewidth}\hfil}

% plain
\IEEEeqnarraydefcol{x}{}{}
\IEEEeqnarraydefcol{X}{$}{$}

% the default column type to use in the event a column type is not defined
\IEEEeqnarraydefcol{@IEEEdefault}{\hfil$\IEEEeqnarraymathstyle}{$\hfil}


% a zero tabskip (used for "-" col types)
\def\@IEEEeqnarraycolSEPzero{0pt plus 0pt minus 0pt}
% a centering tabskip (used for "+" col types)
\def\@IEEEeqnarraycolSEPcenter{1000pt plus 0pt minus 1000pt}

% top level default tabskip glues for the start, end, and inter-column
% may be reset within environments not always at the top level, e.g., \IEEEeqnarraybox
\edef\@IEEEeqnarraycolSEPdefaultstart{\@IEEEeqnarraycolSEPcenter}% default start glue
\edef\@IEEEeqnarraycolSEPdefaultend{\@IEEEeqnarraycolSEPcenter}% default end glue
\edef\@IEEEeqnarraycolSEPdefaultmid{\@IEEEeqnarraycolSEPzero}% default inter-column glue



% creates a vertical rule that extends from the bottom to the top a a cell
% Provided in case other packages redefine \vline some other way.
% usage: \IEEEeqnarrayvrule[rule thickness]
% If no argument is provided, \arrayrulewidth will be used for the rule thickness. 
\newcommand\IEEEeqnarrayvrule[1][\arrayrulewidth]{\vrule\@width#1\relax}

% creates a blank separator row
% usage: \IEEEeqnarrayseprow[separation length][font size commands]
% default is \IEEEeqnarrayseprow[0.25\normalbaselineskip][\relax]
% blank arguments inherit the default values
% uses \skip5 as a scratch register - calls \@IEEEeqnarraystrutsize which uses more scratch registers
\def\IEEEeqnarrayseprow{\relax\@ifnextchar[{\@IEEEeqnarrayseprow}{\@IEEEeqnarrayseprow[0.25\normalbaselineskip]}}
\def\@IEEEeqnarrayseprow[#1]{\relax\@ifnextchar[{\@@IEEEeqnarrayseprow[#1]}{\@@IEEEeqnarrayseprow[#1][\relax]}}
\def\@@IEEEeqnarrayseprow[#1][#2]{\def\@IEEEeqnarrayseprowARGONE{#1}%
\ifx\@IEEEeqnarrayseprowARGONE\@empty%
% get the skip value, based on the font commands
% use skip5 because \IEEEeqnarraystrutsize uses \skip0, \skip2, \skip3
% assign within a bogus box to confine the font changes
{\setbox0=\hbox{#2\relax\global\skip5=0.25\normalbaselineskip}}%
\else%
{\setbox0=\hbox{#2\relax\global\skip5=#1}}%
\fi%
\@IEEEeqnarrayhoptolastcolumn\IEEEeqnarraystrutsize{\skip5}{0pt}[\relax]\relax}

% creates a blank separator row, but omits all the column templates
% usage: \IEEEeqnarrayseprowcut[separation length][font size commands]
% default is \IEEEeqnarrayseprowcut[0.25\normalbaselineskip][\relax]
% blank arguments inherit the default values
% uses \skip5 as a scratch register - calls \@IEEEeqnarraystrutsize which uses more scratch registers
\def\IEEEeqnarrayseprowcut{\multispan{\@IEEEeqnnumcols}\relax% span all the cols
% advance column counter only if the IEEEeqnarray environment wants it
\if@advanceIEEEeqncolcnt\global\advance\@IEEEeqncolcnt by \@IEEEeqnnumcols\relax\fi%
\@ifnextchar[{\@IEEEeqnarrayseprowcut}{\@IEEEeqnarrayseprowcut[0.25\normalbaselineskip]}}
\def\@IEEEeqnarrayseprowcut[#1]{\relax\@ifnextchar[{\@@IEEEeqnarrayseprowcut[#1]}{\@@IEEEeqnarrayseprowcut[#1][\relax]}}
\def\@@IEEEeqnarrayseprowcut[#1][#2]{\def\@IEEEeqnarrayseprowARGONE{#1}%
\ifx\@IEEEeqnarrayseprowARGONE\@empty%
% get the skip value, based on the font commands
% use skip5 because \IEEEeqnarraystrutsize uses \skip0, \skip2, \skip3
% assign within a bogus box to confine the font changes
{\setbox0=\hbox{#2\relax\global\skip5=0.25\normalbaselineskip}}%
\else%
{\setbox0=\hbox{#2\relax\global\skip5=#1}}%
\fi%
\IEEEeqnarraystrutsize{\skip5}{0pt}[\relax]\relax}



% draws a single rule across all the columns optional
% argument determines the rule width, \arrayrulewidth is the default
% updates column counter as needed and turns off struts
% usage: \IEEEeqnarrayrulerow[rule line thickness]
\def\IEEEeqnarrayrulerow{\multispan{\@IEEEeqnnumcols}\relax% span all the cols
% advance column counter only if the IEEEeqnarray environment wants it
\if@advanceIEEEeqncolcnt\global\advance\@IEEEeqncolcnt by \@IEEEeqnnumcols\relax\fi%
\@ifnextchar[{\@IEEEeqnarrayrulerow}{\@IEEEeqnarrayrulerow[\arrayrulewidth]}}
\def\@IEEEeqnarrayrulerow[#1]{\leaders\hrule height#1\hfil\relax% put in our rule 
% turn off any struts
\IEEEeqnarraystrutsize{0pt}{0pt}[\relax]\relax}


% draws a double rule by using a single rule row, a separator row, and then
% another single rule row 
% first optional argument determines the rule thicknesses, \arrayrulewidth is the default
% second optional argument determines the rule spacing, \doublerulesep is the default
% usage: \IEEEeqnarraydblrulerow[rule line thickness][rule spacing]
\def\IEEEeqnarraydblrulerow{\multispan{\@IEEEeqnnumcols}\relax% span all the cols
% advance column counter only if the IEEEeqnarray environment wants it
\if@advanceIEEEeqncolcnt\global\advance\@IEEEeqncolcnt by \@IEEEeqnnumcols\relax\fi%
\@ifnextchar[{\@IEEEeqnarraydblrulerow}{\@IEEEeqnarraydblrulerow[\arrayrulewidth]}}
\def\@IEEEeqnarraydblrulerow[#1]{\relax\@ifnextchar[{\@@IEEEeqnarraydblrulerow[#1]}%
{\@@IEEEeqnarraydblrulerow[#1][\doublerulesep]}}
\def\@@IEEEeqnarraydblrulerow[#1][#2]{\def\@IEEEeqnarraydblrulerowARG{#1}%
% we allow the user to say \IEEEeqnarraydblrulerow[][]
\ifx\@IEEEeqnarraydblrulerowARG\@empty%
\@IEEEeqnarrayrulerow[\arrayrulewidth]%
\else%
\@IEEEeqnarrayrulerow[#1]\relax%
\fi%
\def\@IEEEeqnarraydblrulerowARG{#2}%
\ifx\@IEEEeqnarraydblrulerowARG\@empty%
\\\IEEEeqnarrayseprow[\doublerulesep][\relax]%
\else%
\\\IEEEeqnarrayseprow[#2][\relax]%
\fi%
\\\multispan{\@IEEEeqnnumcols}%
% advance column counter only if the IEEEeqnarray environment wants it
\if@advanceIEEEeqncolcnt\global\advance\@IEEEeqncolcnt by \@IEEEeqnnumcols\relax\fi%
\def\@IEEEeqnarraydblrulerowARG{#1}%
\ifx\@IEEEeqnarraydblrulerowARG\@empty%
\@IEEEeqnarrayrulerow[\arrayrulewidth]%
\else%
\@IEEEeqnarrayrulerow[#1]%
\fi%
}

% draws a double rule by using a single rule row, a separator (cutting) row, and then
% another single rule row 
% first optional argument determines the rule thicknesses, \arrayrulewidth is the default
% second optional argument determines the rule spacing, \doublerulesep is the default
% usage: \IEEEeqnarraydblrulerow[rule line thickness][rule spacing]
\def\IEEEeqnarraydblrulerowcut{\multispan{\@IEEEeqnnumcols}\relax% span all the cols
% advance column counter only if the IEEEeqnarray environment wants it
\if@advanceIEEEeqncolcnt\global\advance\@IEEEeqncolcnt by \@IEEEeqnnumcols\relax\fi%
\@ifnextchar[{\@IEEEeqnarraydblrulerowcut}{\@IEEEeqnarraydblrulerowcut[\arrayrulewidth]}}
\def\@IEEEeqnarraydblrulerowcut[#1]{\relax\@ifnextchar[{\@@IEEEeqnarraydblrulerowcut[#1]}%
{\@@IEEEeqnarraydblrulerowcut[#1][\doublerulesep]}}
\def\@@IEEEeqnarraydblrulerowcut[#1][#2]{\def\@IEEEeqnarraydblrulerowARG{#1}%
% we allow the user to say \IEEEeqnarraydblrulerow[][]
\ifx\@IEEEeqnarraydblrulerowARG\@empty%
\@IEEEeqnarrayrulerow[\arrayrulewidth]%
\else%
\@IEEEeqnarrayrulerow[#1]%
\fi%
\def\@IEEEeqnarraydblrulerowARG{#2}%
\ifx\@IEEEeqnarraydblrulerowARG\@empty%
\\\IEEEeqnarrayseprowcut[\doublerulesep][\relax]%
\else%
\\\IEEEeqnarrayseprowcut[#2][\relax]%
\fi%
\\\multispan{\@IEEEeqnnumcols}%
% advance column counter only if the IEEEeqnarray environment wants it
\if@advanceIEEEeqncolcnt\global\advance\@IEEEeqncolcnt by \@IEEEeqnnumcols\relax\fi%
\def\@IEEEeqnarraydblrulerowARG{#1}%
\ifx\@IEEEeqnarraydblrulerowARG\@empty%
\@IEEEeqnarrayrulerow[\arrayrulewidth]%
\else%
\@IEEEeqnarrayrulerow[#1]%
\fi%
}



% inserts a full row's worth of &'s
% relies on \@IEEEeqnnumcols to provide the correct number of columns
% uses \@IEEEtrantmptoksA, \count0 as scratch registers
\def\@IEEEeqnarrayhoptolastcolumn{\@IEEEtrantmptoksA={}\count0=1\relax%
\loop% add cols if the user did not use them all
\ifnum\count0<\@IEEEeqnnumcols\relax%
\@IEEEappendtoksA{&}%
\advance\count0 by 1\relax% update the col count
\repeat%
\the\@IEEEtrantmptoksA%execute the &'s
}



\newif\if@IEEEeqnarrayISinner % flag to indicate if we are within the lines
\@IEEEeqnarrayISinnerfalse    % of an IEEEeqnarray - after the IEEEeqnarraydecl

\edef\@IEEEeqnarrayTHEstrutheight{0pt} % height and depth of IEEEeqnarray struts
\edef\@IEEEeqnarrayTHEstrutdepth{0pt}

\edef\@IEEEeqnarrayTHEmasterstrutheight{0pt} % default height and depth of
\edef\@IEEEeqnarrayTHEmasterstrutdepth{0pt}  % struts within an IEEEeqnarray

\edef\@IEEEeqnarrayTHEmasterstrutHSAVE{0pt} % saved master strut height
\edef\@IEEEeqnarrayTHEmasterstrutDSAVE{0pt} % and depth

\newif\if@IEEEeqnarrayusemasterstrut % flag to indicate that the master strut value
\@IEEEeqnarrayusemasterstruttrue     % is to be used



% saves the strut height and depth of the master strut
\def\@IEEEeqnarraymasterstrutsave{\relax%
\expandafter\skip0=\@IEEEeqnarrayTHEmasterstrutheight\relax%
\expandafter\skip2=\@IEEEeqnarrayTHEmasterstrutdepth\relax%
% remove stretchability
\dimen0\skip0\relax%
\dimen2\skip2\relax%
% save values
\edef\@IEEEeqnarrayTHEmasterstrutHSAVE{\the\dimen0}%
\edef\@IEEEeqnarrayTHEmasterstrutDSAVE{\the\dimen2}}

% restores the strut height and depth of the master strut
\def\@IEEEeqnarraymasterstrutrestore{\relax%
\expandafter\skip0=\@IEEEeqnarrayTHEmasterstrutHSAVE\relax%
\expandafter\skip2=\@IEEEeqnarrayTHEmasterstrutDSAVE\relax%
% remove stretchability
\dimen0\skip0\relax%
\dimen2\skip2\relax%
% restore values
\edef\@IEEEeqnarrayTHEmasterstrutheight{\the\dimen0}%
\edef\@IEEEeqnarrayTHEmasterstrutdepth{\the\dimen2}}


% globally restores the strut height and depth to the 
% master values and sets the master strut flag to true
\def\@IEEEeqnarraystrutreset{\relax%
\expandafter\skip0=\@IEEEeqnarrayTHEmasterstrutheight\relax%
\expandafter\skip2=\@IEEEeqnarrayTHEmasterstrutdepth\relax%
% remove stretchability
\dimen0\skip0\relax%
\dimen2\skip2\relax%
% restore values
\xdef\@IEEEeqnarrayTHEstrutheight{\the\dimen0}%
\xdef\@IEEEeqnarrayTHEstrutdepth{\the\dimen2}%
\global\@IEEEeqnarrayusemasterstruttrue}


% if the master strut is not to be used, make the current
% values of \@IEEEeqnarrayTHEstrutheight, \@IEEEeqnarrayTHEstrutdepth
% and the use master strut flag, global
% this allows user strut commands issued in the last column to be carried
% into the isolation/strut column
\def\@IEEEeqnarrayglobalizestrutstatus{\relax%
\if@IEEEeqnarrayusemasterstrut\else%
\xdef\@IEEEeqnarrayTHEstrutheight{\@IEEEeqnarrayTHEstrutheight}%
\xdef\@IEEEeqnarrayTHEstrutdepth{\@IEEEeqnarrayTHEstrutdepth}%
\global\@IEEEeqnarrayusemasterstrutfalse%
\fi}



% usage: \IEEEeqnarraystrutsize{height}{depth}[font size commands]
% If called outside the lines of an IEEEeqnarray, sets the height
% and depth of both the master and local struts. If called inside
% an IEEEeqnarray line, sets the height and depth of the local strut
% only and sets the flag to indicate the use of the local strut
% values. If the height or depth is left blank, 0.7\normalbaselineskip
% and 0.3\normalbaselineskip will be used, respectively.
% The optional argument can be used to evaluate the lengths under
% a different font size and styles. If none is specified, the current
% font is used.
% uses scratch registers \skip0, \skip2, \skip3, \dimen0, \dimen2
\def\IEEEeqnarraystrutsize#1#2{\relax\@ifnextchar[{\@IEEEeqnarraystrutsize{#1}{#2}}{\@IEEEeqnarraystrutsize{#1}{#2}[\relax]}}
\def\@IEEEeqnarraystrutsize#1#2[#3]{\def\@IEEEeqnarraystrutsizeARG{#1}%
\ifx\@IEEEeqnarraystrutsizeARG\@empty%
{\setbox0=\hbox{#3\relax\global\skip3=0.7\normalbaselineskip}}%
\skip0=\skip3\relax%
\else% arg one present
{\setbox0=\hbox{#3\relax\global\skip3=#1\relax}}%
\skip0=\skip3\relax%
\fi% if null arg
\def\@IEEEeqnarraystrutsizeARG{#2}%
\ifx\@IEEEeqnarraystrutsizeARG\@empty%
{\setbox0=\hbox{#3\relax\global\skip3=0.3\normalbaselineskip}}%
\skip2=\skip3\relax%
\else% arg two present
{\setbox0=\hbox{#3\relax\global\skip3=#2\relax}}%
\skip2=\skip3\relax%
\fi% if null arg
% remove stretchability, just to be safe
\dimen0\skip0\relax%
\dimen2\skip2\relax%
% dimen0 = height, dimen2 = depth
\if@IEEEeqnarrayISinner% inner does not touch master strut size
\edef\@IEEEeqnarrayTHEstrutheight{\the\dimen0}%
\edef\@IEEEeqnarrayTHEstrutdepth{\the\dimen2}%
\@IEEEeqnarrayusemasterstrutfalse% do not use master
\else% outer, have to set master strut too
\edef\@IEEEeqnarrayTHEmasterstrutheight{\the\dimen0}%
\edef\@IEEEeqnarrayTHEmasterstrutdepth{\the\dimen2}%
\edef\@IEEEeqnarrayTHEstrutheight{\the\dimen0}%
\edef\@IEEEeqnarrayTHEstrutdepth{\the\dimen2}%
\@IEEEeqnarrayusemasterstruttrue% use master strut
\fi}


% usage: \IEEEeqnarraystrutsizeadd{added height}{added depth}[font size commands]
% If called outside the lines of an IEEEeqnarray, adds the given height
% and depth to both the master and local struts.
% If called inside an IEEEeqnarray line, adds the given height and depth
% to the local strut only and sets the flag to indicate the use 
% of the local strut values.
% In both cases, if a height or depth is left blank, 0pt is used instead.
% The optional argument can be used to evaluate the lengths under
% a different font size and styles. If none is specified, the current
% font is used.
% uses scratch registers \skip0, \skip2, \skip3, \dimen0, \dimen2
\def\IEEEeqnarraystrutsizeadd#1#2{\relax\@ifnextchar[{\@IEEEeqnarraystrutsizeadd{#1}{#2}}{\@IEEEeqnarraystrutsizeadd{#1}{#2}[\relax]}}
\def\@IEEEeqnarraystrutsizeadd#1#2[#3]{\def\@IEEEeqnarraystrutsizearg{#1}%
\ifx\@IEEEeqnarraystrutsizearg\@empty%
\skip0=0pt\relax%
\else% arg one present
{\setbox0=\hbox{#3\relax\global\skip3=#1}}%
\skip0=\skip3\relax%
\fi% if null arg
\def\@IEEEeqnarraystrutsizearg{#2}%
\ifx\@IEEEeqnarraystrutsizearg\@empty%
\skip2=0pt\relax%
\else% arg two present
{\setbox0=\hbox{#3\relax\global\skip3=#2}}%
\skip2=\skip3\relax%
\fi% if null arg
% remove stretchability, just to be safe
\dimen0\skip0\relax%
\dimen2\skip2\relax%
% dimen0 = height, dimen2 = depth
\if@IEEEeqnarrayISinner% inner does not touch master strut size
% get local strut size
\expandafter\skip0=\@IEEEeqnarrayTHEstrutheight\relax%
\expandafter\skip2=\@IEEEeqnarrayTHEstrutdepth\relax%
% add it to the user supplied values
\advance\dimen0 by \skip0\relax%
\advance\dimen2 by \skip2\relax%
% update the local strut size
\edef\@IEEEeqnarrayTHEstrutheight{\the\dimen0}%
\edef\@IEEEeqnarrayTHEstrutdepth{\the\dimen2}%
\@IEEEeqnarrayusemasterstrutfalse% do not use master
\else% outer, have to set master strut too
% get master strut size
\expandafter\skip0=\@IEEEeqnarrayTHEmasterstrutheight\relax%
\expandafter\skip2=\@IEEEeqnarrayTHEmasterstrutdepth\relax%
% add it to the user supplied values
\advance\dimen0 by \skip0\relax%
\advance\dimen2 by \skip2\relax%
% update the local and master strut sizes
\edef\@IEEEeqnarrayTHEmasterstrutheight{\the\dimen0}%
\edef\@IEEEeqnarrayTHEmasterstrutdepth{\the\dimen2}%
\edef\@IEEEeqnarrayTHEstrutheight{\the\dimen0}%
\edef\@IEEEeqnarrayTHEstrutdepth{\the\dimen2}%
\@IEEEeqnarrayusemasterstruttrue% use master strut
\fi}


% allow user a way to see the struts
\newif\ifIEEEvisiblestruts
\IEEEvisiblestrutsfalse

% inserts an invisible strut using the master or local strut values
% uses scratch registers \skip0, \skip2, \dimen0, \dimen2
\def\@IEEEeqnarrayinsertstrut{\relax%
\if@IEEEeqnarrayusemasterstrut
% get master strut size
\expandafter\skip0=\@IEEEeqnarrayTHEmasterstrutheight\relax%
\expandafter\skip2=\@IEEEeqnarrayTHEmasterstrutdepth\relax%
\else%
% get local strut size
\expandafter\skip0=\@IEEEeqnarrayTHEstrutheight\relax%
\expandafter\skip2=\@IEEEeqnarrayTHEstrutdepth\relax%
\fi%
% remove stretchability, probably not needed
\dimen0\skip0\relax%
\dimen2\skip2\relax%
% dimen0 = height, dimen2 = depth
% allow user to see struts if desired
\ifIEEEvisiblestruts%
\vrule width0.2pt height\dimen0 depth\dimen2\relax%
\else%
\vrule width0pt height\dimen0 depth\dimen2\relax\fi}


% creates an invisible strut, useable even outside \IEEEeqnarray
% if \IEEEvisiblestrutstrue, the strut will be visible and 0.2pt wide. 
% usage: \IEEEstrut[height][depth][font size commands]
% default is \IEEEstrut[0.7\normalbaselineskip][0.3\normalbaselineskip][\relax]
% blank arguments inherit the default values
% uses \dimen0, \dimen2, \skip0, \skip2
\def\IEEEstrut{\relax\@ifnextchar[{\@IEEEstrut}{\@IEEEstrut[0.7\normalbaselineskip]}}
\def\@IEEEstrut[#1]{\relax\@ifnextchar[{\@@IEEEstrut[#1]}{\@@IEEEstrut[#1][0.3\normalbaselineskip]}}
\def\@@IEEEstrut[#1][#2]{\relax\@ifnextchar[{\@@@IEEEstrut[#1][#2]}{\@@@IEEEstrut[#1][#2][\relax]}}
\def\@@@IEEEstrut[#1][#2][#3]{\mbox{#3\relax%
\def\@IEEEstrutARG{#1}%
\ifx\@IEEEstrutARG\@empty%
\skip0=0.7\normalbaselineskip\relax%
\else%
\skip0=#1\relax%
\fi%
\def\@IEEEstrutARG{#2}%
\ifx\@IEEEstrutARG\@empty%
\skip2=0.3\normalbaselineskip\relax%
\else%
\skip2=#2\relax%
\fi%
% remove stretchability, probably not needed
\dimen0\skip0\relax%
\dimen2\skip2\relax%
\ifIEEEvisiblestruts%
\vrule width0.2pt height\dimen0 depth\dimen2\relax%
\else%
\vrule width0.0pt height\dimen0 depth\dimen2\relax\fi}}


% enables strut mode by setting a default strut size and then zeroing the
% \baselineskip, \lineskip, \lineskiplimit and \jot
\def\IEEEeqnarraystrutmode{\IEEEeqnarraystrutsize{0.7\normalbaselineskip}{0.3\normalbaselineskip}[\relax]%
\baselineskip=0pt\lineskip=0pt\lineskiplimit=0pt\jot=0pt}



\def\IEEEeqnarray{\@IEEEeqnarraystarformfalse\@IEEEeqnarray}
\def\endIEEEeqnarray{\end@IEEEeqnarray}

\@namedef{IEEEeqnarray*}{\@IEEEeqnarraystarformtrue\@IEEEeqnarray}
\@namedef{endIEEEeqnarray*}{\end@IEEEeqnarray}


% \IEEEeqnarray is an enhanced \eqnarray. 
% The star form defaults to not putting equation numbers at the end of each row.
% usage: \IEEEeqnarray[decl]{cols}
\def\@IEEEeqnarray{\relax\@ifnextchar[{\@@IEEEeqnarray}{\@@IEEEeqnarray[\relax]}}
\def\@@IEEEeqnarray[#1]#2{%
   % default to showing the equation number or not based on whether or not
   % the star form was involked
   \if@IEEEeqnarraystarform\global\@eqnswfalse
   \else% not the star form
   \global\@eqnswtrue
   \fi% if star form
   \@IEEEissubequationfalse% default to no subequations
   \@IEEElastlinewassubequationfalse% assume last line is not a sub equation
   \@IEEEeqnarrayISinnerfalse% not yet within the lines of the halign
   \@IEEEeqnarraystrutsize{0pt}{0pt}[\relax]% turn off struts by default
   \@IEEEeqnarrayusemasterstruttrue% use master strut till user asks otherwise
   \IEEEvisiblestrutsfalse% diagnostic mode defaults to off
   % no extra space unless the user specifically requests it
   \lineskip=0pt\relax
   \lineskiplimit=0pt\relax
   \baselineskip=\normalbaselineskip\relax%
   \jot=\normaljot\relax%
   \mathsurround\z@\relax% no extra spacing around math
   \@advanceIEEEeqncolcnttrue% advance the col counter for each col the user uses, 
                             % used in \IEEEeqnarraymulticol and in the preamble build
   \stepcounter{equation}% advance equation counter before first line
   \setcounter{IEEEsubequation}{0}% no subequation yet 
   \def\@currentlabel{\p@equation\theequation}% redefine the ref label
   \IEEEeqnarraydecl\relax% allow a way for the user to make global overrides
   #1\relax% allow user to override defaults
   \let\\\@IEEEeqnarraycr% replace newline with one that can put in eqn. numbers
   \global\@IEEEeqncolcnt\z@% col. count = 0 for first line
   \@IEEEbuildpreamble #2\end\relax% build the preamble and put it into \@IEEEtrantmptoksA 
   % put in the column for the equation number
   \ifnum\@IEEEeqnnumcols>0\relax\@IEEEappendtoksA{&}\fi% col separator for those after the first
   \toks0={##}%
   % advance the \@IEEEeqncolcnt for the isolation col, this helps with error checking
   \@IEEEappendtoksA{\global\advance\@IEEEeqncolcnt by 1\relax}%
   % add the isolation column
   \@IEEEappendtoksA{\tabskip\z@skip\bgroup\the\toks0\egroup}%
   % advance the \@IEEEeqncolcnt for the equation number col, this helps with error checking
   \@IEEEappendtoksA{&\global\advance\@IEEEeqncolcnt by 1\relax}%
   % add the equation number col to the preamble
   \@IEEEappendtoksA{\tabskip\z@skip\hb@xt@\z@\bgroup\hss\the\toks0\egroup}%
   % note \@IEEEeqnnumcols does not count the equation col or isolation col
   % set the starting tabskip glue as determined by the preamble build
   \tabskip=\@IEEEBPstartglue\relax
   % begin the display alignment
   \@IEEEeqnarrayISinnertrue% commands are now within the lines
   $$\everycr{}\halign to\displaywidth\bgroup
   % "exspand" the preamble
   \span\the\@IEEEtrantmptoksA\cr}

% enter isolation/strut column (or the next column if the user did not use
% every column), record the strut status, complete the columns, do the strut if needed,
% restore counters to correct values and exit
\def\end@IEEEeqnarray{\@IEEEeqnarrayglobalizestrutstatus&\@@IEEEeqnarraycr\egroup%
\if@IEEElastlinewassubequation\global\advance\c@IEEEsubequation\m@ne\fi%
\global\advance\c@equation\m@ne%
$$\@ignoretrue}

% need a way to remember if last line is a subequation
\newif\if@IEEElastlinewassubequation%
\@IEEElastlinewassubequationfalse

% IEEEeqnarray uses a modifed \\ instead of the plain \cr to
% end rows. This allows for things like \\*[vskip amount]
% This "cr" macros are modified versions those for LaTeX2e's eqnarray
% the {\ifnum0=`} braces must be kept away from the last column to avoid
% altering spacing of its math, so we use & to advance to the next column
% as there is an isolation/strut column after the user's columns
\def\@IEEEeqnarraycr{\@IEEEeqnarrayglobalizestrutstatus&% save strut status and advance to next column
   {\ifnum0=`}\fi
   \@ifstar{%
      \global\@eqpen\@M\@IEEEeqnarrayYCR
   }{%
      \global\@eqpen\interdisplaylinepenalty \@IEEEeqnarrayYCR
   }%
}

\def\@IEEEeqnarrayYCR{\@testopt\@IEEEeqnarrayXCR\z@skip}

\def\@IEEEeqnarrayXCR[#1]{%
   \ifnum0=`{\fi}%
   \@@IEEEeqnarraycr
   \noalign{\penalty\@eqpen\vskip\jot\vskip #1\relax}}%

\def\@@IEEEeqnarraycr{\@IEEEtrantmptoksA={}% clear token register
    \advance\@IEEEeqncolcnt by -1\relax% adjust col count because of the isolation column
    \ifnum\@IEEEeqncolcnt>\@IEEEeqnnumcols\relax
    \@IEEEclspkgerror{Too many columns within the IEEEeqnarray\MessageBreak
                          environment}%
    {Use fewer \string &'s or put more columns in the IEEEeqnarry column\MessageBreak 
     specifications.}\relax%
    \else
    \loop% add cols if the user did not use them all
    \ifnum\@IEEEeqncolcnt<\@IEEEeqnnumcols\relax
    \@IEEEappendtoksA{&}%
    \advance\@IEEEeqncolcnt by 1\relax% update the col count
    \repeat
    % this number of &'s will take us the the isolation column
    \fi
    % execute the &'s
    \the\@IEEEtrantmptoksA%
    % handle the strut/isolation column
    \@IEEEeqnarrayinsertstrut% do the strut if needed
    \@IEEEeqnarraystrutreset% reset the strut system for next line or IEEEeqnarray
    &% and enter the equation number column
    % is this line needs an equation number, display it and advance the
    % (sub)equation counters, record what type this line was
    \if@eqnsw%
     \if@IEEEissubequation\theIEEEsubequationdis\addtocounter{equation}{1}\stepcounter{IEEEsubequation}%
     \global\@IEEElastlinewassubequationtrue%
     \else% display a standard equation number, initialize the IEEEsubequation counter
     \theequationdis\stepcounter{equation}\setcounter{IEEEsubequation}{0}%
     \global\@IEEElastlinewassubequationfalse\fi%
    \fi%
    % reset the eqnsw flag to indicate default preference of the display of equation numbers
    \if@IEEEeqnarraystarform\global\@eqnswfalse\else\global\@eqnswtrue\fi
    \global\@IEEEissubequationfalse% reset the subequation flag
    % reset the number of columns the user actually used
    \global\@IEEEeqncolcnt\z@\relax
    % the real end of the line
    \cr}





% \IEEEeqnarraybox is like \IEEEeqnarray except the box form puts everything
% inside a vtop, vbox, or vcenter box depending on the letter in the second
% optional argument (t,b,c). Vbox is the default. Unlike \IEEEeqnarray,
% equation numbers are not displayed and \IEEEeqnarraybox can be nested.
% \IEEEeqnarrayboxm is for math mode (like \array) and does not put the vbox
% within an hbox.
% \IEEEeqnarrayboxt is for text mode (like \tabular) and puts the vbox within
% a \hbox{$ $} construct.
% \IEEEeqnarraybox will auto detect whether to use \IEEEeqnarrayboxm or 
% \IEEEeqnarrayboxt depending on the math mode.
% The third optional argument specifies the width this box is to be set to -
% natural width is the default.
% The * forms do not add \jot line spacing
% usage: \IEEEeqnarraybox[decl][pos][width]{cols}
\def\IEEEeqnarrayboxm{\@IEEEeqnarraystarformfalse\@IEEEeqnarrayboxHBOXSWfalse\@IEEEeqnarraybox}
\def\endIEEEeqnarrayboxm{\end@IEEEeqnarraybox}
\@namedef{IEEEeqnarrayboxm*}{\@IEEEeqnarraystarformtrue\@IEEEeqnarrayboxHBOXSWfalse\@IEEEeqnarraybox}
\@namedef{endIEEEeqnarrayboxm*}{\end@IEEEeqnarraybox}

\def\IEEEeqnarrayboxt{\@IEEEeqnarraystarformfalse\@IEEEeqnarrayboxHBOXSWtrue\@IEEEeqnarraybox}
\def\endIEEEeqnarrayboxt{\end@IEEEeqnarraybox}
\@namedef{IEEEeqnarrayboxt*}{\@IEEEeqnarraystarformtrue\@IEEEeqnarrayboxHBOXSWtrue\@IEEEeqnarraybox}
\@namedef{endIEEEeqnarrayboxt*}{\end@IEEEeqnarraybox}

\def\IEEEeqnarraybox{\@IEEEeqnarraystarformfalse\ifmmode\@IEEEeqnarrayboxHBOXSWfalse\else\@IEEEeqnarrayboxHBOXSWtrue\fi%
\@IEEEeqnarraybox}
\def\endIEEEeqnarraybox{\end@IEEEeqnarraybox}

\@namedef{IEEEeqnarraybox*}{\@IEEEeqnarraystarformtrue\ifmmode\@IEEEeqnarrayboxHBOXSWfalse\else\@IEEEeqnarrayboxHBOXSWtrue\fi%
\@IEEEeqnarraybox}
\@namedef{endIEEEeqnarraybox*}{\end@IEEEeqnarraybox}

% flag to indicate if the \IEEEeqnarraybox needs to put things into an hbox{$ $} 
% for \vcenter in non-math mode
\newif\if@IEEEeqnarrayboxHBOXSW%
\@IEEEeqnarrayboxHBOXSWfalse

\def\@IEEEeqnarraybox{\relax\@ifnextchar[{\@@IEEEeqnarraybox}{\@@IEEEeqnarraybox[\relax]}}
\def\@@IEEEeqnarraybox[#1]{\relax\@ifnextchar[{\@@@IEEEeqnarraybox[#1]}{\@@@IEEEeqnarraybox[#1][b]}}
\def\@@@IEEEeqnarraybox[#1][#2]{\relax\@ifnextchar[{\@@@@IEEEeqnarraybox[#1][#2]}{\@@@@IEEEeqnarraybox[#1][#2][\relax]}}

% #1 = decl; #2 = t,b,c; #3 = width, #4 = col specs
\def\@@@@IEEEeqnarraybox[#1][#2][#3]#4{\@IEEEeqnarrayISinnerfalse % not yet within the lines of the halign
   \@IEEEeqnarraymasterstrutsave% save current master strut values
   \@IEEEeqnarraystrutsize{0pt}{0pt}[\relax]% turn off struts by default
   \@IEEEeqnarrayusemasterstruttrue% use master strut till user asks otherwise
   \IEEEvisiblestrutsfalse% diagnostic mode defaults to off
   % no extra space unless the user specifically requests it
   \lineskip=0pt\relax%
   \lineskiplimit=0pt\relax%
   \baselineskip=\normalbaselineskip\relax%
   \jot=\normaljot\relax%
   \mathsurround\z@\relax% no extra spacing around math
   % the default end glues are zero for an \IEEEeqnarraybox
   \edef\@IEEEeqnarraycolSEPdefaultstart{\@IEEEeqnarraycolSEPzero}% default start glue
   \edef\@IEEEeqnarraycolSEPdefaultend{\@IEEEeqnarraycolSEPzero}% default end glue
   \edef\@IEEEeqnarraycolSEPdefaultmid{\@IEEEeqnarraycolSEPzero}% default inter-column glue
   \@advanceIEEEeqncolcntfalse% do not advance the col counter for each col the user uses, 
                              % used in \IEEEeqnarraymulticol and in the preamble build
   \IEEEeqnarrayboxdecl\relax% allow a way for the user to make global overrides
   #1\relax% allow user to override defaults
   \let\\\@IEEEeqnarrayboxcr% replace newline with one that allows optional spacing
   \@IEEEbuildpreamble #4\end\relax% build the preamble and put it into \@IEEEtrantmptoksA
   % add an isolation column to the preamble to stop \\'s {} from getting into the last col
   \ifnum\@IEEEeqnnumcols>0\relax\@IEEEappendtoksA{&}\fi% col separator for those after the first
   \toks0={##}%
   % add the isolation column to the preamble
   \@IEEEappendtoksA{\tabskip\z@skip\bgroup\the\toks0\egroup}% 
   % set the starting tabskip glue as determined by the preamble build
   \tabskip=\@IEEEBPstartglue\relax
   % begin the alignment
   \everycr{}%
   % use only the very first token to determine the positioning
   % this stops some problems when the user uses more than one letter,
   % but is probably not worth the effort
   % \noindent is used as a delimiter
   \def\@IEEEgrabfirstoken##1##2\noindent{\let\@IEEEgrabbedfirstoken=##1}%
   \@IEEEgrabfirstoken#2\relax\relax\noindent
   % \@IEEEgrabbedfirstoken has the first token, the rest are discarded
   % if we need to put things into and hbox and go into math mode, do so now
   \if@IEEEeqnarrayboxHBOXSW \leavevmode \hbox \bgroup $\fi%
   % use the appropriate vbox type
   \if\@IEEEgrabbedfirstoken t\relax\vtop\else\if\@IEEEgrabbedfirstoken c\relax%
   \vcenter\else\vbox\fi\fi\bgroup%
   \@IEEEeqnarrayISinnertrue% commands are now within the lines
   \ifx#3\relax\halign\else\halign to #3\relax\fi%
   \bgroup
   % "exspand" the preamble
   \span\the\@IEEEtrantmptoksA\cr}

% carry strut status and enter the isolation/strut column, 
% exit from math mode if needed, and exit
\def\end@IEEEeqnarraybox{\@IEEEeqnarrayglobalizestrutstatus% carry strut status
&% enter isolation/strut column
\@IEEEeqnarrayinsertstrut% do strut if needed
\@IEEEeqnarraymasterstrutrestore% restore the previous master strut values
% reset the strut system for next IEEEeqnarray
% (sets local strut values back to previous master strut values)
\@IEEEeqnarraystrutreset%
% ensure last line, exit from halign, close vbox
\crcr\egroup\egroup%
% exit from math mode and close hbox if needed
\if@IEEEeqnarrayboxHBOXSW $\egroup\fi}



% IEEEeqnarraybox uses a modifed \\ instead of the plain \cr to
% end rows. This allows for things like \\[vskip amount]
% This "cr" macros are modified versions those for LaTeX2e's eqnarray
% For IEEEeqnarraybox, \\* is the same as \\
% the {\ifnum0=`} braces must be kept away from the last column to avoid
% altering spacing of its math, so we use & to advance to the isolation/strut column
% carry strut status into isolation/strut column
\def\@IEEEeqnarrayboxcr{\@IEEEeqnarrayglobalizestrutstatus% carry strut status
&% enter isolation/strut column
\@IEEEeqnarrayinsertstrut% do strut if needed
% reset the strut system for next line or IEEEeqnarray
\@IEEEeqnarraystrutreset%
{\ifnum0=`}\fi%
\@ifstar{\@IEEEeqnarrayboxYCR}{\@IEEEeqnarrayboxYCR}}

% test and setup the optional argument to \\[]
\def\@IEEEeqnarrayboxYCR{\@testopt\@IEEEeqnarrayboxXCR\z@skip}

% IEEEeqnarraybox does not automatically increase line spacing by \jot
\def\@IEEEeqnarrayboxXCR[#1]{\ifnum0=`{\fi}%
\cr\noalign{\if@IEEEeqnarraystarform\else\vskip\jot\fi\vskip#1\relax}}



% starts the halign preamble build
\def\@IEEEbuildpreamble{\@IEEEtrantmptoksA={}% clear token register
\let\@IEEEBPcurtype=u%current column type is not yet known
\let\@IEEEBPprevtype=s%the previous column type was the start
\let\@IEEEBPnexttype=u%next column type is not yet known
% ensure these are valid
\def\@IEEEBPcurglue={0pt plus 0pt minus 0pt}%
\def\@IEEEBPcurcolname{@IEEEdefault}% name of current column definition
% currently acquired numerically referenced glue
% use a name that is easier to remember
\let\@IEEEBPcurnum=\@IEEEtrantmpcountA%
\@IEEEBPcurnum=0%
% tracks number of columns in the preamble
\@IEEEeqnnumcols=0%
% record the default end glues
\edef\@IEEEBPstartglue{\@IEEEeqnarraycolSEPdefaultstart}%
\edef\@IEEEBPendglue{\@IEEEeqnarraycolSEPdefaultend}%
% now parse the user's column specifications
\@@IEEEbuildpreamble}


% parses and builds the halign preamble
\def\@@IEEEbuildpreamble#1#2{\let\@@nextIEEEbuildpreamble=\@@IEEEbuildpreamble%
% use only the very first token to check the end
% \noindent is used as a delimiter as \end can be present here
\def\@IEEEgrabfirstoken##1##2\noindent{\let\@IEEEgrabbedfirstoken=##1}%
\@IEEEgrabfirstoken#1\relax\relax\noindent
\ifx\@IEEEgrabbedfirstoken\end\let\@@nextIEEEbuildpreamble=\@@IEEEfinishpreamble\else%
% identify current and next token type
\@IEEEgetcoltype{#1}{\@IEEEBPcurtype}{1}% current, error on invalid
\@IEEEgetcoltype{#2}{\@IEEEBPnexttype}{0}% next, no error on invalid next
% if curtype is a glue, get the glue def
\if\@IEEEBPcurtype g\@IEEEgetcurglue{#1}{\@IEEEBPcurglue}\fi%
% if curtype is a column, get the column def and set the current column name
\if\@IEEEBPcurtype c\@IEEEgetcurcol{#1}\fi%
% if curtype is a numeral, acquire the user defined glue
\if\@IEEEBPcurtype n\@IEEEprocessNcol{#1}\fi%
% process the acquired glue 
\if\@IEEEBPcurtype g\@IEEEprocessGcol\fi%
% process the acquired col 
\if\@IEEEBPcurtype c\@IEEEprocessCcol\fi%
% ready prevtype for next col spec.
\let\@IEEEBPprevtype=\@IEEEBPcurtype%
% be sure and put back the future token(s) as a group
\fi\@@nextIEEEbuildpreamble{#2}}


% executed just after preamble build is completed
% warn about zero cols, and if prevtype type = u, put in end tabskip glue
\def\@@IEEEfinishpreamble#1{\ifnum\@IEEEeqnnumcols<1\relax
\@IEEEclspkgerror{No column specifiers declared for IEEEeqnarray}%
{At least one column type must be declared for each IEEEeqnarray.}%
\fi%num cols less than 1
%if last type undefined, set default end tabskip glue
\if\@IEEEBPprevtype u\@IEEEappendtoksA{\tabskip=\@IEEEBPendglue}\fi}


% Identify and return the column specifier's type code
\def\@IEEEgetcoltype#1#2#3{%
% use only the very first token to determine the type
% \noindent is used as a delimiter as \end can be present here
\def\@IEEEgrabfirstoken##1##2\noindent{\let\@IEEEgrabbedfirstoken=##1}%
\@IEEEgrabfirstoken#1\relax\relax\noindent
% \@IEEEgrabfirstoken has the first token, the rest are discarded
% n = number
% g = glue (any other char in catagory 12)
% c = letter
% e = \end
% u = undefined
% third argument: 0 = no error message, 1 = error on invalid char
\let#2=u\relax% assume invalid until know otherwise
\ifx\@IEEEgrabbedfirstoken\end\let#2=e\else
\ifcat\@IEEEgrabbedfirstoken\relax\else% screen out control sequences
\if0\@IEEEgrabbedfirstoken\let#2=n\else
\if1\@IEEEgrabbedfirstoken\let#2=n\else
\if2\@IEEEgrabbedfirstoken\let#2=n\else
\if3\@IEEEgrabbedfirstoken\let#2=n\else
\if4\@IEEEgrabbedfirstoken\let#2=n\else
\if5\@IEEEgrabbedfirstoken\let#2=n\else
\if6\@IEEEgrabbedfirstoken\let#2=n\else
\if7\@IEEEgrabbedfirstoken\let#2=n\else
\if8\@IEEEgrabbedfirstoken\let#2=n\else
\if9\@IEEEgrabbedfirstoken\let#2=n\else
\ifcat,\@IEEEgrabbedfirstoken\let#2=g\relax
\else\ifcat a\@IEEEgrabbedfirstoken\let#2=c\relax\fi\fi\fi\fi\fi\fi\fi\fi\fi\fi\fi\fi\fi\fi
\if#2u\relax
\if0\noexpand#3\relax\else\@IEEEclspkgerror{Invalid character in column specifications}%
{Only letters, numerals and certain other symbols are allowed \MessageBreak
as IEEEeqnarray column specifiers.}\fi\fi}


% identify the current letter referenced column
% if invalid, use a default column
\def\@IEEEgetcurcol#1{\expandafter\ifx\csname @IEEEeqnarraycolDEF#1\endcsname\@IEEEeqnarraycolisdefined%
\def\@IEEEBPcurcolname{#1}\else% invalid column name
\@IEEEclspkgerror{Invalid column type "#1" in column specifications.\MessageBreak
Using a default centering column instead}%
{You must define IEEEeqnarray column types before use.}%
\def\@IEEEBPcurcolname{@IEEEdefault}\fi}


% identify and return the predefined (punctuation) glue value
\def\@IEEEgetcurglue#1#2{%
% ! = \! (neg small)  -0.16667em (-3/18 em)
% , = \, (small)       0.16667em ( 3/18 em)
% : = \: (med)         0.22222em ( 4/18 em)
% ; = \; (large)       0.27778em ( 5/18 em)
% ' = \quad            1em
% " = \qquad           2em
% . = 0.5\arraycolsep
% / = \arraycolsep
% ? = 2\arraycolsep
% * = 1fil
% + = \@IEEEeqnarraycolSEPcenter
% - = \@IEEEeqnarraycolSEPzero
% Note that all em values are referenced to the math font (textfont2) fontdimen6
% value for 1em.
% 
% use only the very first token to determine the type
% this prevents errant tokens from getting in the main text
% \noindent is used as a delimiter here
\def\@IEEEgrabfirstoken##1##2\noindent{\let\@IEEEgrabbedfirstoken=##1}%
\@IEEEgrabfirstoken#1\relax\relax\noindent
% get the math font 1em value
% LaTeX2e's NFSS2 does not preload the fonts, but \IEEEeqnarray needs
% to gain access to the math (\textfont2) font's spacing parameters.
% So we create a bogus box here that uses the math font to ensure
% that \textfont2 is loaded and ready. If this is not done,
% the \textfont2 stuff here may not work.
% Thanks to Bernd Raichle for his 1997 post on this topic.
{\setbox0=\hbox{$\displaystyle\relax$}}%
% fontdimen6 has the width of 1em (a quad).
\@IEEEtrantmpdimenA=\fontdimen6\textfont2\relax%
% identify the glue value based on the first token
% we discard anything after the first
\if!\@IEEEgrabbedfirstoken\@IEEEtrantmpdimenA=-0.16667\@IEEEtrantmpdimenA\edef#2{\the\@IEEEtrantmpdimenA}\else
\if,\@IEEEgrabbedfirstoken\@IEEEtrantmpdimenA=0.16667\@IEEEtrantmpdimenA\edef#2{\the\@IEEEtrantmpdimenA}\else
\if:\@IEEEgrabbedfirstoken\@IEEEtrantmpdimenA=0.22222\@IEEEtrantmpdimenA\edef#2{\the\@IEEEtrantmpdimenA}\else
\if;\@IEEEgrabbedfirstoken\@IEEEtrantmpdimenA=0.27778\@IEEEtrantmpdimenA\edef#2{\the\@IEEEtrantmpdimenA}\else
\if'\@IEEEgrabbedfirstoken\@IEEEtrantmpdimenA=1\@IEEEtrantmpdimenA\edef#2{\the\@IEEEtrantmpdimenA}\else
\if"\@IEEEgrabbedfirstoken\@IEEEtrantmpdimenA=2\@IEEEtrantmpdimenA\edef#2{\the\@IEEEtrantmpdimenA}\else
\if.\@IEEEgrabbedfirstoken\@IEEEtrantmpdimenA=0.5\arraycolsep\edef#2{\the\@IEEEtrantmpdimenA}\else
\if/\@IEEEgrabbedfirstoken\edef#2{\the\arraycolsep}\else
\if?\@IEEEgrabbedfirstoken\@IEEEtrantmpdimenA=2\arraycolsep\edef#2{\the\@IEEEtrantmpdimenA}\else
\if *\@IEEEgrabbedfirstoken\edef#2{0pt plus 1fil minus 0pt}\else
\if+\@IEEEgrabbedfirstoken\edef#2{\@IEEEeqnarraycolSEPcenter}\else
\if-\@IEEEgrabbedfirstoken\edef#2{\@IEEEeqnarraycolSEPzero}\else
\edef#2{\@IEEEeqnarraycolSEPzero}%
\@IEEEclspkgerror{Invalid predefined inter-column glue type "#1" in\MessageBreak
column specifications. Using a default value of\MessageBreak
0pt instead}%
{Only !,:;'"./?*+ and - are valid predefined glue types in the\MessageBreak 
IEEEeqnarray column specifications.}\fi\fi\fi\fi\fi\fi\fi\fi\fi\fi\fi\fi}



% process a numerical digit from the column specification
% and look up the corresponding user defined glue value
% can transform current type from n to g or a as the user defined glue is acquired
\def\@IEEEprocessNcol#1{\if\@IEEEBPprevtype g%
\@IEEEclspkgerror{Back-to-back inter-column glue specifiers in column\MessageBreak
specifications. Ignoring consecutive glue specifiers\MessageBreak
after the first}%
{You cannot have two or more glue types next to each other\MessageBreak 
in the IEEEeqnarray column specifications.}%
\let\@IEEEBPcurtype=a% abort this glue, future digits will be discarded
\@IEEEBPcurnum=0\relax%
\else% if we previously aborted a glue
\if\@IEEEBPprevtype a\@IEEEBPcurnum=0\let\@IEEEBPcurtype=a%maintain digit abortion
\else%acquire this number
% save the previous type before the numerical digits started
\if\@IEEEBPprevtype n\else\let\@IEEEBPprevsavedtype=\@IEEEBPprevtype\fi%
\multiply\@IEEEBPcurnum by 10\relax%
\advance\@IEEEBPcurnum by #1\relax% add in number, \relax is needed to stop TeX's number scan
\if\@IEEEBPnexttype n\else%close acquisition
\expandafter\ifx\csname @IEEEeqnarraycolSEPDEF\expandafter\romannumeral\number\@IEEEBPcurnum\endcsname\@IEEEeqnarraycolisdefined%
\edef\@IEEEBPcurglue{\csname @IEEEeqnarraycolSEP\expandafter\romannumeral\number\@IEEEBPcurnum\endcsname}%
\else%user glue not defined
\@IEEEclspkgerror{Invalid user defined inter-column glue type "\number\@IEEEBPcurnum" in\MessageBreak
column specifications. Using a default value of\MessageBreak
0pt instead}%
{You must define all IEEEeqnarray numerical inter-column glue types via\MessageBreak
\string\IEEEeqnarraydefcolsep \space before they are used in column specifications.}%
\edef\@IEEEBPcurglue{\@IEEEeqnarraycolSEPzero}%
\fi% glue defined or not
\let\@IEEEBPcurtype=g% change the type to reflect the acquired glue
\let\@IEEEBPprevtype=\@IEEEBPprevsavedtype% restore the prev type before this number glue
\@IEEEBPcurnum=0\relax%ready for next acquisition
\fi%close acquisition, get glue
\fi%discard or acquire number
\fi%prevtype glue or not
}


% process an acquired glue
% add any acquired column/glue pair to the preamble
\def\@IEEEprocessGcol{\if\@IEEEBPprevtype a\let\@IEEEBPcurtype=a%maintain previous glue abortions
\else
% if this is the start glue, save it, but do nothing else 
% as this is not used in the preamble, but before
\if\@IEEEBPprevtype s\edef\@IEEEBPstartglue{\@IEEEBPcurglue}%
\else%not the start glue
\if\@IEEEBPprevtype g%ignore if back to back glues
\@IEEEclspkgerror{Back-to-back inter-column glue specifiers in column\MessageBreak
specifications. Ignoring consecutive glue specifiers\MessageBreak
after the first}%
{You cannot have two or more glue types next to each other\MessageBreak 
in the IEEEeqnarray column specifications.}%
\let\@IEEEBPcurtype=a% abort this glue
\else% not a back to back glue
\if\@IEEEBPprevtype c\relax% if the previoustype was a col, add column/glue pair to preamble
\ifnum\@IEEEeqnnumcols>0\relax\@IEEEappendtoksA{&}\fi
\toks0={##}%
% make preamble advance col counter if this environment needs this
\if@advanceIEEEeqncolcnt\@IEEEappendtoksA{\global\advance\@IEEEeqncolcnt by 1\relax}\fi
% insert the column defintion into the preamble, being careful not to expand
% the column definition
\@IEEEappendtoksA{\tabskip=\@IEEEBPcurglue}%
\@IEEEappendNOEXPANDtoksA{\begingroup\csname @IEEEeqnarraycolPRE}%
\@IEEEappendtoksA{\@IEEEBPcurcolname}%
\@IEEEappendNOEXPANDtoksA{\endcsname}%
\@IEEEappendtoksA{\the\toks0}%
\@IEEEappendNOEXPANDtoksA{\relax\relax\relax\relax\relax%
\relax\relax\relax\relax\relax\csname @IEEEeqnarraycolPOST}%
\@IEEEappendtoksA{\@IEEEBPcurcolname}%
\@IEEEappendNOEXPANDtoksA{\endcsname\relax\relax\relax\relax\relax%
\relax\relax\relax\relax\relax\endgroup}%
\advance\@IEEEeqnnumcols by 1\relax%one more column in the preamble
\else% error: non-start glue with no pending column
\@IEEEclspkgerror{Inter-column glue specifier without a prior column\MessageBreak
type in the column specifications. Ignoring this glue\MessageBreak 
specifier}%
{Except for the first and last positions, glue can be placed only\MessageBreak
between column types.}%
\let\@IEEEBPcurtype=a% abort this glue
\fi% previous was a column
\fi% back-to-back glues
\fi% is start column glue
\fi% prev type not a
}


% process an acquired letter referenced column and, if necessary, add it to the preamble
\def\@IEEEprocessCcol{\if\@IEEEBPnexttype g\else
\if\@IEEEBPnexttype n\else
% we have a column followed by something other than a glue (or numeral glue)
% so we must add this column to the preamble now
\ifnum\@IEEEeqnnumcols>0\relax\@IEEEappendtoksA{&}\fi%col separator for those after the first
\if\@IEEEBPnexttype e\@IEEEappendtoksA{\tabskip=\@IEEEBPendglue\relax}\else%put in end glue
\@IEEEappendtoksA{\tabskip=\@IEEEeqnarraycolSEPdefaultmid\relax}\fi% or default mid glue
\toks0={##}%
% make preamble advance col counter if this environment needs this
\if@advanceIEEEeqncolcnt\@IEEEappendtoksA{\global\advance\@IEEEeqncolcnt by 1\relax}\fi
% insert the column definition into the preamble, being careful not to expand
% the column definition
\@IEEEappendNOEXPANDtoksA{\begingroup\csname @IEEEeqnarraycolPRE}%
\@IEEEappendtoksA{\@IEEEBPcurcolname}%
\@IEEEappendNOEXPANDtoksA{\endcsname}%
\@IEEEappendtoksA{\the\toks0}%
\@IEEEappendNOEXPANDtoksA{\relax\relax\relax\relax\relax%
\relax\relax\relax\relax\relax\csname @IEEEeqnarraycolPOST}%
\@IEEEappendtoksA{\@IEEEBPcurcolname}%
\@IEEEappendNOEXPANDtoksA{\endcsname\relax\relax\relax\relax\relax%
\relax\relax\relax\relax\relax\endgroup}%
\advance\@IEEEeqnnumcols by 1\relax%one more column in the preamble
\fi%next type not numeral
\fi%next type not glue
}


%%
%% END OF IEEEeqnarry DEFINITIONS
%%




% set up the running headings, this complex because of all the different
% modes IEEEtran supports
\if@twoside
 \if@technote
   \def\ps@headings{%
       \def\@oddhead{\hbox{}\scriptsize\leftmark \hfil \thepage}
       \def\@evenhead{\scriptsize\thepage \hfil \leftmark\hbox{}}
       \if@draftclsmode
            \if@draftclsmodefoot
               \def\@oddfoot{\scriptsize\@date\hfil DRAFT}
               \def\@evenfoot{\scriptsize DRAFT\hfil\@date}
            \else
               \def\@oddfoot{}\def\@evenfoot{}%
            \fi
       \else
            \def\@oddfoot{}\def\@evenfoot{}
       \fi}
 \else % not a technote
   \def\ps@headings{%
       \if@confmode
        \def\@oddhead{}
        \def\@evenhead{}
       \else
        \def\@oddhead{\hbox{}\scriptsize\rightmark \hfil \thepage}
        \def\@evenhead{\scriptsize\thepage \hfil \leftmark\hbox{}}
       \fi
       \if@draftclsmode
            \def\@oddhead{\hbox{}\scriptsize\rightmark \hfil \thepage}
            \def\@evenhead{\scriptsize\thepage \hfil \leftmark\hbox{}}
            \if@draftclsmodefoot
               \def\@oddfoot{\scriptsize\@date\hfil DRAFT}
               \def\@evenfoot{\scriptsize DRAFT\hfil\@date}
            \else
               \def\@oddfoot{}\def\@evenfoot{}%
            \fi
       \else
            \def\@oddfoot{}\def\@evenfoot{}%
       \fi}
 \fi
\else % single side
\def\ps@headings{%
    \if@confmode
     \def\@oddhead{}
     \def\@evenhead{}
    \else
     \def\@oddhead{\hbox{}\scriptsize\leftmark \hfil \thepage}
     \def\@evenhead{}
    \fi
    \if@draftclsmode
          \def\@oddhead{\hbox{}\scriptsize\leftmark \hfil \thepage}
          \def\@evenhead{}
          \if@draftclsmodefoot
             \def\@oddfoot{\scriptsize \@date \hfil DRAFT}
          \else
             \def\@oddfoot{}
          \fi
    \else
         \def\@oddfoot{}
    \fi
    \def\@evenfoot{}}
\fi


% title page style
\def\ps@titlepagestyle{\def\@oddfoot{}\def\@evenfoot{}%
\if@confmode
   \def\@oddhead{}%
   \def\@evenhead{}%
\else
   \def\@oddhead{\hbox{}\scriptsize\leftmark \hfil \thepage}%
   \def\@evenhead{\scriptsize\thepage \hfil \leftmark\hbox{}}%
\fi
\if@draftclsmode
   \def\@oddhead{\hbox{}\scriptsize\leftmark \hfil \thepage}%
   \def\@evenhead{\scriptsize\thepage \hfil \leftmark\hbox{}}%
   \if@draftclsmodefoot
      \def\@oddfoot{\scriptsize \@date\hfil DRAFT}%
      \def\@evenfoot{\scriptsize DRAFT\hfil \@date}%
   \fi
\else
   % all non-draft mode footers
   \if@IEEEusingpubid
      % for title pages that are using a pubid
      % do not repeat pubid if using peer review option
      \if@peerreviewoption
      \else
         \footskip 0pt%
         \def\@oddfoot{\hss\normalfont\footnotesize\raisebox{1.5ex}[1.5ex]{\@pubid}\hss}%
         \def\@evenfoot{\hss\normalfont\footnotesize\raisebox{1.5ex}[1.5ex]{\@pubid}\hss}%
      \fi
   \fi
\fi}


% peer review cover page style
\def\ps@peerreviewcoverpagestyle{%
\def\@oddhead{}\def\@evenhead{}%
\def\@oddfoot{}\def\@evenfoot{}%
\if@draftclsmode
   \if@draftclsmodefoot
      \def\@oddfoot{\scriptsize \@date\hfil DRAFT}%
      \def\@evenfoot{\scriptsize DRAFT\hfil \@date}%
   \fi
\else
   % non-draft mode footers
   \if@IEEEusingpubid
      \footskip 0pt%
      \def\@oddfoot{\hss\normalfont\footnotesize\raisebox{1.5ex}[1.5ex]{\@pubid}\hss}%
      \def\@evenfoot{\hss\normalfont\footnotesize\raisebox{1.5ex}[1.5ex]{\@pubid}\hss}%
   \fi
\fi}


% start with empty headings
\def\rightmark{}\def\leftmark{}


%% Defines the command for putting the header. \footernote{TEXT} is the same
%% as \markboth{TEXT}{TEXT}. 
%% Note that all the text is forced into uppercase, if you have some text
%% that needs to be in lower case, for instance et. al., then either manually
%% set \leftmark and \rightmark or use \MakeLowercase{et. al.} within the
%% arguments to \markboth.
\def\markboth#1#2{\def\leftmark{\MakeUppercase{#1}}\def\rightmark{\MakeUppercase{#2}}}
\def\footernote#1{\markboth{#1}{#1}}

\def\today{\ifcase\month\or
    January\or February\or March\or April\or May\or June\or
    July\or August\or September\or October\or November\or December\fi
    \space\number\day, \number\year}




%% CITATION AND BIBLIOGRAPHY COMMANDS
%% 
%% V1.6 no longer supports the older, nonstandard \shortcite and \citename setup stuff
% 
% 
% Modify Latex2e \@citex to separate citations with "], ["
\def\@citex[#1]#2{%
  \let\@citea\@empty
  \@cite{\@for\@citeb:=#2\do
    {\@citea\def\@citea{], [}%
     \edef\@citeb{\expandafter\@firstofone\@citeb\@empty}%
     \if@filesw\immediate\write\@auxout{\string\citation{\@citeb}}\fi
     \@ifundefined{b@\@citeb}{\mbox{\reset@font\bfseries ?}%
       \G@refundefinedtrue
       \@latex@warning
         {Citation `\@citeb' on page \thepage \space undefined}}%
       {\hbox{\csname b@\@citeb\endcsname}}}}{#1}}

% V1.6 we create hooks for the optional use of Donald Arseneau's
% cite.sty package. cite.sty is "smart" and will notice that the
% following format controls are already defined and will not
% redefine them. The result will be the proper sorting of the
% citation numbers and auto detection of 3 or more entry "ranges" -
% all in IEEE style:  [1], [2], [5]--[7], [12]
% This also allows for an optional note, i.e., \cite[mynote]{..}.
% If the \cite with note has more than one reference, the note will
% be applied to the last of the listed references. It is generally
% desired that if a note is given, only one reference is listed in
% that \cite.
% Thanks to Mr. Arseneau for providing the required format arguments
% to produce the IEEE style.
\def\citepunct{], [}
\def\citedash{]--[}


% V1.6b providing this command makes hyperref think the natbib package is
% in use so that it will not interfere with cite.sty. However, as a result,
% citation numbers will not be hyperlinked.
\def\NAT@parse{\typeout{IEEEtran error: Attempt to use fake Natbib command 
which is provided to fool Hyperref.}}
% it easy enough to override via:
% \let\NAT@parse\undefined


% V1.6 class files should always provide these
\def\newblock{\hskip .11em\@plus.33em\@minus.07em}
\let\@openbib@code\@empty



% Provide support for the control entries of IEEEtran.bst V1.00 and later.
\def\bstctlcite#1{\@bsphack
  \@for\@citeb:=#1\do{%
    \edef\@citeb{\expandafter\@firstofone\@citeb}%
    \if@filesw\immediate\write\@auxout{\string\citation{\@citeb}}\fi}%
  \@esphack}

% V1.6 provide a way for a user to execute a command just before 
% a given reference number - used to insert a \newpage to balance
% the columns on the last page
\edef\@IEEEtriggerrefnum{0}   % the default of zero means that
                              % the command is not executed
\def\@IEEEtriggercmd{\newpage}

% allow the user to alter the triggered command
\long\def\IEEEtriggercmd#1{\long\def\@IEEEtriggercmd{#1}}

% allow user a way to specify the reference number just before the
% command is executed
\def\IEEEtriggeratref#1{\@IEEEtrantmpcountA=#1%
\edef\@IEEEtriggerrefnum{\the\@IEEEtrantmpcountA}}%

% trigger command at the given reference
\def\@IEEEbibitemprefix{\@IEEEtrantmpcountA=\@IEEEtriggerrefnum\relax%
\advance\@IEEEtrantmpcountA by -1\relax%
\ifnum\c@enumiv=\@IEEEtrantmpcountA\relax\@IEEEtriggercmd\relax\fi}

\def\@biblabel#1{[#1]}

\def\thebibliography#1{\section*{References}%
    \addcontentsline{toc}{section}{References}%
    % V1.6 add some rubber space here and provide a command trigger
    \footnotesize \vskip 0.3\baselineskip plus 0.1\baselineskip minus 0.1\baselineskip%
    \list{\@biblabel{\@arabic\c@enumiv}}%
    {\settowidth\labelwidth{\@biblabel{#1}}%
    \leftmargin\labelwidth
    \advance\leftmargin\labelsep\relax
    \itemsep 0pt plus .5pt\relax%
    \usecounter{enumiv}%
    \let\p@enumiv\@empty
    \renewcommand\theenumiv{\@arabic\c@enumiv}}%
    \let\@IEEElatexbibitem\bibitem%
    \def\bibitem{\@IEEEbibitemprefix\@IEEElatexbibitem}%
\def\newblock{\hskip .11em plus .33em minus .07em}%
% originally:
%   \sloppy\clubpenalty4000\widowpenalty4000%
% by adding the \interlinepenalty here, we make it more
% difficult, but not impossible, for LaTeX to break within a reference.
% IEEE almost never breaks a reference (but they do it more often with
% technotes). You may get an underfull vbox warning around the bibliography, 
% but the final result will be much more like what IEEE will publish. 
% MDS 11/2000
\if@technote\sloppy\clubpenalty4000\widowpenalty4000\interlinepenalty100%
\else\sloppy\clubpenalty4000\widowpenalty4000\interlinepenalty500\fi%
    \sfcode`\.=1000\relax}
\let\endthebibliography=\endlist




% TITLE PAGE COMMANDS
% 
% 
% \IEEEmembership is used to produce the sublargesize italic font used to indicate author 
% IEEE membership.
\def\IEEEmembership#1{{\sublargesize\normalfont\textit{#1}}}
 

% \authorrefmark{} produces a footnote type symbol to indicate author affiliation.
% When given an argument of 1 to 9, \authorrefmark{} follows the standard LaTeX footnote
% symbol sequence convention. However, for arguments 10 and above, \authorrefmark{} 
% reverts to using lower case roman numerals, so it cannot overflow. Do note that you 
% cannot use \footnotemark[] in place of \authorrefmark{} within \author as the footnote
% symbols will have been turned off to prevent \thanks from creating footnote marks.
% \authorrefmark{} produces a symbol that appears to LaTeX as having zero vertical
% height - this allows for a more compact line packing, but the user must ensure that
% the interline spacing is large enough to prevent \authorrefmark{} from colliding
% with the text above.
\def\authorrefmark#1{\raisebox{0pt}[0pt][0pt]{\textsuperscript{\footnotesize\ensuremath{\ifcase#1\or *\or \dagger\or \ddagger\or%
    \mathsection\or \mathparagraph\or \|\or **\or \dagger\dagger%
    \or \ddagger\ddagger \else\textsuperscript{\expandafter\romannumeral#1}\fi}}}}


% FONT CONTROLS AND SPACINGS FOR CONFERENCE MODE AUTHOR NAME AND AFFILIATION BLOCKS
% 
% The default font styles for the author name and affiliation blocks (confmode)
\def\@IEEEauthorblockNstyle{\normalfont\sublargesize}
\def\@IEEEauthorblockAstyle{\normalfont\normalsize}
% The default if the user does not use an author block
\def\@IEEEauthordefaulttextstyle{\normalfont\sublargesize}

% spacing from title (or special paper notice) to author name blocks (confmode)
% can be negative
\def\@IEEEauthorblockconfadjspace{-0.25em}

% spacing between name and affiliation blocks (confmode)
% This can be negative.
% IEEE doesn't want any added spacing here, but I will leave these
% controls in place in case they ever change their mind.
% Personally, I like 0.75ex.
%\def\@IEEEauthorblockNtopspace{0.75ex}
%\def\@IEEEauthorblockAtopspace{0.75ex}
\def\@IEEEauthorblockNtopspace{0.0ex}
\def\@IEEEauthorblockAtopspace{0.0ex}
% baseline spacing within name and affiliation blocks (confmode)
% must be positive, spacings below certain values will make 
% the position of line of text sensitive to the contents of the
% line above it i.e., whether or not the prior line has descenders, 
% subscripts, etc. For this reason it is a good idea to keep
% these above 2.6ex
\def\@IEEEauthorblockNinterlinespace{2.6ex}
\def\@IEEEauthorblockAinterlinespace{2.75ex}

% This tracks the required strut size.
% See the \@IEEEauthorhalign command for the actual default value used.
\def\@IEEEauthorblockXinterlinespace{2.7ex}

% variables to retain font size and style across groups
% values given here have no effect as they will be overwritten later
\gdef\@IEEESAVESTATEfontsize{10}
\gdef\@IEEESAVESTATEfontbaselineskip{12}
\gdef\@IEEESAVESTATEfontencoding{OT1}
\gdef\@IEEESAVESTATEfontfamily{ptm}
\gdef\@IEEESAVESTATEfontseries{m}
\gdef\@IEEESAVESTATEfontshape{n}

% saves the current font attributes
\def\@IEEEcurfontSAVE{\global\let\@IEEESAVESTATEfontsize\f@size%
\global\let\@IEEESAVESTATEfontbaselineskip\f@baselineskip%
\global\let\@IEEESAVESTATEfontencoding\f@encoding%
\global\let\@IEEESAVESTATEfontfamily\f@family%
\global\let\@IEEESAVESTATEfontseries\f@series%
\global\let\@IEEESAVESTATEfontshape\f@shape}

% restores the saved font attributes
\def\@IEEEcurfontRESTORE{\fontsize{\@IEEESAVESTATEfontsize}{\@IEEESAVESTATEfontbaselineskip}%
\fontencoding{\@IEEESAVESTATEfontencoding}%
\fontfamily{\@IEEESAVESTATEfontfamily}%
\fontseries{\@IEEESAVESTATEfontseries}%
\fontshape{\@IEEESAVESTATEfontshape}%
\selectfont}


% variable to indicate if the current block is the first block in the column
\newif\if@IEEEprevauthorblockincol   \@IEEEprevauthorblockincolfalse


% the command places a strut with height and depth = \@IEEEauthorblockXinterlinespace
% we use this technique to have complete manual control over the spacing of the lines
% within the halign environment.
% We set the below baseline portion at 30%, the above
% baseline portion at 70% of the total length.
% Responds to changes in the document's \baselinestretch
\def\@IEEEauthorstrutrule{\@IEEEtrantmpdimenA\@IEEEauthorblockXinterlinespace%
\@IEEEtrantmpdimenA=\baselinestretch\@IEEEtrantmpdimenA%
\rule[-0.3\@IEEEtrantmpdimenA]{0pt}{\@IEEEtrantmpdimenA}}


% blocks to hold the authors' names and affilations. 
% Makes formatting easy for conferences
%
% use real definitions in conference mode
% name block
\def\authorblockN#1{\relax\@IEEEauthorblockNstyle% set the default text style
\gdef\@IEEEauthorblockXinterlinespace{0pt}% disable strut for spacer row
% the \expandafter hides the \cr in conditional tex, see the array.sty docs
% for details, probably not needed here as the \cr is in a macro
% do a spacer row if needed
\if@IEEEprevauthorblockincol\expandafter\@IEEEauthorblockNtopspaceline\fi
\global\@IEEEprevauthorblockincoltrue% we now have a block in this column
%restore the correct strut value
\gdef\@IEEEauthorblockXinterlinespace{\@IEEEauthorblockNinterlinespace}%
% input the author names
#1%
% end the row if the user did not already
\crcr}
% spacer row for names
\def\@IEEEauthorblockNtopspaceline{\cr\noalign{\vskip\@IEEEauthorblockNtopspace}}
%
% affiliation block
\def\authorblockA#1{\relax\@IEEEauthorblockAstyle% set the default text style
\gdef\@IEEEauthorblockXinterlinespace{0pt}%disable strut for spacer row
% the \expandafter hides the \cr in conditional tex, see the array.sty docs
% for details, probably not needed here as the \cr is in a macro
% do a spacer row if needed
\if@IEEEprevauthorblockincol\expandafter\@IEEEauthorblockAtopspaceline\fi
\global\@IEEEprevauthorblockincoltrue% we now have a block in this column
%restore the correct strut value
\gdef\@IEEEauthorblockXinterlinespace{\@IEEEauthorblockAinterlinespace}%
% input the author affiliations
#1%
% end the row if the user did not already
\crcr}
% spacer row for affiliations
\def\@IEEEauthorblockAtopspaceline{\cr\noalign{\vskip\@IEEEauthorblockAtopspace}}


% allow papers to compile even if author blocks are used in modes other
% than conference or peerreviewca. For such cases, we provide dummy blocks.
\if@confmode
\else
   \if@peerreviewcaoption\else
      % not conference or peerreviewca mode
      \def\authorblockN#1{#1}%
      \def\authorblockA#1{#1}%
   \fi
\fi



% we provide our own halign so as not to have to depend on tabular
\def\@IEEEauthorhalign{\@IEEEauthordefaulttextstyle% default text style
   \lineskip=0pt\relax% disable line spacing
   \lineskiplimit=0pt\relax%
   \baselineskip=0pt\relax%
   \@IEEEcurfontSAVE% save the current font
   \mathsurround\z@\relax% no extra spacing around math
   \let\\\@IEEEauthorhaligncr% replace newline with halign friendly one
   \tabskip=0pt\relax% no column spacing
   \everycr{}% ensure no problems here
   \@IEEEprevauthorblockincolfalse% no author blocks yet
   \def\@IEEEauthorblockXinterlinespace{2.7ex}% default interline space
   \vtop\bgroup%vtop box
   \halign\bgroup&\relax\hfil\@IEEEcurfontRESTORE\relax ##\relax
   \hfil\@IEEEcurfontSAVE\@IEEEauthorstrutrule\cr}

% ensure last line, exit from halign, close vbox
\def\end@IEEEauthorhalign{\crcr\egroup\egroup}

% handle bogus star form
\def\@IEEEauthorhaligncr{{\ifnum0=`}\fi\@ifstar{\@@IEEEauthorhaligncr}{\@@IEEEauthorhaligncr}}

% test and setup the optional argument to \\[]
\def\@@IEEEauthorhaligncr{\@testopt\@@@IEEEauthorhaligncr\z@skip}

% end the line and do the optional spacer
\def\@@@IEEEauthorhaligncr[#1]{\ifnum0=`{\fi}\cr\noalign{\vskip#1\relax}}



% flag to prevent multiple \and warning messages
\newif\if@IEEEWARNand
\@IEEEWARNandtrue

% if in conference or peerreviewca modes, we support the use of \and as \author is a
% tabular environment, otherwise we warn the user that \and is invalid
% outside of conference or peerreviewca modes.
\def\and{\relax} % provide a bogus \and that we will then override

\renewcommand{\and}[1][\relax]{\if@IEEEWARNand\typeout{** WARNING: \noexpand\and is valid only
                               when in conference or peerreviewca}\typeout{modes (line \the\inputlineno).}\fi\global\@IEEEWARNandfalse}

\if@confmode%
\renewcommand{\and}[1][\hfill]{\end{@IEEEauthorhalign}#1\begin{@IEEEauthorhalign}}%
\fi
\if@peerreviewcaoption
\renewcommand{\and}[1][\hfill]{\end{@IEEEauthorhalign}#1\begin{@IEEEauthorhalign}}%
\fi


% page clearing command
% based on LaTeX2e's \cleardoublepage, but allows different page styles
% for the inserted blank pages
\def\@IEEEcleardoublepage#1{\clearpage\if@twoside\ifodd\c@page\else
\hbox{}\thispagestyle{#1}\newpage\if@twocolumn\hbox{}\thispagestyle{#1}\newpage\fi\fi\fi}


% user command to invoke the title page
\def\maketitle{\par%
  \begingroup%
  \normalfont%
  \def\thefootnote{}%  the \thanks{} mark type is empty
  \def\footnotemark{}% and kill space from \thanks within author
  \footnotesize%       equal spacing between thanks lines
  \footnotesep 0.7\baselineskip%see global setting of \footnotesep for more info
  \normalsize%
  \if@peerreviewoption
     \newpage\global\@topnum\z@ \@maketitle\@IEEEstatictitlevskip\@IEEEaftertitletext%
     \thispagestyle{peerreviewcoverpagestyle}\@thanks%
  \else
     \if@twocolumn%
        \if@technote%
           \newpage\global\@topnum\z@ \@maketitle\@IEEEstatictitlevskip\@IEEEaftertitletext%
        \else
           \twocolumn[\@maketitle\@IEEEdynamictitlevspace\@IEEEaftertitletext]%
        \fi
     \else
        \newpage\global\@topnum\z@ \@maketitle\@IEEEstatictitlevskip\@IEEEaftertitletext%
     \fi
     \thispagestyle{titlepagestyle}\@thanks%
  \fi
  % pullup page for pubid if used.
  \if@IEEEusingpubid
     \enlargethispage{-\@pubidpullup}%
  \fi 
  \endgroup
  \setcounter{footnote}{0}\let\maketitle\relax\let\@maketitle\relax
  \gdef\@thanks{}
  % v1.6b do not clear these as we will need the title again for peer review papers
  % \gdef\@author{}\gdef\@title{}%
  \let\thanks\relax}


% formats the Title, authors names, affiliations and special paper notice
% THIS IS A CONTROLLED SPACING COMMAND! Do not allow blank lines or unintentional
% spaces to enter the definition - use % at the end of each line
\def\@maketitle{\newpage
\begin{center}%
\if@technote%
   {\bfseries\large\@title\par}\vskip 1.3em{\lineskip .5em\@author\@specialpapernotice\par}%
\else% not a technote
   \vskip0.2em{\Huge\@title\par}\vskip1.0em\par%
   % V1.6 handle \author differently if in conference mode
   \if@confmode%
      {\@specialpapernotice\mbox{}\vskip\@IEEEauthorblockconfadjspace%
      \mbox{}\hfill\begin{@IEEEauthorhalign}\@author\end{@IEEEauthorhalign}\hfill\mbox{}\par}%
   \else% peerreviewca, peerreview or journal
      \if@peerreviewcaoption
         % peerreviewca handles author names just like conference mode
         {\@specialpapernotice\mbox{}\vskip\@IEEEauthorblockconfadjspace%
         \mbox{}\hfill\begin{@IEEEauthorhalign}\@author\end{@IEEEauthorhalign}\hfill\mbox{}\par}%
      \else % journal or peerreview
         {\lineskip.5em\sublargesize\@author\@specialpapernotice\par}%
      \fi
   \fi
\fi\end{center}}



% V1.6b define the \IEEEpeerreviewmaketitle as needed
\if@peerreviewoption
\def\IEEEpeerreviewmaketitle{\@IEEEcleardoublepage{empty}%
\if@twocolumnmode
\twocolumn[\@IEEEpeerreviewmaketitle\@IEEEdynamictitlevspace]
\else
\newpage\@IEEEpeerreviewmaketitle\@IEEEstatictitlevskip
\fi
\thispagestyle{titlepagestyle}}
\else
% \IEEEpeerreviewmaketitle does nothing if peer review option has not been selected
\def\IEEEpeerreviewmaketitle{\relax}
\fi

% peerreview formats the repeated title like the title in journal papers.
\def\@IEEEpeerreviewmaketitle{\begin{center}%
\normalfont\normalsize\vskip0.2em{\Huge\@title\par}\vskip1.0em\par
\end{center}}



% V1.6 
% this is a static rubber spacer between the title/authors and the main text
% used for single column text, or when the title appears in the first column
% of two column text (technotes). 
\def\@IEEEstatictitlevskip{{\normalfont\normalsize
% adjust spacing to next text
% v1.6b handle peer review papers
\if@peerreviewoption
% for peer review papers, the same value is used for both title pages
% regardless of the other paper modes
   \vskip 1\baselineskip plus 0.375\baselineskip minus 0.1875\baselineskip
\else
   \if@confmode% conference
      \vskip 1\baselineskip plus 0.375\baselineskip minus 0.1875\baselineskip%
   \else%
      \if@technote% technote
         \vskip 1\baselineskip plus 0.375\baselineskip minus 0.1875\baselineskip%
      \else% journal uses more space
         \vskip 2.5\baselineskip plus 0.75\baselineskip minus 0.375\baselineskip%
      \fi
   \fi
\fi}}


% V1.6
% This is a dynamically determined rigid spacer between the title/authors 
% and the main text. This is used only for single column titles over two 
% column text (most common)
% This is bit tricky because we have to ensure that the textheight of the
% main text is an integer multiple of \baselineskip
% otherwise underfull vbox problems may develop in the second column of the
% text on the titlepage
% The possible use of \pubid must also be taken into account.
\def\@IEEEdynamictitlevspace{{%
    % we run within a group so that all the macros can be forgotten when we are done
    \def\thanks##1{\relax}%don't allow \thanks to run when we evaluate the vbox height
    \normalfont\normalsize% we declare more descriptive variable names
    \let\@maintextheight=\@IEEEtrantmpdimenA%height of the main text columns
    \let\@INTmaintextheight=\@IEEEtrantmpdimenB%height of the main text columns with integer # lines
    % set the nominal and minimum values for the title spacer
    % the dynamic algorithm will not allow the spacer size to
    % become less than \@MINtitlevspace - instead it will be
    % lengthened
    % default to journal values
    \def\@NORMtitlevspace{2.5\baselineskip}%
    \def\@MINtitlevspace{2\baselineskip}%
    % conferences and technotes need tighter spacing
    \if@confmode%conference
     \def\@NORMtitlevspace{1\baselineskip}%
     \def\@MINtitlevspace{0.75\baselineskip}%
    \fi
    \if@technote%technote
      \def\@NORMtitlevspace{1\baselineskip}%
      \def\@MINtitlevspace{0.75\baselineskip}%
    \fi%
    % get the height that the title will take up
    \if@peerreviewoption
       \settoheight{\@maintextheight}{\vbox{\hsize\textwidth \@IEEEpeerreviewmaketitle}}%
    \else
       \settoheight{\@maintextheight}{\vbox{\hsize\textwidth \@maketitle}}%
    \fi
    \@maintextheight=-\@maintextheight% title takes away from maintext, so reverse sign
    % add the height of the page textheight
    \advance\@maintextheight by \textheight%
    % correct for title pages using pubid
    \if@peerreviewoption\else
       % peerreview papers use the pubid on the cover page only.
       % And the cover page uses a static spacer.
       \if@IEEEusingpubid\advance\@maintextheight by -\@pubidpullup\fi
    \fi%
    % subtract off the nominal value of the title bottom spacer
    \advance\@maintextheight by -\@NORMtitlevspace%
    % \topskip takes away some too
    \advance\@maintextheight by -\topskip%
    % calculate the column height of the main text for lines
    % now we calculate the main text height as if holding
    % an integer number of \normalsize lines after the first
    % and discard any excess fractional remainder
    % we subtracted the first line, because the first line
    % is placed \topskip into the maintext, not \baselineskip like the
    % rest of the lines.
    \@INTmaintextheight=\@maintextheight%
    \divide\@INTmaintextheight  by \baselineskip%
    \multiply\@INTmaintextheight  by \baselineskip%
    % now we calculate how much the title spacer height will
    % have to be reduced from nominal (\@REDUCEmaintextheight is always
    % a positive value) so that the maintext area will contain an integer
    % number of normal size lines
    % we change variable names here (to avoid confusion) as we no longer
    % need \@INTmaintextheight and can reuse its dimen register
    \let\@REDUCEmaintextheight=\@INTmaintextheight%
    \advance\@REDUCEmaintextheight by -\@maintextheight%
    \advance\@REDUCEmaintextheight by \baselineskip%
    % this is the calculated height of the spacer
    % we change variable names here (to avoid confusion) as we no longer
    % need \@maintextheight and can reuse its dimen register
    \let\@COMPENSATElen=\@maintextheight%
    \@COMPENSATElen=\@NORMtitlevspace% set the nominal value
    % we go with the reduced length if it is smaller than an increase
    \ifdim\@REDUCEmaintextheight < 0.5\baselineskip\relax%
     \advance\@COMPENSATElen by -\@REDUCEmaintextheight%
     % if the resulting spacer is too small back out and go with an increase instead
     \ifdim\@COMPENSATElen<\@MINtitlevspace\relax%
      \advance\@COMPENSATElen by \baselineskip%
     \fi%
    \else%
     % go with an increase because it is closer to the nominal than a decrease
     \advance\@COMPENSATElen by -\@REDUCEmaintextheight%
     \advance\@COMPENSATElen by \baselineskip%
    \fi%
    % set the calculated rigid spacer
    \vspace{\@COMPENSATElen}}}



% V1.6
% we allow the user access to the last part of the title area
% useful in emergencies such as when a different spacing is needed
% This text is NOT compensated for in the dynamic sizer.
\let\@IEEEaftertitletext=\relax
\def\IEEEaftertitletext#1{\def\@IEEEaftertitletext{#1}}


% V1.6 have abstract and keywords strip leading spaces, pars and newlines
% so that spacing is more tightly controlled.
\def\abstract{\normalfont%
    \if@twocolumn%
      \@IEEEabskeysecsize\bfseries\textit{Abstract}---\,%
    \else%
      \begin{center}\vspace{-1.78ex}\@IEEEabskeysecsize\textbf{Abstract}\end{center}\quotation\@IEEEabskeysecsize%
    \fi\@IEEEgobbleleadPARNLSP}
% V1.6 IEEE wants only 1 pica from end of abstract to introduction heading when in 
% conference mode (the heading already has this much above it)
\def\endabstract{\relax\if@confmode\vspace{0ex}\else\vspace{1.34ex}\fi\par\if@twocolumn\else\endquotation\fi%
    \normalfont\normalsize}


\def\keywords{\normalfont%
    % IEEE uses the term (in bold italics) "Index Terms" now. 
    \if@twocolumn%
      \@IEEEabskeysecsize\bfseries\textit{Index Terms}---\,\relax%
    \else%
      \begin{center}\@IEEEabskeysecsize\bfseries Index Terms\end{center}\quotation\@IEEEabskeysecsize%
    \fi\@IEEEgobbleleadPARNLSP}
\def\endkeywords{\relax\if@technote\vspace{1.34ex}\else\vspace{0.67ex}\fi%
    \par\if@twocolumn\else\endquotation\fi%
    \normalsize\normalfont}


% gobbles all leading \, \\ and \par, upon finding first token that
% is not a \ , \\ or a \par, it ceases and returns that token
% 
% used to strip leading \, \\ and \par from the input
% so that such things in the beginning of an environment will not
% affect the formatting of the text
\long\def\@IEEEgobbleleadPARNLSP#1{\let\@IEEEswallowthistoken=0%
\let\@IEEEgobbleleadPARNLSPtoken#1%
\let\@IEEEgobbleleadPARtoken=\par%
\let\@IEEEgobbleleadNLtoken=\\%
\let\@IEEEgobbleleadSPtoken=\ %
\def\@IEEEgobbleleadSPMACRO{\ }%
\ifx\@IEEEgobbleleadPARNLSPtoken\@IEEEgobbleleadPARtoken%
\let\@IEEEswallowthistoken=1%
\fi%
\ifx\@IEEEgobbleleadPARNLSPtoken\@IEEEgobbleleadNLtoken%
\let\@IEEEswallowthistoken=1%
\fi%
\ifx\@IEEEgobbleleadPARNLSPtoken\@IEEEgobbleleadSPtoken%
\let\@IEEEswallowthistoken=1%
\fi%
% a control space will come in as a macro
% when it is the last one on a line
\ifx\@IEEEgobbleleadPARNLSPtoken\@IEEEgobbleleadSPMACRO%
\let\@IEEEswallowthistoken=1%
\fi%
% if we have to swallow this token, do so and taste the next one
% else spit it out and stop gobbling
\ifx\@IEEEswallowthistoken 1\let\@IEEEnextgobbleleadPARNLSP=\@IEEEgobbleleadPARNLSP\else%
\let\@IEEEnextgobbleleadPARNLSP=#1\fi%
\@IEEEnextgobbleleadPARNLSP}%




% TITLING OF SECTIONS
\def\@IEEEsectpunct{:\ \,}  % Punctuation after run-in section heading  (headings which are
                            % part of the paragraphs), need little bit more than a space

\def\@seccntformat#1{\csname the#1dis\endcsname\hskip 0.5em\relax}

\def\@sect#1#2#3#4#5#6[#7]#8{%
  \ifnum #2>\c@secnumdepth%
     \def\@svsec{}%
  \else%
     \refstepcounter{#1}
     % load section label and spacer into \@svsec
     \protected@edef\@svsec{\@seccntformat{#1}\relax}%
  \fi%
  \@tempskipa #5\relax%
  \ifdim \@tempskipa>\z@% tempskipa determines whether is treated as a high
     \begingroup #6\relax% or low level heading
      \noindent % subsections are NOT indented
       % print top level headings. \@svsec is label, #8 is heading title
       %\@hangfrom{\hskip #3\relax\@svsec}{\interlinepenalty \@M #8\par}
       % IEEE does not block indent the section title text, it flows like normal
       \relax{\hskip #3\relax\@svsec}{\interlinepenalty \@M #8\par}%
     \endgroup%
     % got rid of sectionmark stuff
     % \csname #1mark\endcsname{#7}
     \addcontentsline{toc}{#1}{\ifnum #2>\c@secnumdepth\relax\else%
               \protect\numberline{\csname the#1\endcsname}\fi#7}%
  \else % printout low level headings
     % svsechd seems to swallow the trailing space, protect it with \mbox{}
     % got rid of sectionmark stuff
     \def\@svsechd{#6\hskip #3\@svsec #8\@IEEEsectpunct\mbox{}%\csname #1mark\endcsname{#7}
     \addcontentsline{toc}{#1}{\ifnum #2>\c@secnumdepth\relax\else%
               \protect\numberline{\csname the#1\endcsname}\fi#7}}
  \fi%skip down
  \@xsect{#5}}
  
% section* handler
\def\@ssect#1#2#3#4#5{\@tempskipa #3\relax%
  \ifdim \@tempskipa>\z@%
     %\begingroup #4\@hangfrom{\hskip #1}{\interlinepenalty \@M #5\par}\endgroup
     % IEEE does not block indent the section title text, it flows like normal
     \begingroup \noindent #4\relax{\hskip #1}{\interlinepenalty \@M #5\par}\endgroup%
  % svsechd swallows the trailing space, protect it with \mbox{}
  \else \def\@svsechd{#4\hskip #1\relax #5\@IEEEsectpunct\mbox{}}\fi%
  \@xsect{#3}}


%% SECTION heading spacing and font
%%
% arguments are: #1 - sectiontype name
% (for \@sect)   #2 - section level
%                #3 - section heading indent
%                #4 - top separation (absolute value used, neg indicates not to indent main text)
%                     If negative, make stretch parts negative too!
%                #5 - (absolute value used) positive: bottom separation after heading,
%                      negative: amount to indent main text after heading
%                Both #4 and #5 negative means to indent main text and use negative top separation
%                #6 - font control
% You've got to have \normalfont\normalsize in the font specs below to prevent
% trouble when you do something like:
% \section{Note}{\ttfamily TT-TEXT} is known to ... 
% IEEE sometimes REALLY stretches the area before a section
% heading by up to about 0.5in. However, it may not be a good
% idea to let LaTeX have quite this much rubber.
\if@confmode%
% IEEE wants section heading spacing to decrease for conference mode
\def\section{\@startsection{section}{1}{\z@}{1.5ex plus 1.5ex minus 0.5ex}%
{0.7ex plus 1ex minus 0ex}{\normalfont\normalsize\centering\scshape}}%
\def\subsection{\@startsection{subsection}{2}{\z@}{1.5ex plus 1.5ex minus 0.5ex}%
{0.7ex plus .5ex minus 0ex}{\normalfont\normalsize\itshape}}%
\else % for journals
\def\section{\@startsection{section}{1}{\z@}{3.0ex plus 1.5ex minus 1.5ex}% V1.6 3.0ex from 3.5ex
{0.7ex plus 1ex minus 0ex}{\normalfont\normalsize\centering\scshape}}%
\def\subsection{\@startsection{subsection}{2}{\z@}{3.5ex plus 1.5ex minus 1.5ex}%
{0.7ex plus .5ex minus 0ex}{\normalfont\normalsize\itshape}}%
\fi
% decided to put in a little rubber above the section, might help somebody
\def\subsubsection{\@startsection{subsubsection}{3}{\parindent}{0ex plus 0.1ex minus 0.1ex}%
{0ex}{\normalfont\normalsize\itshape}}%
\def\paragraph{\@startsection{paragraph}{4}{2\parindent}{0ex plus 0.1ex minus 0.1ex}%
{0ex}{\normalfont\normalsize\itshape}}%



%% ENVIRONMENTS
% "box" symbols at end of proofs
\def\QEDclosed{\mbox{\rule[0pt]{1.3ex}{1.3ex}}} % for a filled box
% V1.6 some journals use an open box instead that will just fit around a closed one
\def\QEDopen{{\setlength{\fboxsep}{0pt}\setlength{\fboxrule}{0.2pt}\fbox{\rule[0pt]{0pt}{1.3ex}\rule[0pt]{1.3ex}{0pt}}}}
\def\QED{\QEDclosed} % default to closed

\def\proof{\noindent\hspace{2em}{\itshape Proof: }}
\def\endproof{\hspace*{\fill}~\QED\par\endtrivlist\unskip}
%\itemindent is set to \z@ by list, so define new temporary variable
\newdimen\@IEEEtmpitemindent
\def\@begintheorem#1#2{\@IEEEtmpitemindent\itemindent\topsep 0pt\rmfamily\trivlist%
    \item[\hskip \labelsep{\indent\itshape #1\ #2:}]\itemindent\@IEEEtmpitemindent}
\def\@opargbegintheorem#1#2#3{\@IEEEtmpitemindent\itemindent\topsep 0pt\rmfamily \trivlist%
% V1.6 IEEE is back to using () around theorem names which are also in italics
% Thanks to Christian Peel for reporting this.
    \item[\hskip\labelsep{\indent\itshape #1\ #2\ (#3):}]\itemindent\@IEEEtmpitemindent}
\def\@endtheorem{\endtrivlist\unskip}

% V1.6
% display command for the section the theorem is in - so that \thesection
% is not used as this will be in Roman numerals when we want arabic.
% LaTeX2e uses \def\@thmcounter#1{\noexpand\arabic{#1}} for the theorem number
% (second part) display and \def\@thmcountersep{.} as a separator.
\def\@IEEEthmcounterin#1{\arabic{#1}}
% redefine the #1#2[#3] form of newtheorem to use a hook to \@IEEEthmcounterin
\def\@xnthm#1#2[#3]{%
  \expandafter\@ifdefinable\csname #1\endcsname
    {\@definecounter{#1}\@newctr{#1}[#3]%
     \expandafter\xdef\csname the#1\endcsname{%
     \noexpand\@IEEEthmcounterin{#3}\@thmcountersep\@thmcounter{#1}}%
     \global\@namedef{#1}{\@thm{#1}{#2}}%
     \global\@namedef{end#1}{\@endtheorem}}}



%% SET UP THE DEFAULT PAGESTYLE
\ps@headings
\pagenumbering{arabic}

% normally the page counter starts at 1
\setcounter{page}{1}
% however, for peerreview the cover sheet is page 0 or page -1
% (for duplex printing)
\if@peerreviewoption
   \if@twoside
      \setcounter{page}{-1}
   \else
      \setcounter{page}{0}
   \fi
\fi

% standard book class behavior - let bottom line float up and down as
% needed when single sided
\if@twoside\else\raggedbottom\fi
% if two column - turn on twocolumn, allow word spacings to stretch more and
% enforce a rigid position for the last lines
\if@twocolumnmode
% the peer review option delays invoking twocolumn
   \if@peerreviewoption\else
      \twocolumn
   \fi
\sloppy 
\flushbottom
\fi




% \APPENDIX and \APPENDICES definitions

% This is the \@ifmtarg command from the LaTeX ifmtarg package
% by Peter Wilson (CUA) and Donald Arseneau
% \@ifmtarg is used to determine if an argument to a command
% is present or not.
% For instance:
% \@ifmtarg{#1}{\typeout{empty}}{\typeout{has something}}
% \@ifmtarg is used with our redefined \section command if
% \appendices is invoked.
% The command \section will behave slightly differently depending
% on whether the user specifies a title: 
% \section{My appendix title}
% or not:
% \section{}
% This way, we can eliminate the blank lines where the title
% would be, and the unneeded : after Appendix in the table of
% contents 
\begingroup
\catcode`\Q=3
\long\gdef\@ifmtarg#1{\@xifmtarg#1QQ\@secondoftwo\@firstoftwo\@nil}
\long\gdef\@xifmtarg#1#2Q#3#4#5\@nil{#4}
\endgroup
% end of \@ifmtarg defs

% save the "original" meaning of \section so we can redefine
% \section after a call to \appendix or \appendices 
\let\@IEEEappendixsavesection\section

% neat trick to grab and process the argument from \section{argument}
% we process differently if the user invoked \section{} with no
% argument (title)
% note we reroute the call to the old \section*
\def\@IEEEprocessthesectionargument#1{%
\@ifmtarg{#1}{%
\@IEEEappendixsavesection*{Appendix \thesectiondis}%
\addcontentsline{toc}{section}{Appendix \thesection}}{%
\@IEEEappendixsavesection*{Appendix \thesectiondis \\* #1}%
\addcontentsline{toc}{section}{Appendix \thesection: #1}}}

% we use this if the user calls \section{} after
% \appendix-- which has no meaning. So, we ignore the
% command and its argument. Then, warn the user.
\def\@IEEEdestroythesectionargument#1{\typeout{** WARNING: Ignoring useless
\protect\section\space in Appendix (line \the\inputlineno).}}


% remember \thesection forms will be displayed in \ref calls
% and in the Table of Contents.
% The \sectiondis form is used in the actual heading itself

% appendix command for one single appendix
% normally has no heading. However, if you want a 
% heading, you can do so via the optional argument:
% \appendix[Optional Heading]
\def\appendix{\relax}
\renewcommand{\appendix}[1][]{\par%
    % v1.6 keep hyperref's identifiers unique
    \def\theHsection{Appendix.A}%
    % v1.6 adjust hyperref's string name for the section
    \xdef\Hy@chapapp{appendix}%
    \setcounter{section}{0}%
    \setcounter{subsection}{0}%
    \setcounter{subsubsection}{0}%
    \setcounter{paragraph}{0}%
    \def\thesection{}%
    \def\thesectiondis{}% 
    \def\thesubsection{\Alph{subsection}}%
    \refstepcounter{section}% update the \ref counter
    \@ifmtarg{#1}{\@IEEEappendixsavesection*{Appendix}%
                  \addcontentsline{toc}{section}{Appendix}}{%
             \@IEEEappendixsavesection*{Appendix \\* #1}%
             \addcontentsline{toc}{section}{Appendix: #1}}%
    % redefine \section command for appendix
    % leave \section* as is
    \def\section{\@ifstar{\@IEEEappendixsavesection*}{%
                    \@IEEEdestroythesectionargument}}% throw out the argument
                                                     % of the normal form
}


% provides the user a way to choose between
% Appendix A
%    and
% Appendix I
% notation
% defaults to Roman.
\newif\ifuseRomanappendices
\useRomanappendicestrue


% appendices command for multiple appendices
% user then calls \section with an argument (possibly empty) to
% declare the individual appendices
\def\appendices{\par%
    % v1.6 keep hyperref's identifiers unique
    \def\theHsection{Appendix.\Alph{section}}%
    % v1.6 adjust hyperref's string name for the section
    \xdef\Hy@chapapp{appendix}%
    \setcounter{section}{-1}% we want \refstepcounter to use section 0
    \setcounter{subsection}{0}%
    \setcounter{subsubsection}{0}%
    \setcounter{paragraph}{0}%
    \ifuseRomanappendices%
    \def\thesection{\Roman{section}}%
    \def\thesectiondis{\Roman{section}}%
    \else%
    \def\thesection{\Alph{section}}%
    \def\thesectiondis{\Alph{section}}%
    \fi%
    \refstepcounter{section}% update the \ref counter
    \setcounter{section}{0}% NEXT \section will be the FIRST appendix
    % redefine \section command for appendices
    % leave \section* as is
    \def\section{\@ifstar{\@IEEEappendixsavesection*}{% process the *-form
                    \refstepcounter{section}% or is a new section so,
                    \@IEEEprocessthesectionargument}}% process the argument 
                                                 % of the normal form
}



% \PARstart
% Definition for the big two line drop cap letter at the beginning of the
% first paragraph of journal papers. The first argument is the first letter
% of the first word, the second argument is the remaining letters of the
% first word which will be rendered in upper case.
% In V1.6 this has been completely rewritten to:
% 
% 1. no longer have problems when the user begins an environment
%    within the paragraph that uses \PARstart.
% 2. auto-detect and use the current font family
% 3. revise handling of the space at the end of the first word so that
%    interword glue will now work as normal.
% 4. produce correctly aligned edges for the (two) indented lines.
% 
% We generalize things via control macros - playing with these is fun too.
% 
% the number of lines that are indented to clear it
\def\@IEEEPARstartDROPLINES{2}
% minimum number of lines left on a page to allow a \@PARstart
% Does not take into consideration rubber shrink, so it tends to
% be overly cautious
\def\@IEEEPARstartMINPAGELINES{2}
% the depth the letter is lowered below the baseline
% the height (and size) of the letter is determined by the sum
% of this value and the height of a capital "T" in the current
% font. It is a good idea to set this value in terms of the baselineskip
% so that it can respond to changes therein.
\def\@IEEEPARstartDROPDEPTH{1.1\baselineskip}
% This is the separation distance from the drop letter to the main text.
% Lengths that depend on the font (i.e., ex, em, etc.) will be referenced
% to the font that is active when PARstart is called. 
\def\@IEEEPARstartSEP{0.15em}


% definition of \PARstart
% THIS IS A CONTROLLED SPACING AREA, DO NOT ALLOW SPACES WITHIN THESE LINES
% 
% The token \@IEEEPARstartfont will be globally defined after the first use
% of \PARstart and will be a font command which creates the big letter
% The first argument is the first letter of the first word and the second
% argument is the rest of the first word(s).
\def\PARstart#1#2{\par{%
% if this page does not have enough space, break it and lets start
% on a new one
\@IEEEtranneedspace{\@IEEEPARstartMINPAGELINES\baselineskip}{\relax}%
% calculate the desired height of the big letter
% it extends from the top of a capital "T" in the current font
% down to \@IEEEPARstartDROPDEPTH below the current baseline
\settoheight{\@IEEEtrantmpdimenA}{T}%
\addtolength{\@IEEEtrantmpdimenA}{\@IEEEPARstartDROPDEPTH}%
% extract the name of the current font in bold
% and place it in \@IEEEPARstartFONTNAME
\def\@IEEEPARstartGETFIRSTWORD##1 ##2\relax{##1}%
{\bfseries%
\edef\@IEEEPARstartFONTNAMESPACE{\fontname\font\space}%
\xdef\@IEEEPARstartFONTNAME{\expandafter\@IEEEPARstartGETFIRSTWORD\@IEEEPARstartFONTNAMESPACE\relax}}%
% define a font based on this name with a point size equal to the desired
% height of the drop letter
\font\@IEEEPARstartsubfont\@IEEEPARstartFONTNAME\space at \@IEEEtrantmpdimenA\relax%
% save this value as a counter (integer) value (sp points)
\@IEEEtrantmpcountA=\@IEEEtrantmpdimenA%
% now get the height of the actual letter produced by this font size
\settoheight{\@IEEEtrantmpdimenB}{\@IEEEPARstartsubfont\MakeUppercase{#1}}%
% If something bogus happens like the first argument is empty or the
% current font is strange, do not allow a zero height.
\ifdim\@IEEEtrantmpdimenB=0pt\relax%
\typeout{** WARNING: PARstart drop letter has zero height! (line \the\inputlineno)}%
\typeout{ Forcing the drop letter font size to 10pt.}%
\@IEEEtrantmpdimenB=10pt%
\fi%
% and store it as a counter
\@IEEEtrantmpcountB=\@IEEEtrantmpdimenB%
% Since a font size doesn't exactly correspond to the height of the capital
% letters in that font, the actual height of the letter, \@IEEEtrantmpcountB,
% will be less than that desired, \@IEEEtrantmpcountA
% we need to raise the font size, \@IEEEtrantmpdimenA 
% by \@IEEEtrantmpcountA / \@IEEEtrantmpcountB
% But, TeX doesn't have floating point division, so we have to use integer
% division. Hence the use of the counters.
% We need to reduce the denominator so that the loss of the remainder will
% have minimal affect on the accuracy of the result
\divide\@IEEEtrantmpcountB by 200%
\divide\@IEEEtrantmpcountA by \@IEEEtrantmpcountB%
% Then reequalize things when we use TeX's ability to multiply by
% floating point values
\@IEEEtrantmpdimenB=0.005\@IEEEtrantmpdimenA%
\multiply\@IEEEtrantmpdimenB by \@IEEEtrantmpcountA%
% \@IEEEPARstartfont is globaly set to the calculated font of the big letter
% We need to carry this out of the local calculation area to to create the
% big letter.
\global\font\@IEEEPARstartfont\@IEEEPARstartFONTNAME\space at \@IEEEtrantmpdimenB%
% Now set \@IEEEtrantmpdimenA to the width of the big letter
% We need to carry this out of the local calculation area to set the
% hanging indent
\settowidth{\global\@IEEEtrantmpdimenA}{\@IEEEPARstartfont\MakeUppercase{#1}}}%
% end of the isolated calculation environment
% add in the extra clearance we want
\advance\@IEEEtrantmpdimenA by \@IEEEPARstartSEP%
% \@IEEEtrantmpdimenA has the width of the big letter plus the
% separation space and \@IEEEPARstartfont is the font we need to use
% Now, we make the letter and issue the hanging indent command
% The letter is placed in a box of zero width and height so that other
% text won't be displaced by it.
\noindent\hangindent\@IEEEtrantmpdimenA\hangafter=-\@IEEEPARstartDROPLINES%
\makebox[0pt][l]{\hspace{-\@IEEEtrantmpdimenA}\raisebox{-\@IEEEPARstartDROPDEPTH}[0pt][0pt]{\@IEEEPARstartfont\MakeUppercase{#1}}}\MakeUppercase{#2}}


% V1.6 \CMPARstart is no longer needed as \PARstart now uses whatever
% the current font family is.
% \CMPARstart is provided here for backward compatability.
\let\CMPARstart=\PARstart



% determines if the space remaining on a given page is equal to or greater
% than the specified space of argument one
% if not, execute argument two (only if the remaining space is greater than zero)
% and issue a \newpage
% 
% example: \@IEEEtranneedspace{2in}{\vfill}
% 
% Does not take into consideration rubber shrinkage, so it tends to
% be overly cautious
% Based on an example posted by Donald Arseneau
% Note this macro uses \@IEEEtrantmpdimenB internally for calculations,
% so DO NOT PASS \@IEEEtrantmpdimenB to this routine
% if you need a dimen register, import with \@IEEEtrantmpdimenA instead
\def\@IEEEtranneedspace#1#2{\penalty-100\begingroup%shield temp variable
\@IEEEtrantmpdimenB\pagegoal\advance\@IEEEtrantmpdimenB-\pagetotal% space left
\ifdim #1>\@IEEEtrantmpdimenB\relax% not enough space left
\ifdim\@IEEEtrantmpdimenB>\z@\relax #2\fi%
\newpage%
\fi\endgroup}



% BIOGRAPHY ENVIRONMENT
% Allows user to enter BIOGRAPHY leaving place for picture (adapts to font size)
% As of V1.5, a new optional argument allows you to have a real graphic!
% V1.5 and later also fixes the "colliding biographies" which could happen when a 
% biography's text was shorter than the space for the photo.
% MDS 7/2001
% V1.6 prevent multiple biographies from making multiple TOC entries
\newif\if@biographyTOCentrynotmade
\global\@biographyTOCentrynotmadetrue

% biography counter so hyperref can jump directly to the biographies
% and not just the previous section
\newcounter{biography}
\setcounter{biography}{0}

% photo area size
\def\@IEEEBIOphotowidth{1.0in}    % width of the biography photo area
\def\@IEEEBIOphotodepth{1.25in}   % depth (height) of the biography photo area
% area cleared for photo
\def\@IEEEBIOhangwidth{1.14in}    % width cleared for the biography photo area
\def\@IEEEBIOhangdepth{1.25in}    % depth cleared for the biography photo area
                                  % actual depth will be a multiple of 
                                  % \baselineskip, rounded up
\def\@IEEEBIOskipN{4\baselineskip}% nominal value of the vskip above the biography

\newenvironment{biography}[2][]{\normalfont\footnotesize%
\unitlength 1in\parskip=0pt\par\parindent 1em\interlinepenalty500%
% we need enough space to support the hanging indent
% the nominal value of the spacer
% and one extra line for good measure
\@IEEEtrantmpdimenA=\@IEEEBIOhangdepth%
\advance\@IEEEtrantmpdimenA by \@IEEEBIOskipN%
\advance\@IEEEtrantmpdimenA by 1\baselineskip%
% if this page does not have enough space, break it and lets start
% with a new one
\@IEEEtranneedspace{\@IEEEtrantmpdimenA}{\relax}%
% nominal spacer can strech, not shrink use 1fil so user can out stretch with \vfill
\vskip \@IEEEBIOskipN plus 1fil minus 0\baselineskip%
% the default box for where the photo goes
\def\@IEEEtempbiographybox{{\setlength{\fboxsep}{0pt}\framebox{\begin{minipage}[b][\@IEEEBIOphotodepth][c]{\@IEEEBIOphotowidth}\centering PLACE\\ PHOTO\\ HERE \end{minipage}}}}%
%
% detect if the optional argument was supplied, this requires the
% \@ifmtarg command as defined in the appendix section above
% and if so, override the default box with what they want
\@ifmtarg{#1}{\relax}{\def\@IEEEtempbiographybox{\mbox{\begin{minipage}[b][\@IEEEBIOphotodepth][c]{\@IEEEBIOphotowidth}%
\centering%
#1%
\end{minipage}}}}% end if optional argument supplied
% Make an entry into the table of contents only if we have not done so before
\if@biographyTOCentrynotmade%
% link labels to the biography counter so hyperref will jump
% to the biography, not the previous section
\setcounter{biography}{-1}%
\refstepcounter{biography}%
\addcontentsline{toc}{section}{Biographies}%
\global\@biographyTOCentrynotmadefalse%
\fi%
% one more biography
\refstepcounter{biography}%
% Make an entry for this name into the table of contents 
\addcontentsline{toc}{subsection}{#2}%
% V1.6 properly handle if a new paragraph should occur while the
% hanging indent is still active. Do this by redefining \par so
% that it will not start a new paragraph. (But it will appear to the
% user as if it did.) Also, strip any leading pars, newlines, or spaces.
\let\@IEEEBIOORGparCMD=\par% save the original \par command
\edef\par{\hfil\break\indent}% the new \par will not be a "real" \par
\settoheight{\@IEEEtrantmpdimenA}{\@IEEEtempbiographybox}% get height of biography box
\@IEEEtrantmpdimenB=\@IEEEBIOhangdepth%
\@IEEEtrantmpcountA=\@IEEEtrantmpdimenB% countA has the hang depth
\divide\@IEEEtrantmpcountA by \baselineskip%  calculates lines needed to produce the hang depth
\advance\@IEEEtrantmpcountA by 1% ensure we overestimate
% set the hanging indent
\hangindent\@IEEEBIOhangwidth%
\hangafter-\@IEEEtrantmpcountA%
% reference the top of the photo area to the top of a capital T
\settoheight{\@IEEEtrantmpdimenB}{\mbox{T}}%
% set the photo box, give it zero width and height so as not to disturb anything
\noindent\makebox[0pt][l]{\hspace{-\@IEEEBIOhangwidth}\raisebox{\@IEEEtrantmpdimenB}[0pt][0pt]{\raisebox{-\@IEEEBIOphotodepth}[0pt][0pt]{\@IEEEtempbiographybox}}}%
% now place the author name and begin the bio text
\noindent\textbf{#2\ }\@IEEEgobbleleadPARNLSP}{\relax\let\par=\@IEEEBIOORGparCMD\par%
% 7/2001 V1.5 detect when the biography text is shorter than the photo area
% and pad the unused area - preventing a collision from the next biography entry
% MDS
\ifnum \prevgraf <\@IEEEtrantmpcountA\relax% detect when the biography text is shorter than the photo
    \advance\@IEEEtrantmpcountA by -\prevgraf% calculate how many lines we need to pad
    \advance\@IEEEtrantmpcountA by -1\relax% we compensate for the fact that we indented an extra line
    \@IEEEtrantmpdimenA=\baselineskip% calculate the length of the padding
    \multiply\@IEEEtrantmpdimenA by \@IEEEtrantmpcountA%
    \noindent\rule{0pt}{\@IEEEtrantmpdimenA}% insert an invisible support strut
\fi%
\par\normalfont}



% V1.6
% added biography without a photo environment
\newenvironment{biographynophoto}[1]{%
% Make an entry into the table of contents only if we have not done so before
\if@biographyTOCentrynotmade%
% link labels to the biography counter so hyperref will jump
% to the biography, not the previous section
\setcounter{biography}{-1}%
\refstepcounter{biography}%
\addcontentsline{toc}{section}{Biographies}%
\global\@biographyTOCentrynotmadefalse%
\fi%
% one more biography
\refstepcounter{biography}%
% Make an entry for this name into the table of contents 
\addcontentsline{toc}{subsection}{#1}%
\normalfont\footnotesize\interlinepenalty500%
\vskip 4\baselineskip plus 1fil minus 0\baselineskip%
\parskip=0pt\par%
\noindent\textbf{#1\ }\@IEEEgobbleleadPARNLSP}{\relax\par\normalfont}


% provide the user with some old font commands
% got this from article.cls
\DeclareOldFontCommand{\rm}{\normalfont\rmfamily}{\mathrm}
\DeclareOldFontCommand{\sf}{\normalfont\sffamily}{\mathsf}
\DeclareOldFontCommand{\tt}{\normalfont\ttfamily}{\mathtt}
\DeclareOldFontCommand{\bf}{\normalfont\bfseries}{\mathbf}
\DeclareOldFontCommand{\it}{\normalfont\itshape}{\mathit}
\DeclareOldFontCommand{\sl}{\normalfont\slshape}{\@nomath\sl}
\DeclareOldFontCommand{\sc}{\normalfont\scshape}{\@nomath\sc}
\DeclareRobustCommand*\cal{\@fontswitch\relax\mathcal}
\DeclareRobustCommand*\mit{\@fontswitch\relax\mathnormal}


% SPECIAL PAPER NOTICE COMMANDS
% 
% holds the special notice text
\def\@specialpapernotice{\relax}
 
% for special papers, like invited papers, the user can do:
% \specialpapernotice{(Invited Paper)} before \maketitle
\def\specialpapernotice#1{\if@confmode%
\def\@specialpapernotice{{\sublargesize\textit{#1}\vspace*{1em}}}%
\else%
\def\@specialpapernotice{{\\*[1.5ex]\sublargesize\textit{#1}}\vspace*{-2ex}}%
\fi}




% PUBLISHER ID COMMANDS
% to insert a publisher's ID footer
% V1.6 \pubid has been changed so that the change in page size and style
% occurs in \maketitle. \pubid must now be issued prior to \maketitle
% use \pubidadjcol as before - in the second column of the title page
% These changes allow \maketitle to take the reduced page height into
% consideration when dynamically setting the space between the author 
% names and the maintext.
%
% the amount the main text is pulled up to make room for the
% publisher's ID footer
% IEEE uses about 1.3\baselineskip for journals, 
% dynamic title spacing will clean up the fraction
\def\@pubidpullup{1.3\baselineskip}
\if@technote
% for technotes it must be an integer of baselineskip as there can be no
% dynamic title spacing for two column mode technotes (the title is in the
% in first column) and we should maintain an integer number of lines in the
% second column
% There are some examples (such as older issues of "Transactions on
% Information Theory") in which IEEE really pulls the text off the ID for
% technotes - about 0.55in (or 4\baselineskip). We'll use 2\baselineskip
% and call it even.
\def\@pubidpullup{2\baselineskip}
\fi

% holds the ID text
\def\@pubid{\relax}

% flag so \maketitle can tell if \pubid was called
\newif\if@IEEEusingpubid
\global\@IEEEusingpubidfalse
% issue this command in the page to have the ID at the bottom
% V1.6 use before \maketitle
\def\pubid#1{\def\@pubid{#1} \global\@IEEEusingpubidtrue}


% command which will pull up (shorten) the column it is executed in
% to make room for the publisher ID. Place in the second column of
% the title page when using \pubid
% Is smart enough not to do anything when in single column text or
% if the user hasn't called \pubid
% currently needed in for the second column of a page with the
% publisher ID. If not needed in future releases, please provide this
% command and define it as \relax for backward compatibility
% v1.6b do not allow command to operate if the peer review option has been 
% selected because \pubidadjcol will not be on the cover page.
\def\pubidadjcol{\if@peerreviewoption\else\if@twocolumn\if@IEEEusingpubid\enlargethispage{-\@pubidpullup}\fi\fi\fi}

% Special thanks to Peter Wilson, Daniel Luecking, and the other
% gurus at comp.text.tex, for helping me to understand how best to
% implement the pubid command in LaTeX.



%% Lockout some commands under various conditions

% general purpose bit bucket
\newsavebox{\@IEEEtranrubishbin}

% flags to prevent multiple warning messages
\newif\if@IEEEWARNthanks
\newif\if@IEEEWARNPARstart
\newif\if@IEEEWARNCMPARstart
\newif\if@IEEEWARNkeywords
\newif\if@IEEEWARNbiography
\newif\if@IEEEWARNbiographynophoto
\newif\if@IEEEWARNpubid
\newif\if@IEEEWARNpubidadjcol
\newif\if@IEEEWARNIEEEmembership
\newif\if@IEEEWARNIEEEaftertitletext
\@IEEEWARNthankstrue
\@IEEEWARNPARstarttrue
\@IEEEWARNCMPARstarttrue
\@IEEEWARNkeywordstrue
\@IEEEWARNbiographytrue
\@IEEEWARNbiographynophototrue
\@IEEEWARNpubidtrue
\@IEEEWARNpubidadjcoltrue
\@IEEEWARNIEEEmembershiptrue
\@IEEEWARNIEEEaftertitletexttrue


%% Lockout some commands when in various modes, but allow them to be restored if needed
%%
% save commands which might be locked out
% so that the user can later restore them if needed
\let\@IEEESAVECMDthanks\thanks
\let\@IEEESAVECMDPARstart\PARstart
\let\@IEEESAVECMDCMPARstart\CMPARstart
\let\@IEEESAVECMDkeywords\keywords
\let\@IEEESAVECMDendkeywords\endkeywords
\let\@IEEESAVECMDbiography\biography
\let\@IEEESAVECMDendbiography\endbiography
\let\@IEEESAVECMDbiographynophoto\biographynophoto
\let\@IEEESAVECMDendbiographynophoto\endbiographynophoto
\let\@IEEESAVECMDpubid\pubid
\let\@IEEESAVECMDpubidadjcol\pubidadjcol
\let\@IEEESAVECMDIEEEmembership\IEEEmembership
\let\@IEEESAVECMDIEEEaftertitletext\IEEEaftertitletext


% disable \PARstart when in draft mode
% This may have originally been done because the pre-V1.6 drop letter
% algorithm had problems with a non-unity baselinestretch
% At any rate, it seems too formal to have a drop letter in a draft
% paper.
\if@draftclsmode
\def\PARstart#1#2{#1#2\if@IEEEWARNPARstart\typeout{** ATTENTION: \noexpand\PARstart is disabled in draft mode (line \the\inputlineno).}\fi\global\@IEEEWARNPARstartfalse}
\def\CMPARstart#1#2{#1#2\if@IEEEWARNPARstart\typeout{** ATTENTION: \noexpand\CMPARstart is disabled in draft mode (line \the\inputlineno).}\fi\global\@IEEEWARNCMPARstartfalse}
\fi
% and for technotes
\if@technote
\def\PARstart#1#2{#1#2\if@IEEEWARNPARstart\typeout{** WARNING: \noexpand\PARstart is locked out for technotes (line \the\inputlineno).}\fi\global\@IEEEWARNPARstartfalse}
\def\CMPARstart#1#2{#1#2\if@IEEEWARNPARstart\typeout{** WARNING: \noexpand\CMPARstart is locked out for technotes (line \the\inputlineno).}\fi\global\@IEEEWARNCMPARstartfalse}
\fi


% lockout unneeded commands when in conference mode
\if@confmode
% when locked out, \thanks, \keywords, \biography, \biographynophoto, \pubid,
% \IEEEmembership and \IEEEaftertitletext will all swallow their given text. 
% \PARstart and \CMPARstart will output a normal character instead
% warn the user about these commands only once to prevent the console screen
% from filling up with redundant messages
\def\thanks#1{\if@IEEEWARNthanks\typeout{** WARNING: \noexpand\thanks is locked out when in conference mode (line \the\inputlineno).}\fi\global\@IEEEWARNthanksfalse}
\def\PARstart#1#2{#1#2\if@IEEEWARNPARstart\typeout{** WARNING: \noexpand\PARstart is locked out when in conference mode (line \the\inputlineno).}\fi\global\@IEEEWARNPARstartfalse}
\def\CMPARstart#1#2{#1#2\if@IEEEWARNPARstart\typeout{** WARNING: \noexpand\CMPARstart is locked out when in conference mode (line \the\inputlineno).}\fi\global\@IEEEWARNCMPARstartfalse}

\renewenvironment{keywords}[1]{\if@IEEEWARNkeywords\typeout{** WARNING: \noexpand\keywords is locked out when in conference mode (line \the\inputlineno).}\fi\global\@IEEEWARNkeywordsfalse%
\setbox\@IEEEtranrubishbin\vbox\bgroup}{\egroup\relax}

% LaTeX treats environments and commands with optional arguments differently.
% the actual ("internal") command is stored as \\commandname 
% (accessed via \csname\string\commandname\endcsname )
% the "external" command \commandname is a macro with code to determine
% whether or not the optional argument is presented and to provide the 
% default if it is absent. So, in order to save and restore such a command
% we would have to save and restore \\commandname as well. But, if LaTeX
% ever changes the way it names the internal names, the trick would break.
% Instead let us just define a new environment so that the internal
% name can be left undisturbed.
\newenvironment{@IEEEbogusbiography}[2][]{\if@IEEEWARNbiography\typeout{** WARNING: \noexpand\biography is locked out when in conference mode (line \the\inputlineno).}\fi\global\@IEEEWARNbiographyfalse%
\setbox\@IEEEtranrubishbin\vbox\bgroup}{\egroup\relax}
% and make biography point to our bogus biography
\let\biography=\@IEEEbogusbiography
\let\endbiography=\end@IEEEbogusbiography

\renewenvironment{biographynophoto}[1]{\if@IEEEWARNbiographynophoto\typeout{** WARNING: \noexpand\biographynophoto is locked out when in conference mode (line \the\inputlineno).}\fi\global\@IEEEWARNbiographynophotofalse%
\setbox\@IEEEtranrubishbin\vbox\bgroup}{\egroup\relax}

\def\pubid#1{\if@IEEEWARNpubid\typeout{** WARNING: \noexpand\pubid is locked out when in conference mode (line \the\inputlineno).}\fi\global\@IEEEWARNpubidfalse}
\def\pubidadjcol{\if@IEEEWARNpubidadjcol\typeout{** WARNING: \noexpand\pubidadjcol is locked out when in conference mode (line \the\inputlineno).}\fi\global\@IEEEWARNpubidadjcolfalse}
\def\IEEEmembership#1{\if@IEEEWARNIEEEmembership\typeout{** WARNING: \noexpand\IEEEmembership is locked out when in conference mode (line \the\inputlineno).}\fi\global\@IEEEWARNIEEEmembershipfalse}
\def\IEEEaftertitletext#1{\if@IEEEWARNIEEEaftertitletext\typeout{** WARNING: \noexpand\IEEEaftertitletext is locked out when in conference mode (line \the\inputlineno).}\fi\global\@IEEEWARNIEEEaftertitletextfalse}
\fi


% provide a way to restore the commands that are locked out
\def\IEEEoverridecommandlockouts{%
\typeout{** ATTENTION: Overriding command lockouts (line \the\inputlineno).}%
\let\thanks\@IEEESAVECMDthanks%
\let\PARstart\@IEEESAVECMDPARstart%
\let\CMPARstart\@IEEESAVECMDCMPARstart%
\let\keywords\@IEEESAVECMDkeywords%
\let\endkeywords\@IEEESAVECMDendkeywords%
\let\biography\@IEEESAVECMDbiography%
\let\endbiography\@IEEESAVECMDendbiography%
\let\biographynophoto\@IEEESAVECMDbiographynophoto%
\let\endbiographynophoto\@IEEESAVECMDendbiographynophoto%
\let\pubid\@IEEESAVECMDpubid%
\let\pubidadjcol\@IEEESAVECMDpubidadjcol%
\let\IEEEmembership\@IEEESAVECMDIEEEmembership%
\let\IEEEaftertitletext\@IEEESAVECMDIEEEaftertitletext}


\endinput

%%%%%%%%%%%%%%%%%%%%%%%%%%%%% End of IEEEtran.cls  %%%%%%%%%%%%%%%%%%%%%%%%%%%%
% That's all folks!

