\chapter{Introduction}
\label{ch:Intro}

The Standard Model (SM) is a robust framework that allows for accurate predictions of processes involving the interactions of elementary particles. It has been developed over the course of many decades which involved many additions such as three generations of quarks and leptons and the combination of Electromagnetism and the Weak force into a single theory. Unfortunately, we have evidence of that is not explained by the SM and had eluded physicists for many years. We intend to do a search for a potential particle beyond the current SM.

\section{Motivation}
\label{sec:Motivation}

Through various methods of experiment, we have seen that the current knowledge of elementary particles in the SM do not explain all of the known matter in the universe. Due to galactic velocity rotational curves we can deduce that the mass of galaxies must be much more than the visible matter that can be measured. This "dark matter" must act quite weakly with the three forces of the SM, but can still be quite important for gravitational effects. There are many theory beyond the SM that can explain these effects, but we will concentrate on Supersymmetry (SUSY) because of the potential for a dark matter candidate, a solution to the Hierarchy problem, and a potential Grand Unified Theory (GUT).

SUSY allows for every fermion to have a bosonic partner, and vice-versa, which has all of the same quantum numbers except for a difference of $\frac{1}{2}$-integer spin. We know that since it has not yet been found that the theory must be a broken symmetry, such that the masses of the SUSY partners must have a higher mass than the SM particle. One of the main aspect of SUSY, is the conservation of $R$-parity, which implies the existence of a Lightest Supersymmetric Particle (LSP). This LSP could be a potential dark matter candidate since it is stable and weakly interacting. 

The Hierarchy problem is due to the loop interactions of massive quarks with the Higgs boson. This coupling causes a quadratic divernence of the Higgs mass, $m_H$, and can only be renormalized fine tuning the coupling parameters. A potential solution is the coupling of an addition bosonic particle to the fermionic quarks. This additional coupling allows for a cancellation of quadratic divergence into a logarithmic divergence which is then renormalizable by the normal methods. This can renormalize the mass of the Higgs boson to the known value of $m_H=125.18$ \GeV{} that was discovered in 2012. Finally, SUSY also allows for a mechanism for a potential GUT. The additional superpartners allows for the three forces of the SM to converge at a large energy scale of order $10^{16}$ GeV. We now need to develop a search strategy to try and detect these potential SUSY particles.

\section{Search}
\label{sec:search}

There are many possible potential searches for SUSY particles. In the Minimal Supersymmetric Model (MSSM), it can be determined that the lightest squark that can be produced at the Large Hadron Collider (LHC) is the stop, \st{}, which will then decay into SM particles and the LSP.  We are developing a all hadronic search to find the \st{} which will be as inclusive as possible, so we will include all possible decay modes to get additional limits or possible detection. 

Due to the all hadronic aspect of the decay the main backgrounds are caused by, a lost lepton due to a lepton passing the kinematic cuts of the detector, \Znunu{} background where the missing energy is caused by the neutrinos escaping the detector, QCD background which can be caused by the mis-measurement of jets in the event, and a rare background caused by many different types of processes which are estimated by a three lepton method of identification. 

We have developed 183 search regions to look for the top squark, \st{}. This is then used to get a statistical estimation of the limits on the cross section for each of the processes that we are interested in searching. In the following chapters, we will look into the derivation of the SM and motivations for SUSY, then a description of the CMS detector and the various subdetectors, next the description of the object selection and search strategy for the analysis, the methods of each individual reaction are described in the next section, and finally the description and analysis of the results and limits. 

