\chapter{Compact Muon Solenoid}
\label{ch:CMS}

\section{The Detector}
\label{sec:cmsIntro}

The Compact Muon Solenoid (CMS) is a particle detector as part of the Large Hadron Collider (LHC) which is located near Geneva, Switzerland as part of the CERN collaboration. The CMS detector is 21.6 m long, 15 m diameter, and 14,000 tons and is used to detect many different species of particles. It is separated into layers that, from the interaction vertex outward are, the silicon tracker, Electromagnetic Calorimeter (ECAL), Hadronic Calorimeter (HCAL), superconducting solenoid, and the muon chanbers, see Fig. \ref{CMSSlice}. 

\begin{figure}
 	\centering
	\includegraphics[width=0.75\textwidth]{cms_slice.jpg}
 	\caption{A cross-section of the CMS detector, oriented by looking down the direction of the beam pipe. }
 	\label{CMSSlice} 
\end{figure}

The CMS detector is designed to detect the decay products of most of the particles of the SM, except for neutrinos since they are weakly interacting and will almost certainly pass through the entire earth without an interaction. A defining feature of CMS is the 12.6-m long, 5.9 m inner diameter, 3.8 T superconducting solenoid. This is used to bend the trajectory of charged particles thoughout the detector, such that we can reconstruct the momentum and charge of the particles. The LHC provides a 13 TeV proton-proton beam (4.5 TeV heavy ion) beam with a bunch crossing every 25 ns (50 ns) to produce interaction at luminosities of $2.5\times10^{34} \cm^{-2}s^{-1}$. 

The coordinate system of CMS has the origin at the nominal collision point in the center of the detector. The $y$-axis points vertically upward, $x$-axis points radially inward toward the center of the LHC, and $z$-axis points along the beam directions towards the Jura mountains from LHC Point 5. The polar angle $\theta$ is measured from the $z$-axis and the azimuthal angle $\phi$ is measured in the $x-y$ plane from the $x$-axis. The pseudorapidity of a particle is defined as $\eta=-\ln\tan(\theta/2)$ where two notable values are $\eta=0$ at $\theta=\pi/2$ and $\eta=\inf$ at $\theta=0$. The transverse components of momentum, $p_T$, and energy, $E_T$, are computed using the $x$ and $y$ components of the particles. 

\subsection{Tracker}
\label{sec:Tracker}

The silicon tracker is made up of two different detectors, the silicon pixels and the silicon strip tracker. This is the inner most detector for CMS and sees the largest flux of particles during operation. This requires it to be radiation hard and operate with a fine granularity. 

\begin{figure}
 	\centering
	\includegraphics[width=0.75\textwidth]{Tracker_Configuration.png}
 	\caption{Geometry of the CMS Tracker, the inner most region in green is the pixel detector while the outer region in blue and red are the silicon strips.}
 	\label{CMSTracker} 
\end{figure}

\subsubsection{Pixel Detector}
\label{subsec:Pixel}

The pixel detector was recently upgraded during the winter of 2016/2017. It is approximately 1 m long with four barrel layers ranging from 3.0, 6.8, 10.2, and 16.0 \cm  and three endcap disks, see fig. \ref{CMSTracker}. Since it is the closest detector to the interaction vertex it therefore has the highest particle flux at $10^7/\cm^2/s$ at $r\equiv10$ cm. The resolution is $9.4$ \mum in $r-\phi$ and $20-45$ \mum in $z$.

\begin{figure}
 	\centering
	\includegraphics[width=0.5\textwidth]{Pixel_Module.png}
 	\caption{Components of the pixel modules. Made up of a silicon layer, a grid of 8 ROCs which are attached via bump bonds. This is all controlled with a TBM connection to read out data.}
 	\label{PixelModule} 
\end{figure}

The pixel detector contains 1184 modules in the barrel pixels (BPIX) and 672 modules in the forward pixels (FPIX). The number of individual pixels is 79 (45) million in the BPIX (FPIX) regions, respectively, with a pixel size of $100\times150 \text{ }\mum^2$. A pixel module contains two layers, a silicon layer that is bump bonded to 16 Readout Chips (ROCs) which form a module of 66560 pixels, see fig. \ref{PixelModule}. Each unit is controlled with one or more Token Bit Managers (TBMs) which controls the readout of the digital signal from the pixels to the Front-end Driver (FED). For BPIX Layers 3,4 and all of FPIX there is 1 TBM per module. BPIX layer 2 has 2 TBMs with each one controlling 8 ROCs, while BPIX layer 1 has 4 TBMs with each one controlling 4 ROCs. The information from each module is split into two channels with each containing 8 ROCs. These are encoded together by the TBM before being sent to the FED.

The silicon pixel system is set up as a reverse p-n junction, where the pixels are in the n-type region. As a charged particle travels through the silicon it creates electron-hole pairs. A voltage difference is applied to the silicon such that the electrons will deposit onto the pixels. Since the detector is inside of the magnetic field, a lorentz drift will cause the electrons to reach more than one pixel and increase the resolution. As the pixel system continues to be irradiated with large quantities of particles the voltage in the silicon decreases. This will lead to less charge sharing between the pixels and a decrease in resolution of particle locations. 

\begin{figure}
 	\centering
	\includegraphics[width=0.75\textwidth]{FEROL_Data_Throughput_Thesis.pdf}
 	\caption{Measuring the throughput of the FED with the emulated and simulated events provided by the FED Tester. The Data rate is shown as the solid blue line with the corresponding trigger rate as the dotted red line. The simulated event sizes are shown as their equivalent emulated hits/ROC/channel on the data line.}
 	\label{FEDThroughput} 
\end{figure}

The data, from the pixels, is sent via optical fiber to the FED where it decodes and processes the information. The FED is responsible for identifying the relevant data, determining possible error states, and packaging the information to be sent to the central Data Acquisition (cDAQ) of CMS. Each of the 108 FEDs, for the pixels, receives 24 independent fibers from the detector. Each of these fibers contains 2 channels from the pixel module. Through robust testing with the FED Tester [Cite here], we have confirmed that the FED is able to attain a maximum data throughput of approximately 7.5 Gbps, see fig. \ref{FEDThroughput}. 

\subsubsection{Silicon Strips}
\label{subsec:Strips}

The silicon strips have a 200 m$^2$ active region with 15148 modules that are distributed in 10 barrel layers and $9+3$ endcap disks.
This has a cell size ranging from $10 \cm \times80$ \mum to $25 \cm \times 180$ \mum since the particle flux decreases further away from the vertex, [MAYBE ANOTHER FIGURE]. It has a resolution of $23-24$ \mum in $r-\phi$ and $23$ \mum in $z$ for the microstrip tracker

How is it different from the pixels?

\subsection{Electromagnetic Calorimeter}
\label{sec:ECAL}

The ECAL is a homogeneous calorimeter made out of 61200 lead tungstate $(\text{PbWO}_4)$ crystals in the barrel and 7324 crystals in each endcap. The barrel region has an inner radius of 129 cm and covers a pseudorapidity range of $0<|\eta|<1.479$. The encaps are 314 cm from the interaction point and cover a range $1.479<|\eta|<3.0$ in pseudorapidity. Lead tungstate was chosen for teh crystals since it has a short radiation length, $X_0=0.89\cm$, fast with 80\% of the light being emitted within 25 ns, and it's radiation hard. Each crystal in the barrel has a cross-section of $\approx22\times22 \text{mm}^2$ and length of 230 mm, while the endcap crystals are $28.6\times28.6 \text{mm}^2$ and length of 220 mm corresponding to $25.8X_0$ and $24.7X_0$, respectively. An ECAL uses electromagnetic showers to detect charged particles or photons that interact electromagnetically. Electrons traveling through the material will radiate a photon via bremsstrahlung. A photon will pair produce two electrons. Combining these two processes leads to electromagnetic showers as the particles travel through the detector. The process will continue until a critical energy is reached such that an electron cannot radiate any further and will then lose energy via collisions. The resulting light is recorded by silicon avalanche photodiode (vaccuum phototriodes) in the barrel (endcap). 

\subsection{Hadronic Calorimeter}
\label{sec:HCAL}

The HCAL is a hermetic calorimeter consisting of alternating layers of brass as the absorber material and a scintillator. Brass is chosen since it is non-magnetic and has a relatively short interaction length. In the scintillator, a portion of the energy from the hadron in converted into visible light which is then measured by a hybrid photodiode tube to measure the energy. The barrel part of the HCAL consists of 2304 towers that are segmented into $\Delta\eta\times\Delta\phi=0.087\times0.087$ pieces that cover a region $0<|\eta|<1.4$ in pseudorapidity. The encap region consists of 2304 towers with varying segmentation sizes and a coverage of $1.3<|\eta|<3.0$. 

There are two additional parts of the HCAL to allow for maximum coverage of the detector volume. There is a outer hadron detector that is located outside the superconducting solenoid, which covers a slightly smaller pseudorapidity range as the barrel region. They serve as a tail catcher for hadron showers that penetrate all the way through the inner HCAL and solenoid. A forward hadron calorimeter, located 11.2 m from the interaction point covering a pseudorapidity $3.0<\eta<5.0$, made out of steel/quartz fiber is specifically designed for the columnated Cerenkov light in this region. 

\subsection{Superconducting solenoid}
\label{sec:Solenoid}

Containing all of this is the superconducting solenoid which is 12.6 m long and a 5.9 m radius. The field strength is 3.8 T which has a stored energy of approximately 2.7 GJ. The magnet is designed such that a muon with momentum, $p=1$ TeV, will have a momentum resolution of $\Delta p/p\approx10\%$. The solenoid is a high-purity aluminium-stabilized conductor, which is a similar material used in other large solenoids. 

\subsection{Muon Chambers}
\label{sec:muCham}

The muon system has three main detection systems that are use to identify a muon candidate. In the barrel region, drift tube (DT) chambers are used since the neutron background, muon rate, and magnetic field are all small. In the endcaps, cathode strip chambers (CSCs) are used since the relative values stated before are much larger. Included throughout the whole system are resistive plate chambers (RPC). 

\begin{figure}
 	\centering
	\includegraphics[width=0.75\textwidth]{MuonChambers.png}
 	\caption{A quarter cross-section of the three detection systems for CMS. }
 	\label{MuonChambers} 
\end{figure}

The DT consists on 250 chambers in 4 barrel layers at a radii of 4.0, 4.9, 5.9, and 7.0 m from the beam axis. A DT chamber is an array of anode wires in a gaseous medium where the walls are cathodes. A muon passing through the gas will ionize some atoms which are then forced towards the anode wires by an electric field. The drift time of the electrons can then be calculated to within a couple of ns such that a good spatial resolution is achieved. The maximum designed drift length is 2.0 cm. Each station of the DT will give muon vector for each candidate with a $\phi$ precision of 100 \mum in postion and 1 mrad in direction. 

The CSC system uses the same concept as the DT system, but also includes a measurement of the ions that follow the electric field to the cathode strips. In this system the anode wires and the cathode strips are perpendicular so the collected charge on both provide a accurate position measurement. 

The RPC system contains two parallel plates, anode and cathode, the charge is measured by external metallic strips that can quickly measure the momentum of a muon and decide if the event is worth triggering.

\section{Physics Objects}\label{PhysObj}
There are many different types of physics ojbect that we are interested in when working with particle physics experiments. Since the particles that we are interested in have very short lifetimes, $\mathcal{O}(\text{decay})=10^{-23}$ s, we mainly interact with the decay products of the event, such as, jets, heavy object tagging, particle identification, missing transverse momentum $\met$, $H_T$, $N_{SV}$, $p_T^b$, soft b-tagging, $m_T(b_{1,2}, \met)$, Initial State Radiation, and lepton identification.

NEEDS MORE SPECIFICS ON JETS IDS AND B JET CSVs

\subsection{Jets}\label{Jets}
In an interaction, because of color confinement, whenever a quark is made it must come in pairs $(q, q^\prime)$ such that the total color charge of the interaction in neutral. Typically due to conservation of momentum the quarks may originally be produced near the interaction point but will quickly start to move away from each other. Eventually the quarks will move far enough apart and will have enough potential energy in the gluon connections between them that it is now more efficient to create a new quark-antiquark $(q\overline{q})$ pair. This wil continue to occur in a sequence of radiating gluons and producing new pairs of charged particles. In the final state, the energy deposited in the HCAL is due to a cluster of charged particles of a certain radius, $\Delta R=\sqrt{\Delta\eta^2+\Delta\phi^2}$. There are many algorithms to reconstruct the jets, we are mainly interested in the anti-kT Jet algorithm [reference] method with uses the transverse momentum of the particles within a certain radius $\Delta R = 0.4 (0.8)$ for slim(fat) jets. 

\subsection{Heavy Object Tagging}\label{HeavyObject}
Since this search is looking for a massive particle which then decays to slightly less massive particles we need to be able to identify and distinguish between them. Firstly, \bquark-tagged jets which are jets that are likely to have originated from a \bquark quark. These are identified reconstructing where the jet originated from and comparing the distance away from the interaction point. A \bquark quark is a relatively long-lived particle and can travel many millimeters before decaying. Since we have a resolution of \mum this is not a problem. For \bquark quarks with large transverse momentum, we use a Combined Secondary Vertex (CSVv2/DeepCSV) algorithm depending on the year the data was taken. 

\subsection{B-Tagging}\label{Btagging}
B-tagged jets are identified using the Run 2 version of the Deep Combined Secondary Vertex (DeepCSV) algorithm. The medium working point recommended by the B-tag POG, corresponding to a threshold of 0.6324m 0.4941, and 0.4184 for the 2016, 2017, and 2018 on the DeepCSV, is used for the b-jets for this analysis.

\subsection{Missing Transverse Momentum}\label{MET}
The missing transverse momentum is the negative vector sum of the total transverse momentum measured in the detector,

\begin{equation}
\met=-\sum_{i\in\text{vis}}\overrightarrow{p}_{i, T},
\end{equation}

where the momentum runs over every visible(vis) particle in the event. Ideally, if the detector was $100\%$ this quantity would always be zero due to conservation of momentum, but many things, such as detector efficiency, paticles that are weakly interacting, or particles beyond the SM that will be entirely missed by the detector. Because of these, this object is a good discriminator for searching for physics beyond the SM. 

\subsection{$H_T$}\label{HT}
Another interesting quantity is $H_T$, which is the scalar sum of the $p_T$ of all of the jets in an event,

\begin{equation}
H_T=\sum_{i\in\text{jets}}p_{i,T}.
\end{equation}

This quantity is quite useful when trying to identify massive particles, which are much more massive than the average particle. 

\subsection{Secondary Vetices}\label{SV}

The ability to identify secondary vertices is essential for search for the top squark. In searches for the top squark, one of the main decay is to a bottom quark, \bquark, and a neutralino, \neutralino. Since the b quark is a long lived particle, about $10^{}-12$ seconds, that will travel many mm before decaying into other particles. The tracks from the charged particles will cause tracks in the various detectors of CMS. These tracks are defined by hits that are then reconstructed using multiple algorithms which can reconstruct the paths of the particles through the detector. Once the tracks are reconstructed, we can backtrack to a point where the particle originated from and then identify is flavor. When we reconstruct a particles to a point that is displaced from the primary vertex it is known as a secondary vertex and has the potential to be an long lived particle. 

\subsection{B Quark Transverse Momentum}\label{bTransverseMomentum}

Initial-state radiation (ISR) may be clustered into one of the large-$R$ jets clustered with a distance parameter of $0.8$. We use the larger radius jets to be sensitive to ISR with gluon splitting. The ISR jet is identified as being the hardest of the large-$R$ jets with $p_T>200$ GeV which fails the loose b-tagging working point and is not identified as a top or W. 

\subsection{Soft b-tagging} \label{softB}
This search targets also models that produce very soft bottom or charm quarks. A typical $p_T$ spectrum of the \bquark quarks in three T2fbd signal models is shown in fig. . A large fraction of events contain b quarks with $p_T$ below the 20 GeV jet-$p_T$ threshold which may thus fail to be reconstructed as jets or become b-tagged. Identification of these soft quarks improves our ability to separate potential signal events from the SM background. We therefore aim to identify b/c quarks based on the presence of a secondary vertex (SV) reconstructed using the Inclusive Vertex Finder (IVF) algorithm. Additional requirements on SV observables are applied to suppress the background originating from light quarks. These selected SV may be referred to as soft b tags and are constructed to be orthogonal to the jets and b-tagged jet used in this analysis. 

The requirements on each SV to pass the soft b-tagging definition are:
\begin{itemize}
	 \item The distance in the transverse plane between the SV and the PV is less than 3 cm.
	 \item The significance of the distance, SIP3D, between the SV and thePV is greater than 4.
	 \item The pointing angle, defined as $cos(\overrightarrow{PV,SV},\overrightarrow{p}_{SV})$, is greater than 0.98, where $\overrightarrow{p}_{SV}$ is the total four-momentum of the tracks associated to the SV. 
	 \item The number of tracks associated to the SV is greater or equal to 3.
	 \item The $p_T$ of the SV is less than 20 GeV.
	 \item The distance to any jet, $\Delta R(\text{jet},\overrightarrow{p}_{SV})$, is greater than 0.4 to achieve the orthogonality to the jets and b-tagged jets.
 \end{itemize}
 
 This definition leads to $\approx 50 \%$ efficiency to identify a soft b/c quark, for $\approx3\%$ fake rate.
 
 \subsection{Lepton Identification}\label{EleMuonID}
 There are two sets of selection criteria used in the analysis for electrons and muons. A set of veto criteria are used to efficiently reject events with an isolated electron of muon in the search region, while more stringent requirements are imposed in some control regions that require high purity samples of isolated leptons.
 
 Electron candidates are identified via a set of selection criteria established by the EGamma POG based on "Spring16" simulated samples in the 25ns bunch spacing scenario. The corresponding thresholds imposed on relevant variables are summarized in Table . The "Veto" working point is used to keep events with electrons out of the final search region. The "Medium" working point is used for the selection of leptonic control regions. 
 
 \begin{table}
 \centering
 \begin{tabular}{|*{5}{c|}}
 \hline
 \multirow{2}{*}{} & \multicolumn{2}{c}{Veto Working Point} & \multicolumn{2}{c}{Medium Working Point} \\
 \hline
 & Barrel & Endcap & Barrel & Endcap \\
\hline
$\sigma_{i\eta i \eta} (\text{full} 5 \times 5) <$ & 0.0114 & 0.0352 & 0.0101 & 0.0283 \\
$|\Delta\eta_{\text{in}}|<$ & 0.0152 & 0.0113 & 0.0103 & 0.00733 \\
$|\Delta\phi_{\text{in}}|<$ & 0.216 & 0.237 & 0.0336 & 0.114 \\
$\frac{h}{E} <$ & 0.181 & 0.116 & 0.0876 & 0.0678 \\
$\frac{1}{E}-\frac{1}{p}<$ & 0.207 & 0.174 & 0.0174 & 0.0898 \\
$|d_0|<$ & 0.0564 & 0.222 & 0.0118 & 0.0739 \\
$|d_z|<$ & 0.472 & 0.921 & 0.373 & 0.602 \\
$N(\text{expected missing inner hits})\leq$ & 2 & 3 & 2 & 1 \\
Conversion veto & pass & pass & pass & pass \\
\hline
 \end{tabular}
 \caption{Electron identification requirements, defined separately for electrons in the ECAL barrel and endcap regions. The tabulated numbers for each working point are the thresholds applied to the corresponding quantities in the first column.}
 \label{ElectronID}
 \end{table}
 
 The loose muon definition recommended by the Muon POG is used for the purposes of the muon veto. A loose muon is identified as a PF muon and can be either a global muon or an arbitrated tracker muon. Only candidates with transverse (longitudinal) impact parameter $|d_0|<0.2$ cm $(|d_z|<0.5 \text{cm},)$ with respect to the primary vertex, are considered. The muon selection in leptonic control regions relies on the medium working point defined by the Muon POG for higher purity. The medium muon requirements are summarized in Table . They are applied in addition to the loose muon requirements. Selected muons are required to have transverse (longitudinal) impact parameters $|d_0|<0.05$ cm $(|d_z|<0.1 \text{cm}),$ with respect to the primary vertex. 

\begin{table}
 \centering
 \begin{tabular}{|*{2}{c|}}
 \hline
Loose muon selection & yes \\ 
Fraction of valid tracker hits & $> 0.8$ \\
\hline
\hline
\multicolumn{2}{c}{In addition, either of the following sets of requirements:} \\
\hline
Global muon & yes \\
Normalized global-track $\chi^2$ & $< 3$ \\
Tracker-Standalone position match & $< 12$ \\
Track kink finder & $< 20$ \\
Segment compatibility & $>0.303$ \\
\hline
\multicolumn{2}{c}{OR} \\
\hline
Segment compatibility & $> 0.451$ \\
\hline
 \end{tabular}
 \caption{}
 \label{MuonID}
 \end{table}
 
 The isolation requirements used in the selection of electrons and muons aim to achieve a high efficiency for correctly identifying events that contain prompt leptons (i.e. leptons that do not originate from heavy flavor decays) in the boosted topologies and busy hadronic environments that are typical for this analysis. They are based on the mini-isolation quantity, which is a measure of the lepton's local isolation. Mini-isolation is computed as the summed $p_T$ of PF candidates within a $\Delta R$ cone centered on the lepton candidate. The cone size depends on the lepton $p_T$ as indicated in Table  and is intended to be small enought to reduce overlaps with jets in the event while also being large enough to contain the products of leptonic b-decays. Higher values of instantaneous luminosity result in a reduced efficiency for the isolation requirement due to an increase in the number of particles originating from additional (pileup) interactions that enter the isolation cone. In order to minimize this effect, the calculated isolation quantity is corrected for the estimated contribution from pileup particles. The correction is applied by subtracting the product of the estimated average pileup density $(\rho)$ in the event with an effective area, $A_{eff}$, related to the geometrical size of the isolation cone. The residual dependence of the isolation quantity on $\rho$ for a given cone size is accounted for in $A_{eff}$, which is determined by taking the slope of a linear fit to the uncorrected isolation as a function of $\rho$. The correction factor is $\eta$-dependent, and calibrated separately for charged particles, neutral hadrons, and photons contributing to the isolation. 
 
 Electrons and muons are considered to fulfill the veto isolation criteria if their mini-isolation is less than 0.1 or 0.2 respectrively, relative to the lepton $p_T$. The same thresholds are applied on the relative mini-isolation for electrons and muons in control samples. 
 
\subsection{Tau Identification}\label{TauID}
The Tau ID has been studies extensively. After multiple studies with looked into the custum MVA of the Analysis Note [] and the IsoTrack methods and Tau POG MVA method of Identifying Taus. 

\section{Search Strategy}\label{BaselineSearch}
\subsection{Trigger}


\subsection{Baseline Selection} \label{Baseline}

Following the same methods as above, we have a loose pre-selection which is refered to as the baseline selection. This will place a selection on jets and \met which is used to eliminate a large fraction of background events. We define the baseline selection as:

\begin{itemize}
	\item $N_{e/\mu}, (p_T>5 \GeV, |\eta|<2.4)$
\end{itemize}

