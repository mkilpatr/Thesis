\chapter{Search Region Design}
\label{ch:SR}

Using MC simulations that model the SM background for this process we want to reduce the number of events in our Search region. This is an all hadronic search so we are looking at event with zero tagged leptons. Unfortunately, some can get in by not passing the kinematic cuts or just by the non 100 \% of the detector. There is a small nonzero inefficiency of mistagging a lepton as something else. 

\section{Minimizing the ttZ background}
\label{sec:ttZBack}

For the ttZ interactions, we produce two top quarks that can then decay to two b quarks and two W bosons. A possible way to mimick our search region is two have multiple jets, i.e. b quarks that hadronize and W bosons that decay hadronically, but we also need missing energy. This will be in addition to the Z boson decaying into two neutrinos and thus creating a large amount of missing energy.

We now try to look at the differing kinematic structure of the background, ttZ, and the signal region, stop quarks decaying. Under the assumption that the Z boson is created by radiated from the top quark the resulting decay to neutrinos should be close, small $\Delta\phi$, between the resulting jets. For the signal, the missing energy is produced by the neutralino. When the stop quark decays into top quark and neutralino the top quark should recoil off of neutralino to essentially be back-to-back. This will cause a large angle, $\Delta\phi$, between them. We then want to use the kinematic variable, \dphisr, where 

\section{Lost Lepton Application}
\label{sec:ttbarsr}

Can we apply this to other backgrounds. For boosted tops the the missing energy caused by missing the lepton in the W boson decay. The variable \dphisr should also apply. Should work for wjts, tW, ttW.

\section{Search Regions}
\label{sec:SR}

The HM and LM Search regions should be defined and explained. Why are they defined the way they are?

\section{Search Region Optimization}
\label{sec:dphisrapp}

Look for an optimized cust for \dphisr to maximize \srootb{} in each SR. Could have a different cut for each region, but a combination to make it all the same would be nice. Since the signal can decay in multiple ways we need to optimize for all possible scenarios. Explain why we are maximizing \srootb

\section{Limits}
\label{sec:LimitswCut}

Looking at the significance and limits for the mass regions of the stop quark decay. Using the Higgs Combined tool, which includes statistics with a "maximal likelihood" fit? The cut, \dphisr, would hopefully improve the values, but an optimized cut has not been chosen yet. 