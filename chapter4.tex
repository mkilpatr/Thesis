\chapter{Supersymmetry}
\label{ch:SUSY}

Why does the universe have to have this symmetry?
Universe hase the inverse square law for gravity and EM.

More symmetry allows for a simpler description of the values. 

\section{Fundamental Problems in the standard model}
\label{sec:Hierarchy}

Hierarchy problem?
Dark Matter?
Grand Unified Theory?

Support material for each of these unknowns and how SUSY can solve them. Fine Tuning

\section{Superpartners}
\label{sec:superpartners}

Initial assumption that every fermion has a boson partner and vice versa. These partners are exactly the same but differ by half integer spin. Changes Higgs mass divergence from quadratic to logrithmic (renormalizable)

Must be broken such that the masses of these partners are larger.

\subsection{Chirality}
\label{subsec:chiral}

Equal numbers of fermions and bosons. How does the spin change? 

\section{Minimal Supersymetric Standard Model}
\label{sec:MSSM}

Soft supersymetry breaking. 

\subsection{R Parity}
\label{subsec:rparity}

New conserved parameter known as R parity. With this is allows for a stable particle that is a dark matter candidate. Other consequences.

\section{Mass Spectrums}

Higgs boson corrections. spectrum of squarks. 






