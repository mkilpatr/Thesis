\chapter{Conclusions}
\label{ch:Conclusions}

We have now combined all of the data, \datalumi{}, and simulation from Run 2 of CMS and computed limits on the cross-sections for the direct top squark production, $\st\rightarrow t\neutralino$, and an indirect top squark production mode, $\widetilde{g}\rightarrow t \bar{t}\neutralino$. 

\section{Interpreting Results}
\label{sec:Interp}

The limits have been improved to a limit on the masses $m_{\st}=1175$ GeV and $m_{\widetilde{g}}=2.05$ TeV, for the top squark and guino, respectively. Our analysis was however not able to exclude some low mass regions in the T2tt and a single island in the T1tttt mode. These regions are sensitive to SM fluctuations since they are of similar shape to the signals in this search. 

It is possible that a SM processes such as $t\bar{t}t\bar{t}$ could have an over estimated cross section which potentially could cause the missing energy needed to for the island in T1tttt. The regions in T2tt that cannot be excluded are sensitive to \Znunu{} background. The \Znunu{} systematics on the normalization and shape could play a significant role in the limit on these cross sections. 

\section{Outlook}
\label{sec:outlook}

We have been able to exclude some potential masses at a 95\% CL. This is by no means a statement that SUSY does not exist. As we have seen with the differences in Fig. \ref{fig:Run2-T2tt-limits} and \ref{T2ttANLimits}, taking more data and including improved estimations of backgrounds can still cause some of the previously excluded regions to not be excluded anymore. We still need to look at the CLs and limits for the other interesting signal in this search. 