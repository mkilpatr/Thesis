\chapter{Conclusions}
\label{ch:Conclusions}

We have now combined all of the data, \datalumi{}, and simulation from Run 2 of CMS and computed limits on the cross sections for the direct top squark production, $\st\rightarrow t\neutralino$, and an indirect top squark production mode, $\widetilde{g}\rightarrow t \bar{t}\neutralino$. 

\section{Interpreting Results}
\label{sec:Interp}

The limits have been improved to a mass $m_{\st}=1175$ GeV and $m_{\widetilde{g}}=2.05$ TeV, for the top squark and guino, respectively. Our analysis was however not able to exclude some low mass regions in the T2tt and a single island in the T1tttt mode. These regions are sensitive to SM fluctuations since they are of similar shape to the signals in this search or possible cross section variations when compared to the theory. It is possible that a SM processes such as $t\bar{t}t\bar{t}$ could have an over estimated cross section which potentially could cause the missing energy needed to for the island in T1tttt. 

We have described a new search for the top squark with a direct production mode and an indirect production mode. These have improved the expected and observed limits from previous analysis in References \cite{sirunyan_search_2017} and \cite{cms_collaboration_search_2018}. 

\section{Outlook}
\label{sec:outlook}

We have been able to expand the excludable regions of some potential masses at a 95\% CL. As we have seen with the differences in Fig. \ref{fig:Run2-T2tt-limits} and the equivalent one in Ref. \cite{sirunyan_search_2017}, taking more data and including improved estimations of backgrounds can still cause some of the previously excluded regions to not be excluded anymore. With this realization it is evident the by measuring more particle collisions at CMS will will improve upon current limits. As such, better methods for particle identification and background estimation will be necessary for searches in the future. 

We have only discussed two decay modes for this simplified SUSY model. We still need to analyze the other modes and discuss the improvements and shortcomings, that can lead to new searches in the field. We have expanded our knowledge of the top squark mass parameter space, but there are still properties under investigation for the future.
