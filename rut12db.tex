% REPORT DOCUMENT STYLE -- Released 19 Jan 88
%    for LaTeX version 2.09
% Copyright (C) 1988 by Leslie Lamport

% PREPARING A FOREIGN LANGUAGE VERSION:
%
% This document style is for documents prepared in the English language.
% To prepare a version for another language, various English words must
% be replaced.  All the English words that required replacement are
% indicated below, where we give the name of the command in which the
% words appear, and the entire line containing the word(s), with the
% actual words underlined.
% 
% \chapter and \appendix
%   \def\@chapapp{Appendix}
%                 ~~~~~~~~

%  ****************************************
%  *               FONTS                  *
%  ****************************************
%

\lineskip 1pt            % \lineskip is 1pt for all font sizes.
\normallineskip 1pt
\def\baselinestretch{1}

% Each size-changing command \SIZE executes the command
%        \@setsize\SIZE{BASELINESKIP}\FONTSIZE\@FONTSIZE
% where:
%   BASELINESKIP = Normal value of \baselineskip for that size.  (Actual 
%                  value will be \baselinestretch * BASELINESKIP.)
%
%  \FONTSIZE     = Name of font-size command.  The currently available
%                  (preloaded) font sizes are: \vpt (5pt), \vipt (6pt),
%                  \viipt (etc.), \viiipt, \ixpt, \xpt, \xipt, \xiipt,
%                  \xivpt, \xviipt, \xxpt, \xxvpt.
%  \@FONTSIZE    = The same as the font-size command except with an
%                  '@' in front---e.g., if \FONTSIZE = \xivpt then
%                  \@FONTSIZE = \@xivpt.
%
% For reasons of efficiency that needn't concern the designer,
% the document style defines \@normalsize instead of \normalsize .  This is
% done only for \normalsize, not for any other size-changing commands.

\def\@normalsize{\@setsize\normalsize{14.5pt}\xiipt\@xiipt
\abovedisplayskip 12pt plus3pt minus7pt%
\belowdisplayskip \abovedisplayskip
\abovedisplayshortskip  \z@ plus3pt%   
\belowdisplayshortskip  6.5pt plus3.5pt minus3pt%
\let\@listi\@listI}   % Setting of \@listi added 9 Jun 87

\def\normalsizedb{\@setsize\normalsize{29pt}\xiipt\@xiipt
\abovedisplayskip 12pt plus3pt minus7pt%
\belowdisplayskip \abovedisplayskip
\abovedisplayshortskip  \z@ plus3pt%   
\belowdisplayshortskip  6.5pt plus3.5pt minus3pt%
\let\@listi\@listI}   % Setting of \@listi added 9 Jun 87

\def\small{\@setsize\small{13.6pt}\xipt\@xipt
\abovedisplayskip 11pt plus3pt minus6pt%
\belowdisplayskip \abovedisplayskip
\abovedisplayshortskip  \z@ plus3pt%   
\belowdisplayshortskip  6.5pt plus3.5pt minus3pt
\def\@listi{\leftmargin\leftmargini %% Added 22 Dec 87
\parsep 4.5pt plus 2pt minus 1pt
            \itemsep \parsep
            \topsep 9pt plus 3pt minus 5pt}}

\def\footnotesize{\@setsize\footnotesize{12pt}\xpt\@xpt
\abovedisplayskip 10pt plus2pt minus5pt%
\belowdisplayskip \abovedisplayskip
\abovedisplayshortskip  \z@ plus3pt%   
\belowdisplayshortskip  6pt plus3pt minus3pt
\def\@listi{\leftmargin\leftmargini %% Added 22 Dec 87
\topsep 6pt plus 2pt minus 2pt\parsep 3pt plus 2pt minus 1pt
\itemsep \parsep}}

\def\scriptsize{\@setsize\scriptsize{9.5pt}\viiipt\@viiipt}
\def\tiny{\@setsize\tiny{7pt}\vipt\@vipt}
\def\large{\@setsize\large{18pt}\xivpt\@xivpt}
\def\Large{\@setsize\Large{22pt}\xviipt\@xviipt}
\def\LARGE{\@setsize\LARGE{25pt}\xxpt\@xxpt}
\def\huge{\@setsize\huge{30pt}\xxvpt\@xxvpt}
\let\Huge=\huge

\normalsize  % Choose the normalsize font.


%  ****************************************
%  *            PAGE LAYOUT               *
%  ****************************************
%
% All margin dimensions measured from a point one inch from top and side
% of page.  

% SIDE MARGINS:
\if@twoside                 % Values for two-sided printing:
   \oddsidemargin 0.5in      %   Left margin on odd-numbered pages.
   \evensidemargin 0.5in     %   Left margin on even-numbered pages.
   \marginparwidth 0pt    %   Width of marginal notes.
\else                       % Values for one-sided printing:
   \oddsidemargin 0.5in      %   Note that \oddsidemargin = \evensidemargin
   \evensidemargin 0.5in
   \marginparwidth 0pt 
\fi
\marginparsep 10pt          % Horizontal space between outer margin and 
                            % marginal note


% VERTICAL SPACING:        
                         % Top of page:
\topmargin 0in          %    Nominal distance from top of page to top of
                         %    box containing running head.
\headheight 12pt         %    Height of box containing running head.
\headsep 25pt            %    Space between running head and text.
% \topskip = 10pt        %    '\baselineskip' for first line of page.
                         % Bottom of page:
\footskip 30pt           %    Distance from baseline of box containing foot 
                         %    to baseline of last line of text.


% DIMENSION OF TEXT:
% 24 Jun 86: changed to explicitly compute \textheight to avoid roundoff.
% The value of the multiplier was calculated as the floor of the
% old \textheight minus \topskip, divided by \baselineskip for \normalsize.
% The old value of \textheight was 536.5pt.
% \textheight is the height of text (including footnotes and figures, 
% excluding running head and foot).

\textheight = 39\baselineskip
\advance\textheight by \topskip
\textwidth 6.0in         % Width of text line.
                         % For two-column mode: 
\columnsep 10pt          %    Space between columns 
\columnseprule 0pt       %    Width of rule between columns.

% A \raggedbottom command causes 'ragged bottom' pages: pages set to
% natural height instead of being stretched to exactly \textheight.

% FOOTNOTES:

\footnotesep 8.4pt    % Height of strut placed at the beginning of every
                      % footnote = height of normal \footnotesize strut,
                      % so no extra space between footnotes.

\skip\footins 10.8pt plus 4pt minus 2pt  % Space between last line of text and 
                                         % top of first footnote.

% FLOATS: (a float is something like a figure or table)
%
%  FOR FLOATS ON A TEXT PAGE:
%
%    ONE-COLUMN MODE OR SINGLE-COLUMN FLOATS IN TWO-COLUMN MODE:
\floatsep 14pt plus 2pt minus 4pt        % Space between adjacent floats moved
                                         % to top or bottom of text page.
\textfloatsep 14pt plus 2pt minus 4pt    % Space between main text and floats
                                         % at top or bottom of page.
\intextsep 14pt plus 4pt minus 4pt       % Space between in-text figures and 
                                         % text.
\@maxsep 20pt                            % The maximum of \floatsep, 
                                         % \textfloatsep and \intextsep (minus
                                         % the stretch and shrink).
%    TWO-COLUMN FLOATS IN TWO-COLUMN MODE:
\dblfloatsep 14pt plus 2pt minus 4pt     % Same as \floatsep for double-column
                                         % figures in two-column mode.
\dbltextfloatsep 20pt plus 2pt minus 4pt % \textfloatsep for double-column 
                                         % floats.
\@dblmaxsep 20pt                         % The maximum of \dblfloatsep and 
                                         % \dbltexfloatsep.

%  FOR FLOATS ON A SEPARATE FLOAT PAGE OR COLUMN:
%    ONE-COLUMN MODE OR SINGLE-COLUMN FLOATS IN TWO-COLUMN MODE:
\@fptop 0pt plus 1fil    % Stretch at top of float page/column. (Must be    
                         % 0pt plus ...)                                    
\@fpsep 10pt plus 2fil    % Space between floats on float page/column.       
\@fpbot 0pt plus 1fil    % Stretch at bottom of float page/column. (Must be 
                         % 0pt plus ... )                                   

%   DOUBLE-COLUMN FLOATS IN TWO-COLUMN MODE.
\@dblfptop 0pt plus 1fil % Stretch at top of float page. (Must be 0pt plus ...)
\@dblfpsep 10pt plus 2fil % Space between floats on float page.
\@dblfpbot 0pt plus 1fil % Stretch at bottom of float page. (Must be 
                         % 0pt plus ... )                                   
% MARGINAL NOTES:
%
\marginparpush 7pt       % Minimum vertical separation between two marginal 
                         % notes.


%  ****************************************
%  *           PARAGRAPHING               *
%  ****************************************
%
\parskip 0pt plus 1pt              % Extra vertical space between paragraphs.
\parindent 1.5em                   % Width of paragraph indentation.
%\topsep 10pt plus 4pt minus 6pt   % Extra vertical space, in addition to 
                                   % \parskip, added above and below list and
                                   % paragraphing environments.
\partopsep 3pt plus 2pt minus 2pt  % Extra vertical space, in addition to 
                                   % \parskip and \topsep, added when user
                                   % leaves blank line before environment.
%\itemsep 5pt plus 2.5pt minus 1pt % Extra vertical space, in addition to
                                   % \parskip, added between list items.
% See \@listI for values of \topsep and \itemsep
% (Change made 9 Jun 87)

% The following page-breaking penalties are defined

\@lowpenalty   51      % Produced by \nopagebreak[1] or \nolinebreak[1]
\@medpenalty  151      % Produced by \nopagebreak[2] or \nolinebreak[2]
\@highpenalty 301      % Produced by \nopagebreak[3] or \nolinebreak[3]

\@beginparpenalty -\@lowpenalty    % Before a list or paragraph environment.
\@endparpenalty   -\@lowpenalty    % After a list or paragraph environment.
\@itempenalty     -\@lowpenalty    % Between list items.

% \clubpenalty         % 'Club line'  at bottom of page.
% \widowpenalty        % 'Widow line' at top of page.
% \displaywidowpenalty % Math display widow line.
% \predisplaypenalty   % Breaking before a math display.
% \postdisplaypenalty  % Breaking after a math display.
% \interlinepenalty    % Breaking at a line within a paragraph.
% \brokenpenalty       % Breaking after a hyphenated line.


%    ****************************************
%    *        CHAPTERS AND SECTIONS         *
%    ****************************************
%
%%%%%%%%%%%%%%%%%%%%%%%%%%%%%%%%%%%%%%%%%%%%%%%%%%%%%%%%
%                        PART                          %
%%%%%%%%%%%%%%%%%%%%%%%%%%%%%%%%%%%%%%%%%%%%%%%%%%%%%%%%

\def\part{\cleardoublepage   % Starts new page.
   \thispagestyle{plain}     % Page style of part page is 'plain'
  \if@twocolumn              % IF two-column style
     \onecolumn              %  THEN \onecolumn
     \@tempswatrue           %       @tempswa := true
    \else \@tempswafalse     %  ELSE @tempswa := false
  \fi                        % FI
  \hbox{}\vfil               % Add fil glue to center title  
%%  \bgroup  \centering      % BEGIN centering %% Removed 19 Jan 88
  \secdef\@part\@spart}     


\def\@part[#1]#2{\ifnum \c@secnumdepth >-2\relax  % IF secnumdepth > -2
        \refstepcounter{part}                     %   THEN step part counter
        \addcontentsline{toc}{part}{\thepart      %        add toc line
        \hspace{1em}#1}\else                      %   ELSE add unnumbered line
        \addcontentsline{toc}{part}{#1}\fi        % FI
   \markboth{}{}
   {\centering                       % %% added 19 Jan 88
    \ifnum \c@secnumdepth >-2\relax  % IF secnumdepth > -2
      \huge\bf Part \thepart         %   THEN Print 'Part' and number
    \par                             %         in \huge bold.
    \vskip 20pt \fi                  %        Add space before title.
    \Huge \bf                        % FI
    #1\par}\@endpart}                % Print Title in \Huge bold.
    
    
% \@endpart finishes the part page
%
\def\@endpart{\vfil\newpage   % End page with 1fil glue.
   \if@twoside                % IF twoside printing
       \hbox{}                %   THEN Produce totally blank page
       \thispagestyle{empty}  
       \newpage        
   \fi                        % FI
   \if@tempswa                % IF @tempswa = true       
     \twocolumn               %   THEN \twocolumn 
   \fi}                       % FI

\def\@spart#1{{\centering      % %% added 19 Jan 88
   \Huge \bf                   % Print title in \huge boldface
   #1\par}\@endpart}



% Definition of \part moved to report.doc  19 Jan 88

% \@makechapterhead {TEXT} : Makes the heading for the \chapter command.
%

\def\@makechapterhead#1{             % Heading for \chapter command
%  \vspace*{50pt}                     % Space at top of text page.
  { \parindent 0pt \centering %\raggedright 
    \ifnum \c@secnumdepth >\m@ne     % IF secnumdepth > -1 THEN
      \Large\bf \@chapapp{} \thechapter % Print 'Chapter' and number.
      \par 
%      \vskip 20pt 
        \fi                  % Space between number and title.
    \Large \bf                        % Title.
    #1\par 
    \nobreak                         % TeX penalty to prevent page break.
    \vskip 20pt                      % Space between title and text.
  } }

% \@makeschapterhead {TEXT} : Makes the heading for the \chapter* command.
%

\def\@makeschapterhead#1{             % Heading for \chapter* command
%  \vspace*{50pt}                     % Space at top of page.
  { \parindent 0pt \centering %\raggedright 
    \Large \bf                        % Title.
    #1\par 
    \nobreak                         % TeX penalty to prevent page break.
    \vskip 20pt                      % Space between title and text.
  } }

% \secdef{UNSTARCMDS}{STARCMDS} :
%    When defining a \chapter or \section command without using
%    \@startsection, you can use \secdef as follows:
%       \def\chapter { ... \secdef \CMDA \CMDB }
%       \def\CMDA    [#1]#2{ ...   % Command to define \chapter[...]{...}
%       \def\CMDB    #1{ ...       % Command to define \chapter*{...}

\def\chapter{\clearpage      % Starts new page.
   \thispagestyle{empty}     % Page style of chapter page is 'botcent'
   \global\@topnum\z@        % Prevents figures from going at top of page.
   \@afterindenttrue         % Suppresses indent in first paragraph.  Change
   \secdef\@chapter\@schapter}   % to \@afterindenttrue to have indent.
\def\bibchapter{\clearpage      % Starts new page.
   \thispagestyle{plain}     % Page style of chapter page is 'botcent'
   \global\@topnum\z@        % Prevents figures from going at top of page.
   \@afterindenttrue         % Suppresses indent in first paragraph.  Change
   \secdef\@chapter\@schapter}   % to \@afterindenttrue to have indent.

\def\@chapter[#1]#2{\ifnum \c@secnumdepth >\m@ne
        \refstepcounter{chapter}
        \typeout{\@chapapp\space\thechapter.}
        \addcontentsline{toc}{chapter}{\protect
        \numberline{\thechapter}#1}\else
      \addcontentsline{toc}{chapter}{#1}\fi
   \chaptermark{#1}
   \addtocontents{lof}{\protect\addvspace{10pt}} % Adds between-chapter space
   \addtocontents{lot}{\protect\addvspace{10pt}} % to lists of figs & tables.
   \if@twocolumn                                 % Tests for two-column mode.  
           \@topnewpage[\@makechapterhead{#2}]  
     \else \@makechapterhead{#2}
           \@afterheading                  % Routine called after chapter and
     \fi}                                  % section heading.

\def\@schapter#1{\if@twocolumn \@topnewpage[\@makeschapterhead{#1}]
        \else \@makeschapterhead{#1} 
              \@afterheading\fi}

% \@startsection {NAME}{LEVEL}{INDENT}{BEFORESKIP}{AFTERSKIP}{STYLE} 
%            optional * [ALTHEADING]{HEADING}
%    Generic command to start a section.  
%    NAME       : e.g., 'subsection'
%    LEVEL      : a number, denoting depth of section -- e.g., chapter=1,
%                 section = 2, etc.  A section number will be printed if
%                 and only if LEVEL < or = the value of the secnumdepth
%                 counter.
%    INDENT     : Indentation of heading from left margin
%    BEFORESKIP : Absolute value = skip to leave above the heading.  
%                 If negative, then paragraph indent of text following 
%                 heading is suppressed.
%    AFTERSKIP  : if positive, then skip to leave below heading,
%                       else - skip to leave to right of run-in heading.
%    STYLE      : commands to set style
%  If '*' missing, then increments the counter.  If it is present, then
%  there should be no [ALTHEADING] argument.  A sectioning command
%  is normally defined to \@startsection + its first six arguments.

\def\section{\@startsection {section}{1}{\z@}{-3.5ex plus -1ex minus 
    -.2ex}{0.8ex plus .2ex}{\large\bf}}
\def\subsection{\@startsection{subsection}{2}{\z@}{-3.25ex plus -1ex minus 
   -.2ex}{1.5ex plus .2ex}{\normalsize\bf}}
\def\subsubsection{\@startsection{subsubsection}{3}{\z@}{-3.25ex plus 
   -1ex minus -.2ex}{1.5ex plus .2ex}{\normalsize\bf}}
\def\paragraph{\@startsection
     {paragraph}{4}{\z@}{3.25ex plus 1ex minus .2ex}{-1em}{\normalsize\bf}}
\def\subparagraph{\@startsection
     {subparagraph}{4}{\parindent}{3.25ex plus 1ex minus 
     .2ex}{-1em}{\normalsize\bf}}

% Default initializations of \...mark commands.  (See below for their
% us in defining page styles.
%

\def\chaptermark#1{}
% \def\sectionmark#1{}           % Preloaded definitions
% \def\subsectionmark#1{}
% \def\subsubsectionmark#1{}
% \def\paragraphmark#1{}
% \def\subparagraphmark#1{}

% The value of the counter secnumdepth gives the depth of the
% highest-level sectioning command that is to produce section numbers.
%

\setcounter{secnumdepth}{2}

% APPENDIX
%
% The \appendix command must do the following:
%    -- reset the chapter counter to zero
%    -- set \@chapapp to Appendix (for messages)
%    -- redefine the chapter counter to produce appendix numbers
%    -- reset the section counter to zero
%    -- redefine the \chapter command if appendix titles and headings are
%       to look different from chapter titles and headings.

\def\appendix{\par
  \setcounter{chapter}{0}
  \setcounter{section}{0}
  \def\@chapapp{Appendix}
  \def\thechapter{\Alph{chapter}}}


%    ****************************************
%    *                LISTS                 *
%    ****************************************
%

% The following commands are used to set the default values for the list
% environment's parameters. See the LaTeX manual for an explanation of
% the meanings of the parameters.  Defaults for the list environment are
% set as follows.  First, \rightmargin, \listparindent and \itemindent
% are set to 0pt.  Then, for a Kth level list, the command \@listK is
% called, where 'K' denotes 'i', 'ii', ... , 'vi'.  (I.e., \@listiii is
% called for a third-level list.)  By convention, \@listK should set
% \leftmargin to \leftmarginK.
%
% For efficiency, level-one list's values are defined at top level, and
% \@listi is defined to set only \leftmargin.

\leftmargini 2.5em
\leftmarginii 2.2em     % > \labelsep + width of '(m)'
\leftmarginiii 1.87em   % > \labelsep + width of 'vii.'
\leftmarginiv 1.7em     % > \labelsep + width of 'M.'
\leftmarginv 1em
\leftmarginvi 1em

\leftmargin\leftmargini
\labelsep .5em
\labelwidth\leftmargini\advance\labelwidth-\labelsep
%\parsep 5pt plus 2.5pt minus 1pt   %(Removed 9 Jun 87)

% \@listI defines top level and \@listi values of
% \leftmargin, \topsep, \parsep, and \itemsep
% (Added 9 Jun 87)
\def\@listI{\leftmargin\leftmargini \parsep 5pt plus 2.5pt minus 1pt%
\topsep 10pt plus 4pt minus 6pt%
\itemsep 5pt plus 2.5pt minus 1pt}

\let\@listi\@listI
\@listi 

\def\@listii{\leftmargin\leftmarginii
   \labelwidth\leftmarginii\advance\labelwidth-\labelsep
   \topsep 5pt plus 2.5pt minus 1pt
   \parsep 2.5pt plus 1pt minus 1pt
   \itemsep \parsep}

\def\@listiii{\leftmargin\leftmarginiii
    \labelwidth\leftmarginiii\advance\labelwidth-\labelsep
    \topsep 2.5pt plus 1pt minus 1pt 
    \parsep \z@ \partopsep 1pt plus 0pt minus 1pt
    \itemsep \topsep}

\def\@listiv{\leftmargin\leftmarginiv
     \labelwidth\leftmarginiv\advance\labelwidth-\labelsep}

\def\@listv{\leftmargin\leftmarginv
     \labelwidth\leftmarginv\advance\labelwidth-\labelsep}

\def\@listvi{\leftmargin\leftmarginvi
     \labelwidth\leftmarginvi\advance\labelwidth-\labelsep}

%    ****************************************
%    *      TABLES and TABBING              *
%    ****************************************
%

%The following commands create new environments that are slight variations
%on the standard environments for ``table'' and ``tabbing''.  Written by
%Angela M. Bellavance (9/16/1999)

\newenvironment{tablesingleindouble}{\begin{table}[htp]
		\begin{center}
		\normalsize}
	{\normalsizedb
		\end{center}
		\end{table}}

\newenvironment{tablesingleinsingle}{\begin{table}[htp]
		\begin{center}
		\normalsize}
	{\end{center}
		\end{table}}

\newenvironment{tabbingsingleindouble}{\normalsize
		\begin{tabbing}AAAAA\=AAAAA\=\kill\\}
	{\normalsizedb
		\end{tabbing}}

\newenvironment{tabbingsingleinsingle}{\begin{tabbing}AAAAA\=AAAAA\=\kill}
	{\end{tabbing}}
